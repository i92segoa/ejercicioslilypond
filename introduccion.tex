\section*{Introducción}

Esto no es un curso completo de LilyPond ni creo que sustituya a unas
clases directas; tan sólo pretende servir como material de apoyo. La
forma de utilizar estos ejercicios es bastante obvia si se examina uno
cualquiera de ellos:

\begin{enumerate}
\item Observar el modelo y buscar los elementos desconocidos.

\item Leer cuidadosamente el texto para aprender a realizar estos
  elementos.  Habrá que adaptarlos para recrear el modelo.

\item Tratar de tipografiar el modelo exactamente.

\end{enumerate}

El orden de los ejercicios es importante porque siempre los elementos
nuevos, necesarios para el modelo, están explicados en el mismo
apartado.

Dos son los complementos necesarios para aprender a tipografiar música
con LilyPond.  El primero es la documentación oficial que está en
lilypond.org; el segundo es la comunidad de usuarios, que está a su
disposición en http://lists.gnu.org/mailman/listinfo/lilypond-es
(lista en español) y en
http://lists.gnu.org/mailman/listinfo/lilypond-user (en inglés).

El estado actual de este documento es ``borrador incompleto''.  Dado
que los ejercicios son semanales, su número al término de la colección
será de 30, aproximadamente el número de semanas de un curso
académico.

%Antes de publicar este borrador en una forma definitiva, posiblemente
%haya que emplear fragmentos alternativos de ciertas obras protegidas.

Mi deseo es que esta pequeña recopilación sea de utilidad a alguien en
algún lugar.  Gracias por leerla.


