\section*{Introducción}

Esto no es un curso completo de LilyPond ni creo que sustituya a unas
clases directas; tan sólo pretende servir como material de apoyo. La
forma de utilizar estos ejercicios es bastante obvia si se examina uno
cualquiera de ellos:

\begin{enumerate}
\item Observar el modelo y buscar los elementos desconocidos.

\item Leer cuidadosamente el texto para aprender a realizar estos
  elementos.  Habrá que adaptarlos para recrear el modelo.

\item Tratar de tipografiar el modelo exactamente.

\end{enumerate}

El orden de los ejercicios es importante porque los elementos nuevos,
necesarios para el modelo, siempre están explicados en el mismo
apartado.

Dos son los complementos necesarios para aprender a tipografiar música
con LilyPond.  El primero es la documentación oficial que está en
lilypond.org; el segundo es la comunidad de usuarios, que está a su
disposición en la dirección de la lista en
español\footnote{http://lists.gnu.org/mailman/listinfo/lilypond-es} y
en la lista general en
inglés\footnote{http://lists.gnu.org/mailman/listinfo/lilypond-user}.

El estado actual de este documento es ``borrador incompleto''.  Dado
que los ejercicios son semanales, su número al término de la colección
será de 30, justo el número de semanas de un curso académico.  El
estilo de las indicaciones es muy escueto porque en su origen estas
``lecciones'' estaban pensadas para entregar a los alumnos en mano, en
una hoja o dos como máximo.  Dado que ésta ya no es la situación
actual, la vía está abierta a un estilo más ``de libro'' que suavice
la severidad que el escrito presenta por el momento.

\section*{LilyPond ¿por qué?}

Todo el que aún no ha quedado atrapado en el irresistible atractivo de
LilyPond como idea y como solución al problema de la
tipografía\footnote{El llamado ``problema de la tipografía musical''
  se expresa como el dilema a que se enfrenta un usuario cuando tiene
  que elegir entre usar copias no autorizadas de software privativo o
  pagar el alto precio (en dinero y en libertades) que cuesta una
  licencia. Como probablemente ya sepa, LilyPond ofrece la tercera
  vía: la del software libre.}, se preguntará, y con razón, ¿por qué
embarcarse en la incómoda y tediosa tarea de aprender un
\emph{lenguaje de programación} estricto, poco intuitivo, no visual,
lleno de reglas imposibles de recordar, pudiendo simplemente
\emph{tomar prestada} una copia de \emph{ese otro programa que todo el
  mundo usa}? La respuesta es simple: todos estos inconvenientes son
relativos y el proceso en su conjunto merece la pena.  No aburriré al
lector con la lista de ventajas que el software libre ofrece frente a
los productos comerciales privativos y cerrados; Internet ofrece
amplísima información sobre este punto.  Es cierto que la adopción de
LilyPond supone un cambio de paradigma, bastante radical, respecto a
la típica aplicación que nos permite editar partituras colocando notas
con el ratón sobre un pentagrama vacío. Sin embargo, al mismo tiempo
que reconocemos que llegar a dominar este gran programa requiere un
cierto esfuerzo, debemos resaltar la gran potencia de sus conceptos
principales:

\begin{itemize}
\item documentos de texto que podemos leer sin el programa
\item la robustez de no depender de hacer \emph{clic} en el lugar equivocado
\item la posibilidad de reutilizar el material de forma consistente
\item gratuidad y disponibilidad del programa y su documentación
\item calidad final muy elevada, sin compromiso
\item adaptación automática del material a la página
\item solución automática de las colisiones entre elementos
\item escalabilidad: idoneidad para proyectos de todos los tamaños
\end{itemize}

%Antes de publicar este borrador en una forma definitiva, posiblemente
%haya que emplear fragmentos alternativos de ciertas obras protegidas.

LilyPond está hecho por un grupo de desarrolladores en su tiempo
libre. Es un ejemplo de desarrollo colaborativo. Pertenecer a este
equipo me llena de orgullo.  Mi deseo es que esta pequeña recopilación
sea de utilidad a alguien en algún lugar.  Gracias por leerla.


\section*{Licencia}

Esta recopilación de ejercicios semanales ha nacido de la necesidad de
tener un material de trabajo para mis clases de Edición de partituras
en el Conservatorio Superior de Badajoz.  Esta asignatura optativa ya
no existe, pero en cambio la comunidad hispanohablante de usuarios de
LilyPond ha ido creciendo poco a poco y quizá el uso de estos apuntes
encuentre un hueco en un ámbito menos local.

Este documento está publicado bajo la licencia (cc)(by)(sa) Creative
Commons Atribución - Compartir igual 3.0 España.  Algunos derechos
reservados.

Usted es libre de copiar, distribuir y comunicar públicamente la obra.
Puede hacer obras derivadas y distribuirlas, siempre que añada el
siguiente texto:

\begin{quote}
\emph{Basado en un trabajo anterior de Francisco Vila,
  \texttt{http://www.paconet.org} }
\end{quote}

visible en una de las dos primeras páginas, o una de las dos últimas.

La obra derivada debe distribuirse bajo la misma licencia.

El texto completo de esta licencia está en
http://creativecommons.org/licenses/by-sa/2.0/es/legalcode.es



