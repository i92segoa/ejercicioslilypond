\section{Ornamentos barrocos: Aria de las Variaciones Goldberg.}


\subsection{Modelo}

El siguiente fragmento es el comienzo del Aria de las ``Variaciones
Goldberg'' BWV 988 de Bach.  Contiene abundantes apoyaturas y
ornamentos barrocos, y nos servirá para introducir las notas de adorno
en general.

\bigskip

\begin[staffsize=17.5,line-width=17\cm]{lilypond}
\relative c''' {
	\key g \major \time 3/4
        g4 g( a8.\prallmordent) b16
        a8 \appoggiatura g16 fis8 \appoggiatura e16 d2
        g,4\prallmordent g4.\downprall fis16 g
        a32( g fis16) g32( fis e16) \appoggiatura e8 d2
        d'4 d( e8.\prallmordent) f16
        e8 \appoggiatura d16 c8 \appoggiatura b16
        a4.
        fis'8 \turn
        g32( fis16.) a32( g16.) fis32( e16.) d32( c16.)
        \appoggiatura c8 a'8. c,16
        b32( g16.) fis8
        \appoggiatura fis8 g2\prallmordent
}
\end{lilypond}

\subsection{Notas de adorno}
Para conseguir un mordente de una nota (que está tachado por una línea
inclinada y se ejecuta rápidamente) o una apoyatura (que tiene el
valor que representa) empleamos las instrucciones \verb+\appoggiatura+
y \verb+\acciaccatura+, respectivamente, como prefijos:

\begin[relative=2,verbatim,staffsize=17.5]{lilypond}
g2 \acciaccatura b8 a8 g a b
\appoggiatura gis4 a2 r
\end{lilypond}

Estas notas se dibujan con una ligadura que las une a la nota
principal.  Al utilizar \verb+\grace+ como prefijo de una expresión
obtenemos mordentes de varias notas, pero es necesario escribir la
ligadura explícitamente:

\begin[relative=0,verbatim,staffsize=17.5]{lilypond}
\clef bass
\grace { a32[( c e] } a8) a a a
\end{lilypond}


\subsection{Algunas abreviaturas y otros ornamentos barrocos}

Nuestro modelo no utiliza acciaccaturas pero sí emplea grupos
abreviados de notas de adorno muy utilizados en el barroco; las
palabras clave se emplean como sufijos, a modo de articulaciones, pero
sin el guión de éstas.  Usaremos \verb+\prallmordent+ para el
semitrino largo con resolución descendente, \verb+\downprall+ para el
semitrino con preparación descendente y \verb+\turn+ para el grupeto
circular.

\begin[relative=3,verbatim,staffsize=17.5]{lilypond}
a2 \prallmordent
g  \downprall
f1 \turn
\end{lilypond}


