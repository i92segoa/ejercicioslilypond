\includepdf{holst-marte}

\section{Un gran ejemplo orquestal: Marte, de Holst (1)}


\subsection{Modelo}

Para esta gran partitura orquestal perteneciente a la suite ``Los
planetas'' de Gustav Holst, vamos a trabajar en varias fases. Por
ahora añadiremos dos instrumentos: los timbales y el gong. El
ejemplo contiene repeticiones de trémolo, silencios de compás
completo en 5/4 y una partitura completa dentro de un marcado.

\bigskip

% Aumentar la separación entre sistemas
\def\betweenLilyPondSystem#1{\vspace{0.4cm}\linebreak}

\begin[line-width=17cm]{lilypond}

\version "2.13.18"

           timpaniI = \relative g, { \clef bass
                         \key c \major
   \times 2/3 { g8\p ^"wooden sticks" g g }  g4 g g8 g g4
   \times 2/3 { g8 g g } g4 g g8 g g4
   \times 2/3 { g8 g g } g4 g g8 g g4
   \times 2/3 { g8 \< g g } g4 g g8 g g4\!
   \times 2/3 { g8 \> g g } g4 g g8 g g4\! }

           timpaniII = { \clef bass
                         \key c \major
	   R1*5/4 R1*5/4 R1*5/4 R1*5/4 R1*5/4 }

	gong = { g2.:32 \pp  g2:32 g2.:32 g2:32 g2.:32 g2:32 g2.\< :32 g2\!:32 g2.\>:32 g2:32 \! }


incipitTimpaniGroup = \markup {
	\score{
		 \new PianoStaff  <<
                    \set PianoStaff.instrumentName= \markup {
				\center-column {"6 Timpani" "(two players)"}
			}
			\new Staff { \set Staff.instrumentName = "I"
				\clef bass
				\time 3/2
				\cadenzaOn g,2 d2 bes,2
			}
			\new Staff { \set Staff.instrumentName = "II"
				\clef bass
				\time 3/2
				\cadenzaOn c2 es2 a,2
			}
		>>

	\layout {
		\context { \Staff
                         \remove "Time_signature_engraver"
                         \remove "Clef_engraver"
		}
		line-width=2.5\cm
                indent=1\cm
		%margin-left=0\cm
	} %layout
  } %score
} %markup

        #(set-global-staff-size 15.5)
	% #(set-default-paper-size "a3")

   \paper {  ragged-right=##f
             system-count=1
             }


%\header { title =  "I. Mars, the Bringer of War"
% title = \markup { \fontsize #6 { \smallCaps {  "I. Mars, " } "the Bringer of War" } }
	%	copyright = "Francisco Vila, sobre un trabajo de Guadalupe Cuevas Piñero"
%}


\score {
    \new StaffGroup <<   %\tempo "Allegro"       % main
	\time 5/4

    \new PianoStaff \with { systemStartDelimiter=#'SystemStartBar } <<  %timpani
         \set PianoStaff.instrumentName =
		\markup {
			\incipitTimpaniGroup
		}
	    \new Staff  {  \timpaniI }
	    \new Staff  { \timpaniII } >>

    \new RhythmicStaff		%gong
	{ \set Staff.instrumentName = "Gong"
	\gong }

>> %main
   \layout { indent=3.5\cm
	   \context { \Staff
%                        \override InstrumentName #'padding = #-50
%              \override VerticalAxisGroup #'minimum-Y-extent = #'(-3 . 3)
%	       \override instrumentName #'font-size = #8.0
	   }
   }%layout

} %score

\end{lilypond}

\subsection{Preparación del incipit: partitura dentro de un marcado}

A veces se denomina ``incipit'' a una pequeña partitura previa al
sistema que contiene la música real, y que tiene una finalidad
informativa.  La parte de timbales contiene un incipit que
prepararemos como una partitura normal y luego la insertaremos
dentro del nombre del sistema de piano de los timbales.

Haremos un elemento de marcado que contendrá una partitura
completa en la forma de un bloque \verb+\score+, y que deberá
incluir un bloque \verb+\layout{}+. Este elemento de marcado se
asignará a una variable que luego podrá utilizarse en cualquier
lugar donde se admita un marcado.  Aquí vemos un ejemplo sencillo:

\begin[verbatim]{lilypond}
partiturita=\markup{
  \score {
    \new RhythmicStaff { \time 2/4
      \times 2/3 { c'4 c'8 }
    }
    \layout{
      \context { \RhythmicStaff
               \remove "Time_signature_engraver"
      }
    }
  }
}

\new Staff \relative c' {
  c8_\partiturita c c c c c c c
}
\end{lilypond}

\subsection{Cadenza}

El fragmento del incipit aparece sin compasear, lo haremos de la
siguiente manera:

\begin[verbatim,relative=2]{lilypond}
\new Staff { \cadenzaOn c2 d e }
\end{lilypond}

Además no tiene indicación de compás ni clave, para ello hay que
eliminar los grabadores correspondientes del contexto de Staff: véase
el apartado \ref{consists} de la página \pageref{consists}.

\subsection{Columnas en el marcado}

Son muy numerosas las instrucciones de marcado y es recomendable
consultar la documentación oficial para aprender a usarlas.  La
instrucción de marcado que utilizaremos para el nombre de dos
líneas del sistema de timbales (en sustitución de una simple
línea) es \verb+\center-column+.  Esta instrucción admite un
bloque con varias expresiones textuales, que colocará centradas y
superpuestas:

\begin[verbatim]{lilypond}
\markup{
  \center-column {"6 Timpani" "(two players)"}
}

\end{lilypond}

Después emplearemos este marcado como el valor de la propiedad
\verb+PianoStaff.instrumentName+ de nuestra pequeña partitura.

Así pues, tenemos un marcado con dos líneas que es el nombre del
sistema de piano de una pequeña partitura, que a su vez está
dentro de un marcado que es el nombre del sistema de piano de los
timbales de la partitura grande.

\subsection{Figuras escaladas}

El escalado de valores se efectúa gracias al asterisco como
símbolo de multiplicación.  De esta forma podemos escribir una
figura con el aspecto de un valor dado, pero con la duración
reducida o ampliada según un factor.

Lo usaremos para rellenar compases de 5/4 con silencios de compás
completo, que se escriben mediante la letra R~mayúscula.  Si además
utilizamos otro factor de multiplicación entero, aparecerá
repetido el compás de silencio.

\begin[verbatim,relative=2]{lilypond}
\time 5/4 R1*5/4 c2. c2 R1*5/4*3
\end{lilypond}


\subsection{Repeticiones de trémolo}

Los trémolos, usados en el golpe repetido del gong, se expresan
mediante el símbolo de dos puntos seguido del valor de repetición.

\begin[verbatim,relative=2]{lilypond}
c2:8 c2:16 c2:32
\end{lilypond}


\subsection{Notas}

\begin{itemize}
\item Consulte en el apartado \ref{rhythmicstaff} de la página
\pageref{rhythmicstaff} la forma de crear un pentagrama de una sola
línea para percusión.

\item Puede probar muchas combinaciones para las variables del
  bloque \verb+\layout+ de la partitura del incipit, pero éstas
  pueden servir de orientación:

\begin{verbatim}
  line-width=2.5\cm
  indent=1\cm
\end{verbatim}

\item Suprimiremos las llaves del sistema de piano de los timbales
  con la asignación

\begin{verbatim}
\new PianoStaff \with { systemStartDelimiter=#'SystemStartBar }
  << ...
\end{verbatim}


\end{itemize}
