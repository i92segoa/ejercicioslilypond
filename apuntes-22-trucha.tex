\section{Trabajo colaborativo: quinteto ``La trucha'' de Schubert.}


\subsection{Modelo}

He aquí el comienzo del Tema con variaciones del quinteto D.667 para
piano, violín, viola, violoncello y contrabajo, ``La trucha'', de
F. Schubert.  La realización de este ejercicio puede hacerse en grupos
de dos a cuatro personas.  Aprenderemos a incluir el contenido de
distintos documentos dentro de uno solo, y a variar el tamaño de los
pentagramas.


\bigskip

\begin[staffsize=15,
line-width=17\cm,
indent=0
]{lilypond}
#(set-global-staff-size 17)


global = { \key d \major \time 2/4 \tempo "Andantino" \partial 8 }


violin = \relative c'' { \global a8 \pp
	d8.-. ( d16-. fis8-. fis-.)
	d4( a)
	a8.. a32 e'16.( d32 cis16. b32)
	a4. a8
	d8.-. ( d16-. fis8-. fis-.)
	d4( a8) d
	cis8[( \grace{ d16[ cis] } b16.) cis32] d8(-> gis,)
	a4. a8
	}

viola = \relative c' { \global \clef alto r8 \pp
	<a fis'>4-.( q8-. q-.)
	fis'2
	<g a>4-. ( q8-. q-.)
	q8-. e( fis g)
	<a, fis'>4-.( q8-. q-.)
	fis'4. fis8
	e4.  e8
	e8-. g( fis e)
	}

cello = \relative c' { \global \clef bass r8 ^\pp
	d4-.( d8-. d-.)
	a8( b16 cis d4)
	cis4 cis16.( d32 e16. d32)
	cis8-. cis( d e)
	d4-.( d8-. d-.)
	a8( b16. cis32 d8 a)
	a8( gis16.) a32 b8( d)
	cis8-. e( d cis)
	}

contrabajo = \relative c {
\global \clef bass r8 \pp
	d4-.( d8-. d-.)
	d2
	a'4-.( a8-. a-.)
	a4 r
	d,4-.( d8-. d-.)
	d4. d8
	e4. e8 a,4 r
	}

pianoManoDerecha = \relative c'' { \global \clef treble r8
	R2*8
	}

pianoManoIzquierda = \relative c'' { \global \clef bass r8
	R2*8
	}

<<
	\new Staff \with { fontSize = #-3
		\override StaffSymbol #'staff-space = #(magstep -3)
	%	\override StaffSymbol #'thickness = #(magstep -3)
	}
		{  \violin }
	\new Staff \with { fontSize = #-3
		\override StaffSymbol #'staff-space = #(magstep -3)
	%	\override StaffSymbol #'thickness = #(magstep -3)
	}
		{ \viola }
	\new Staff \with { fontSize = #-3
		\override StaffSymbol #'staff-space = #(magstep -3)
	%	\override StaffSymbol #'thickness = #(magstep -3)
	}
		{ << \cello \\ \contrabajo >> }
	\new PianoStaff <<
	  \new Staff { \pianoManoDerecha }
	  \new Staff { \pianoManoIzquierda }
	>>
>>


\paper { indent=0 system-count =1 }

\end{lilypond}


\subsection{Inclusión de documentos}
\label{include}

La inclusión de archivos externos nos será de utilidad para mantener
la independencia entre el contenido musical y la estructura de una
partitura.  Mediante esta técnica podemos crear documentos que
dependen de otros archivos, quizá realizados por otras personas.  De
esa manera, un equipo puede trabajar de forma colaborativa sobre un
proyecto común.

La inclusión de archivos externos funciona de la siguiente manera:
supongamos que el archivo \verb+violin.ly+ contiene solamente lo
siguiente:

\begin{verbatim}
violin = \relative c'' { \key d \major \time 2/4 \partial 8
  a8 \pp d8.-. ( d16-. fis8-. fis-.)
  }
\end{verbatim}

Un archivo diferente, llamado \verb+parte-violin.ly+, puede incluir a
éste por su nombre:

\begin{verbatim}
\include "violin.ly"

\score {
   \new Staff { \violin }
}
\end{verbatim}

Entonces, todo el contenido del archivo especificado se inserta en
sustitución de la instrucción \verb+\include+, cuando se procesa el
archivo \verb+parte-violin.ly+.

\begin[relative=2,staffsize=17.5]{lilypond}
\key d \major \time 2/4 \partial 8
a8 \pp d8.-. ( d16-. fis8-. fis-.)
\end{lilypond}

Otro documento puede contener una estructura distinta e incluir el
mismo archivo que contiene la música, por ejemplo
\verb+piano-general.ly+ podría ser algo así:

\begin{verbatim}
\include "violin.ly"
\include "viola.ly"
\include "cello.ly"
\include "contrabajo.ly"
\include "piano.ly"

<<
  \new Staff { \violin }
  \new Staff { \viola }
  \new Staff { << \cello \\ \contrabajo >> }
  \new PianoStaff <<
    \new Staff { \pianoManoDerecha }
    \new Staff { \pianoManoIzquierda }
  >>
>>
\end{verbatim}

De esa forma estamos produciendo la partitura general del pianista, y
la particella de los instrumentos, a partir de la misma fuente y en
archivos independientes.  Para este ejercicio pediremos a cada miembro
de un grupo, que se encargue de elaborar una parte de los
instrumentos, en archivos separados, tales que cuando se inserten en
la estructura que hemos dado arriba, produzcan la partitura general de
piano.  Para probar el resultado de su trabajo parcial, puede crear un
archivo de particella que incluya solamente su parte.

Dado que el nombre del archivo tiene que coincidir exactamente con el
argumento de la instrucción \verb+\include+, se recomienda elegirlo de
tal forma que contenga solamente letras minúsculas, y no espacios o
vocales acentuadas.

\subsection{Tamaño de la partitura y de los pentagramas}
\label{tamano-global}

El tamaño normal de una partitura de LilyPond es ``20'', pero puede
cambiarse de forma global con una instrucción del lenguaje interno
``Scheme'' \verb+set-global-staff-size+, como en este ejemplo que
produce música en miniatura, de tamaño 10:


\begin[verbatim,relative=2]{lilypond}
#(set-global-staff-size 10)
\key d \major \time 2/4 \partial 8
a8 \pp d8.-. ( d16-. fis8-. fis-.)
\end{lilypond}

Las instrucciones del lenguaje Scheme van precedidas del carácter de
almohadilla \verb+#+.  Esta instrucción tiene un efecto general sobre
la partitura; para pentagramas sueltos debe especificarse dentro de
una cláusula \verb+\with{}+ al crear el contexto, con las siguientes
instrucciones:

\begin{verbatim}
\new Staff \with {
  fontSize = #-3
  \override StaffSymbol #'staff-space = #(magstep -3)
}
\end{verbatim}

Aquí hemos usado la sobreescritura de propiedades que se estudiará
con detalle en un ejercicio posterior (véase el apartado
\ref{override}, pág. \pageref{override}).  La primera instrucción
reduce el tamaño de la fuente de tipografía musical y la segunda
el tamaño del pentagrama.  Una especificación de -3 para el
tamaño, en ambos casos, establece una reducción como la que hemos
aplicado en el ejemplo de Schubert para los instrumentos de
cuerda.
