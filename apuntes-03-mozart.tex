\section{\emph{Serenata nocturna}, de Mozart}
\subsection{Modelo}

Esta vez utilizaremos como modelo el conocido comienzo de la
\emph{Serenata nocturna} de Mozart.

\bigskip

\begin[relative=2,staffsize=18,fragment]{lilypond}
\key g \major \tempo "Allegro"
<g d' g>4 \f r8 d' g4 r8 d
g8 d g b d4 r
c4 r8 a c4 r8 a
c8 a fis a d,4 r4 \bar "||"
\end{lilypond}

Este fragmento contiene los elementos nuevos de notación musical
que iremos revisando en los apartados siguientes.

\subsection{Tonalidad}
Podemos definir la armadura de la tonalidad mediante la
instrucción \verb+\key+ seguida del nombre de una nota y de la
instrucción \verb+\major+ (para mayor) o \verb+\minor+ (para
menor):

\begin[verbatim,relative=2,staffsize=13]{lilypond}
\key f \minor
f g a
\end{lilypond}

Observemos que '\verb+a+' produce un La natural aunque la armadura
es de Fa menor.

\subsection{Indicación de tempo}

Mediante \verb+\tempo+ seguido de una expresión entre comillas,
colocamos en el lugar adecuado una indicación de tempo.

\begin[verbatim,relative=1,staffsize=13,fragment]{lilypond}
\tempo "Allegro con fuoco" c1 c c
\end{lilypond}

\subsection{Acordes}

Los acordes se introducen escribiendo las notas entre ángulos, en
cualquier orden.  La duración se coloca después del ángulo de
cierre.
\begin[verbatim,relative=1,staffsize=13,fragment]{lilypond}
< c e g >2
\end{lilypond}

%\begin{minipage}{\textwidth}
El modo relativo funciona dentro de un acorde, pero es la primera
nota del acorde la que se tiene en cuenta para las notas que
siguen.  El último Do del siguiente ejemplo no es relativo a la
tercera nota del acorde, sino a la primera.

\begin[verbatim,relative=1,staffsize=13,fragment]{lilypond}
< c e g > c
\end{lilypond}
%\end{minipage}

\newpage
\subsection{Notas alteradas}

El sostenido se obtiene añadiendo ``\verb+is+'' al nombre de la nota, y el bemol añadiendo ``\verb+es+'':

\begin[verbatim,relative=2,staffsize=13,fragment]{lilypond}
c cis a aes
\end{lilypond}


En LilyPond se introduce siempre la altura real de las notas,
naturales o alteradas, aunque no presenten una alteración
accidental.  Por ejemplo, en la tonalidad de Fa mayor es necesario
escribir ``\verb+bes+'' para obtener el Si bemol, aunque la
armadura ya contiene esta alteración.

\begin[verbatim,relative=2,staffsize=13,fragment]{lilypond}
\key f \major bes1
\end{lilypond}

\subsection{Matices}

Para imprimir una indicación de dinámica podemos escribir
\verb+\p+, \verb+\mf+, \verb+\f+, etc. después de una nota.

\begin[verbatim,relative=2,staffsize=13,fragment]{lilypond}
c2 \pp c \mf  c \f  c \ff
\end{lilypond}

\subsection{Silencios}

Los silencios se escriben como si fueran notas con el nombre
'\verb+r+':

\begin[verbatim,relative=2,staffsize=13,fragment]{lilypond}
r2 r4 r8 r16
\end{lilypond}




\subsection{Barra doble}

La barra introducida mediante \verb+\bar "||"+ produce una doble
barra simple, distinta a la doble barra final que se obtiene
mediante \verb+\bar "|."+

\begin[verbatim,relative=1,staffsize=13,fragment]{lilypond}
c1 \bar "||"
\end{lilypond}

En el teclado español, el signo de barra '|' está en AltGr + 1.

