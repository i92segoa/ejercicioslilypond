% \version "2.17.0"

\section{Polifonía en un pentagrama}


\subsection{Modelo}

En la jerga de LilyPond, ``polifonía'' significa más de una voz en el mismo pentagrama.

El siguiente ejemplo puede obtenerse a partir del ejercicio anterior
sin alterar la música:

\bigskip

\begin[staffsize=17.5,line-width=17\cm]{lilypond}
\score{
\new Staff <<

 \relative c''' {
\time 12/8 \key f\major
    c4. ~ c8 b a g4. ~ g8 f e |
    d b c f4. ~ f8 e g c4 es,8

}
\\
 \relative c'' {
\time 12/8 \key f\major
    e16( d )e c g c f( e )f d b d g( f )g e c e a( g )a f g a |
    b,8 g' c, ~ c a b!-\trill c16( b )c g e g f( g a bes )c a }

>>
\layout{ system-count=1 }
}

\end{lilypond}


\subsection{La construcción de voces polifónicas}

Supongamos que tenemos música simultánea en dos pentagramas:

\begin[verbatim,relative=2,staffsize=17.5]{lilypond}
<<
  \new Staff { e4 f g2 e4 f g2 g8 a g f e4 c4 g'8 a g f e4 c4 }
  \new Staff { c,4 d e c c d e c e f g2 e4 f g2 }
>>
\end{lilypond}

\bigskip

La construcción

\begin{quote}
  \verb+<< { música } \\ { música } >>+
\end{quote}

permite crear dos voces dentro de un pentagrama; partiendo del ejemplo
anterior es fácil hacer lo siguiente:

\begin[verbatim,relative=2,staffsize=17.5]{lilypond}
\new Staff
  <<
    { e4 f g2 e4 f g2 g8 a g f e4 c4 g'8 a g f e4 c4 }
      \\
    { c,4 d e c c d e c e f g2 e4 f g2 }
  >>
\end{lilypond}


La primera expresión es la voz 1 y tiene las plicas hacia arriba; la
segunda expresión es la voz 2 y tiene las plicas hacia abajo.

