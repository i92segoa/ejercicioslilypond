\section{Lección cero: teclear código de LilyPond}

\subsection{Modelo}
\verb+\\\\\\\\\\{{{{{{{{{}}}}}}}}}''''''''''''''~~~~~~~~~+

\verb+\{}\{}\{}'~'~'~'~'~+

\subsection{Las teclas más usuales no son nada usuales}

Hemos observado que el lenguaje de entrada de LilyPond es bastante
ajeno al uso que los hispanohablantes hacemos del teclado. Con esta
``lección cero'' queremos allanar el camino a la utilización cómoda de
estas teclas especiales que son tan frecuentes en el código de la
música.

Una sencillísima partitura como la siguiente

\begin{lilypond}[verbatim]
\relative c' {
  \key c \minor
  g'1 ~ g
   
}
\end{lilypond}

requiere los cinco caracteres que más nos preocupan: la barra
invertida (\verb+\+), las llaves curvas (\verb+{+,\verb+}+), el
apóstrofo (\verb+'+) y la tilde curva(\verb+~+).

La barra invertida es el prefijo de todas las instrucciones del
lenguaje; las llaves encierran a la partitura completa y a todas las
expresiones musicales; el apóstrofo, en modo relativo denota un salto
mayor que una cuarta respecto a la nota anterior, y una octava más por
cada nuevo apóstrofo, y en modo absoluto designa la octava 4, desde el
Do4 o \emph{Do central} hasta el Si de la tercera línea, en clave de
Sol. La tilde curva se utiliza para la ligadura de unión.

Por ello proponemos reproducir este modelo tan aparentemente absurdo,
pero no una vez, sino varias. Repítalo hasta que pueda hacerlo con los
ojos cerrados y sólo entonces podrá escribir en lilypond como si
escribiera en español. Recuerde: en los teclados españoles,

\begin{itemize}
\item La barra invertida está en AltGr º, a la izquierda de la cifra 1
\item Las llaves curvas están a la derecha de la 'Ñ' y también con AltGr
\item El apóstrofo está a la derecha de la cifra 0 (directamente, sin AltGr ni tecla de cambio alguna)
\item La tilde curva está en AltGr 4
\end{itemize}

\subsection{Otros caracteres extraños}

Aún nos encontraremos en nuestras partituras con la necesidad de
emplear algún que otro carácter poco común. La lista no definitiva es
la siguiente:

\begin{itemize}
\item La almohadilla (\verb+#+) es necesaria para las expresiones en
  lenguaje Scheme. Está en AltGr~3.
\item El porcentaje (\verb+%+) sirve para hacer comentarios de una
  línea, código libre informativo o anotaciones que no se procesan
  como música. También permite convertir un bloque de código en un
  comentario, encerrándolo entre \verb+%{+ y \verb+%}+. Está en la
  cifra~5, con Mayús.
\item La barra vertical se usa para la comprobación de líneas de
  compás y forma parte de la indicación \verb+\bar "|."+ que produce
  una doble barra final. La tenemos en AltGr~1.
\item El signo de admiración (\verb+!+) es necesario para forzar las
  alteraciones. Lo podemos obtener en la tecla de la cifra~1, con
  Mayús.
\item El signo de interrogación (\verb+?+) produce el paréntesis sobre
  las alteraciones de cortesía. Está en la misma tecla que el
  apóstrofo, a la derecha de la cifra~0, pero con Mayús.
\item La barra inclinada (\verb+/+) es el signo de división, muy usado
  en la indicación de compás como \verb+\time 2/4+. Lo tenemos en la
  tecla del~7, con Mayús.
\item Los corchetes rectos (\verb+[+ y \verb+]+) se emplean en las
  barras de corchea manuales, marcando las notas inicial y final
  \emph{por la derecha}, así: \verb+g[ g g]+. Estas teclas están a la
  derecha de la~P, con AltGr.
\end{itemize}

Para practicar estas últimas teclas, le proponemos el ejercicio siguiente.

\verb+#####%%%%%|||||!!!!????////[[[[]]]]+

\verb+##%%||!!??//[[]]#%|!?[]#%|!?[]+
