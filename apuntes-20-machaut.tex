%\setcounter{section}{19} %para 20 Machaut


\section{Grabadores. Misa de Notre Dame, de Machaut}


\subsection{Modelo}


Esta transcripción moderna del ``Ite missa est'' de la Misa de Notre
Dame, de Guillaume de Machaut (s.XIV) contiene indicaciones de
tesitura y omite la indicación de compás.  Esto se hace añadiendo o
retirando los complementos grabadores o ``plug-ins'' encargados de
hacer esta tarea, en todos los contextos de pentagrama.  Además, la
letra contiene apenas dos sílabas, por lo que es necesario saltar
muchas notas de una sílaba a la siguiente.  También utiliza un sistema
especial para coro, sin líneas divisorias entre los pentagramas.

\bigskip

\begin[staffsize=15,
line-width=17\cm,
indent=0
]{lilypond}
#(set-global-staff-size 14)

%{

\header { title = "Ite missa est"
	subtitle = \markup {
		\score { { c'8 c c c c c c c c c c c c } \layout { indent=0  \context { \Staff \remove "Time_signature_engraver"
	\remove Bar_engraver } } }
	}


}



%}

triplum = \relative c'' { \time 3/2 c1.
g1.
e2 bes'8 a g f e4 g
f1.
r4 g e  r4 r8  a4 g8
f4. a4 c a8 a g a f
e2 fis4. g4 a fis8
g1.
}

motetus = \relative c' { f1.
d1.
c2 d4 f e d
c1.
e2 r d
f4 c2  d4 r8  c4 b8
cis2. d4 r8  d cis d
e1.
}

tenor = \relative c { \clef "G_8" f1.
g1.
a1.
f1.
g1.
r2 f f
e2 d1(
c1.)
}

contratenor = \relative c' { \clef "G_8" c1.
r2 bes1
a2 d, e
f2 a1
g4 c,2 d e4
f2 r g
a2 b1
c1.
}

\score {
\new ChoirStaff <<
\new Staff { \triplum } \addlyrics { De -- \repeat unfold 26 { \skip 8 } o }
\new Staff { \motetus } \addlyrics { De -- \repeat unfold 14 { \skip 8 } o  gra- }
\new Staff { \tenor } \addlyrics { De -- \repeat unfold 7 { \skip 8 } o }
\new Staff { \contratenor } \addlyrics { De -- \repeat unfold 14 { \skip 8 } o }
>>

\layout {
   \context { \Staff \remove Time_signature_engraver
	            \consists Ambitus_engraver }
           }
        }

\end{lilypond}


\subsection{Skips}

Cuando se escribe la letra de un pasaje melismático (que tiene muchas
notas para una sílaba), es frecuente recurrir al empleo del salto o
``skip'', que se inserta en la letra una vez por cada nota que
queremos saltar.  La instrucción requiere una duración como argumento,
aunque esta duración se desprecia y no influye en el resultado.

\begin[verbatim,relative=1,staffsize=17.5]{lilypond}
{ c d e f } \addlyrics { De -- \skip 4 \skip 4 o }
\end{lilypond}

Se recomienda el empleo de repeticiones desplegadas para insertar
múltiples saltos seguidos.
\label{unfold}
\begin[verbatim,relative=1,staffsize=17.5]{lilypond}
{ \time 3/4 f16 g f e f e d e d c b d c4 }
\addlyrics { Pen -- \repeat unfold 11  \skip 4  sier! }
\end{lilypond}


\subsection{Añadir y quitar grabadores}

Existen más de 120 grabadores o complementos distintos, que se agrupan
en contextos como los conocidos \verb+Staff+ o \verb+Voice+.  Si un
grabador se suprime de un contexto, no se trazan aquellos elementos
que estaban a su cargo; si se añade un grabador nuevo a un contexto,
podrán aparecer elementos nuevos.

Los contextos se añaden o se suprimen con las instrucciones
\verb+\consists+ y \verb+\remove+.  Existen dos formas de hacerlo:
para cada contexto uno a uno, o para todos los contextos de una
clase al mismo tiempo.
\label{consists}

\textbf{1.} Para añadir un grabador a todos los contextos de una
clase, abrimos un bloque \verb+\layout+ fuera de la expresión
principal de nuestra partitura.  La sintaxis para añadir un
grabador (aquí el grabador \verb+Ambitus_engraver+ que se encarga
de la indicación de tesitura) es la siguiente:


\begin[verbatim,staffsize=15]{lilypond}
{ c'2 c'' }
\layout {
  \context { \Staff \consists Ambitus_engraver }
}

\end{lilypond}

Se pueden introducir varias instrucciones \verb+\consists+ o
\verb+\remove+ dentro del mismo bloque \verb+\context+.

\textbf{2.} Para gestionar los grabadores en un solo contexto,
introducimos las instrucciones \verb+\consists+ y \verb+\remove+
en el momento de la creación del contexto, dentro de un bloque
\verb+\with+.  En el siguiente ejemplo vamos a crear dos contextos
de pentagrama y en el segundo de ellos quitaremos los grabadores
de la indicación de compás y de la clave.

\begin[verbatim,relative=2,staffsize=15]{lilypond}
<<
  \new Staff { c1 }
  \new Staff \with { \remove Time_signature_engraver
                     \remove Clef_engraver
                   } { c1 }
>>
\end{lilypond}


\subsection{Sistemas de coro}

La música vocal polifónica suele emplear sistemas de pentagramas sin
líneas divisorias entre ellos, para no entorpecer al texto.  LilyPond
lo hace mediante el contexto \verb+ChoirStaff+.  Aquí vemos un ejemplo
con dos pentagramas:



\begin[verbatim,staffsize=15]{lilypond}
letra = \lyricmode { Hal -- le -- lu -- jah! }
\new ChoirStaff <<
  \time 2/4 \partial 8
  \new Staff \relative c'' { c16 c c8 c }            \addlyrics { \letra }
  \new Staff \relative c   { \clef bass e16 e f8 c } \addlyrics { \letra }
>>
\end{lilypond}
