\setcounter{section}{16} %para 17 set


\section{Cuarteto de cuerda. La instrucción \texttt{set}.}


\subsection{Modelo}

Presentamos el comienzo de un cuarteto de Beethoven en el que puede
verse el nombre de los instrumentos, clave de Do en la viola y una
serie de títulos adicionales.  El contexto que engloba a los
pentagramas es \verb+StaffGroup+.

\bigskip

\begin[staffsize=12.5,line-width=17\cm]{lilypond}
%#(set-global-staff-size 12.5)


\header {
title = "SECHS QUARTETTE"
subtitle = "für 2 Violinen, Bratsche und Violoncell"
composer = "L. VAN BEETHOVEN"
opus = "Opus 18. nº1."
piece = "Quartett nº1."
%subsubtitle = "."
dedication = "Dem Fürsten von Lobkowitz gewidmet."}

violinUno = \relative c' { \set Staff.instrumentName = #"Violino I "
			\key f \major 	\time 3/4
			\tempo "Allegro con brio"
			f4\p ~
			 f8( g16 f) e8_. f_.
			c4 r4 r4
			f4~ f8( g16 f) e8_. f_.
			d4 r4 r4
			f'4~ \< f8( g16 f) e8-. f-.
			g2(\> bes,4)
			a2(\! d8. bes16)
			a2( g4) \break

}


violinDos = \relative c' { \set Staff.instrumentName = #"Violino II " \key f \major 	\time 3/4
			f4~\p f8( g16 f) e8_. f_.
			c4 r4 r4
			f4~ f8( g16 f) e8_. f_.
			d4 r4 r4
			bes'2.(\<
			bes2)\>( g4)
			f2(\! bes8. g16)
			f2( e4)
}


viola = \relative c { \set Staff.instrumentName = #"Viola" \key f \major \time 3/4  \clef alto
			f4~ \p f8( g16 f) e8_. f_.
			c4 r4 r4
			f4~ f8( g16 f) e8_. f_.
			d4 r4 r4
			d'2.(\<
			c2.)(\>
			c4)(\! d g,)
			c8( b c b c4)
}

cello = \relative c { \set Staff.instrumentName = #"Violoncello" \key f \major \time 3/4  \clef bass
			f4~ \p f8( g16 f) e8^. f^.
			c4 r4 r4
			f4~ f8( g16 f) e8^. f^.
			d4 r4 r4
			d2.(\<
			e2.)(\>
			f4)(\! d bes)
			c2.

}


\score {
\new StaffGroup {
<<	\new Staff {\violinUno}
    \new Staff {\violinDos}
    \new Staff {\viola}
    \new Staff {\cello}  >>
    }
    \layout{indent=1.5\cm}
    \midi{}
}
\end{lilypond}

\subsection{Establecer el nombre del instrumento con \texttt{set}.}

En LilyPond, los contextos tienen una serie de propiedades que podemos
modificar mediante la instrucción \verb+\set+ indicando el nombre del
contexto y de la propiedad que se quiere modificar, separados mediante
un punto, después un signo igual ``\verb+=+'' y finalmente el valor
deseado para la propiedad.  Por ejemplo, si queremos establecer la
propiedad \verb+instrumentName+ (nombre del instrumento) del contexto
\verb+Staff+ al valor ``Flauta'', escribimos lo siguiente:

\begin[relative=2,verbatim,staffsize=17.5]{lilypond}
\set Staff.instrumentName = #"Flauta"
f2.
\end{lilypond}

\subsection{Clave de viola: Do en tercera}

Podemos aplicar la clave de Do en tercera línea que utiliza la viola
con la abreviatura ``alto'' como argumento para la instrucción
\verb+\clef+.  Como siempre, las notas se deben introducir en su
altura real, independientemente de la clave:

\begin[relative=1,verbatim,staffsize=17.5]{lilypond}
\set Staff.instrumentName = #"Viola"
\clef alto
c2.
\end{lilypond}


\subsection{Títulos adicionales}

Para el ejemplo hemos cumplimentado algunos títulos adicionales en el
bloque \verb+\header+.  Ya conocemos \verb+title+ (título),
\verb+substitle+ (subtítulo) y \verb+composer+ (autor).  Ahora
añadimos los siguientes: \verb+opus+, \verb+piece+ y \verb+dedication+
para el número de Opus, denominación de la pieza y dedicatoria,
respectivamente.

\begin{verbatim}
\header{ title="Título"
 dedication="Dedicatoria"
 opus="Número de Opus"
 piece="Pieza"
}
\end{verbatim}

