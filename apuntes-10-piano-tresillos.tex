\section{Sistemas de piano. Tresillos}


\subsection{Modelo}

Este fragmento de música para piano tiene una llave que une los dos
pentagramas.  En él hay tresillos y dos voces en el pentagrama
inferior.

\bigskip

\begin[staffsize=17.5]{lilypond}
\new PianoStaff <<
\new Staff \relative c' { \time 2/4
	\times 2/3 { c8 e g } d4
	e8 c d4
	\times 2/3 { c8 e g } \times 2/3 { f e d }
	c4 d
}
\new Staff \relative c { \clef bass
	<< {
	c4 fis
	g4 fis
	e4 fis
	e d }
	\\
	{ c2 ~ c ~ c ~ c } >>
}
>>
\end{lilypond}


\subsection{Tresillos y otros grupos de valoración especial}

He aquí cómo se pueden expresar los tresillos del Bolero de M. Ravel:


\begin[verbatim,relative=3,staffsize=17.5]{lilypond}
\time 3/4 g8[ \times 2/3 { g16 g g] } g8[ \times 2/3 { g16 g g] } g8 g
\end{lilypond}


Para componer tipográficamente un grupo de valoración especial se usa
la instrucción \verb+\times+ \emph{fracción} \verb+{ ... }+, que
multiplica la expresión entre llaves por la fracción expresada.

Por ejemplo, el siguiente grupo vale como 6 corcheas:

\begin[verbatim,relative=2,staffsize=17.5]{lilypond}
\time 3/4 \times 6/7 { ees8( f ees d ees ges8. f16) }
\end{lilypond}


\subsection{Sistemas de piano}

Declarando el contexto explícito \verb+PianoStaff+ podemos dibujar un
sistema de piano e introducir dentro de él los pentragramas superior e
inferior:

\begin[verbatim,staffsize=17.5]{lilypond}
\new PianoStaff <<
  \new Staff \relative c' { c4 c c c }
  \new Staff \relative c  { \clef bass c4 c c c }
>>
\end{lilypond}

