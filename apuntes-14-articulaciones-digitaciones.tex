
\setcounter{section}{13} %para 14 articulaciones


\section{Articulaciones y digitaciones: Sonatina de Bartok (I)}


\subsection{Modelo}

Este ejercicio procede de la Sonatina para piano de Bela
Bartok. Contiene una indicación metronómica, digitaciones, acentos y
otras articulaciones.

\bigskip

\begin[staffsize=17.5,line-width=17\cm]{lilypond}
\new Staff \relative c' { \time 2/4 \tempo "Moderato" 4=80

	e32(-> -2 \mf
	%-"pesante"
	fis e8 d16) e32( ->fis e8 d16)
	c16(-3 b)-. a-. b-. c4-3--->

	e32(-> -2
	fis e8 d16) e32( ->fis e8 d16)
	c16(-4 b)-. a-. g-. a4-3--->
}
\end{lilypond}

\subsection{Tempo con indicación metronómica}
Además de la instrucción normal de tempo del tipo
\verb+\tempo "Allegro"+, podemos añadir un valor de figura, seguido de
un signo igual y un número, que se imprimirán entre paréntesis como
indicación metronómica. La indicación metronómica aparecerá sola, si
no se escribe ningún texto dentro de las comillas. También puede
aparecer sin los paréntesis, quitando el texto y las comillas.

\begin[relative=1,verbatim,staffsize=17.5]{lilypond}
\tempo "Allegro" 1=120 c1 c c \tempo "" 1=80 c c c \tempo 1=40 c c c
\end{lilypond}

\subsection{Digitaciones y articulaciones}

Mediante el guión podemos adjuntar a una nota articulaciones,
digitaciones o textos:

\begin[relative=1,verbatim,staffsize=17.5]{lilypond}
 c4-> c-- c-. c-2 c1-"texto"
\end{lilypond}

En general se recomienda dejar la situación automática que LilyPond da
a las articulaciones, pero también se puede forzar su posición encima
o debajo de la nota sustituyendo el guión por un circunflejo o una
barra baja, respectivamente:

\begin[relative=2,verbatim,staffsize=17.5]{lilypond}
 b4-> b-- b-. b-2 b1-"texto"   % automático
 b4^> b^- b^. b^2 b1^"texto"   % siempre arriba
 b4_> b_- b_. b_2 b1_"texto"   % siempre abajo
\end{lilypond}



