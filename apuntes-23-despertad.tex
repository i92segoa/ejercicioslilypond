% \version "2.17.0"

\section{Ejercicios para voz y pulso, para niños.}


\subsection{Modelo}

En la enseñanza musical para niños pequeños, es muy frecuente la
utilización de canciones sencillas con un acompañamiento en forma
de \emph{ostinato} rítmico.  Para realizar este ejemplo hemos
utilizado un contexto especial que traza una pauta de una sola
línea, y hemos hecho que presente una clave especial utilizada en
percusión.  Además, las figuras tienen la cabeza en forma de línea
inclinada.

% Aumentar la separación entre sistemas
\def\betweenLilyPondSystem#1{\vspace{0.2cm}\linebreak}

\bigskip

\begin[staffsize=15,
line-width=16.5\cm,
indent=0
]{lilypond}
#(set-global-staff-size 17)
musica =  \relative c'' { \time 2/4
	g4 g e2
	g4 g e2
	g4 g e e
	g4 g e e

	e4 e c2
	e4 e c2
	e4 e c c
	e4 e c2 \bar "|."
}

letra = \lyricmode { Des -- per -- tad a mi voz,
me re -- pi -- ten las cam -- pa -- nas.
Din din don.
Din din don.
¡La ma -- ña -- na ya lle -- gó!
}

\header { title = "DESPERTAD" }

\score { << \new Staff { \set Staff.instrumentName="Voz" \musica }
		\addlyrics { \letra }
	\new RhythmicStaff \with { \consists "Clef_engraver" }

	{  \set Staff.instrumentName="Pulso" \clef percussion
	\improvisationOn \stemDown
	\repeat unfold 32 { c4 } }
>>
}
\end{lilypond}


\subsection{Pauta rítmica}
\label{rhythmicstaff}

El contexto \verb+RhythmicStaff+ traza una pauta sin clave, de una
sola línea. La altura de las notas no se tiene en cuenta, como puede
verse en el ejemplo siguiente:

\begin[verbatim,relative=2,staffsize=17.5]{lilypond}
\new RhythmicStaff { c4 d e2
  c8 c c d e2 }
\end{lilypond}


\subsection{Clave de percusión}

La clave de percusión puede conseguirse mediante la instrucción
\verb+\clef+ con la palabra ``percussion'' como argumento.

\begin[verbatim,relative=1,staffsize=17.5]{lilypond}
\clef percussion c4 d e2 c8 c c d e2
\end{lilypond}


\subsection{Figuras de estilo improvisado}

A veces se representan mediante figuras en forma de línea inclinada
las notas que no se especifican con una altura determinada, sino que
se deja que el intérprete las improvise.  Para esto se utiliza la
instrucción \verb+\improvisationOn+:

\begin[verbatim,relative=2,staffsize=17.5]{lilypond}
\improvisationOn c4 d e2 c8 c c d e2
\end{lilypond}

Más tarde, en caso necesario, es posible volver a la notación normal
mediante \verb+\improvisationOff+.

\subsection{Notas}

\begin{itemize}
\item Utilice repeticiones desplegadas para el pulso (véase el
  apartado \ref{unfold}, página \pageref{unfold})
\item Añada el grabador de clave, llamado \verb+Clef_engraver+, al
  contexto \verb+RhythmicStaff+, (véase el apartado
  \ref{consists}, página \pageref{consists})
\item El grabador encargado de ignorar la altura de las notas para
  imprimirlas sobre una línea es \verb+Pitch_squash_engraver+,
  puede añadirse al contexto \verb+Voice+, pero ya viene incluido
  en el contexto \verb+RhythmicStaff+.
\end{itemize}

%anular separación entre sistemas para el futuro
\let\betweenLilyPondSystem\undefined
