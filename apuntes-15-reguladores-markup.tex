\section{Reguladores.  Elementos de marcado. \emph{Sonatina} de Bartok (II)}


\subsection{Modelo}

Para este ejercicio podemos reutilizar la parte hecha en el anterior.
Aquí hemos incorporado una inscripción textual en tipo itálica, en el
primer compás, y hemos añadido otros cuatro compases que contienen
reguladores.

\bigskip

\begin[staffsize=17.5,line-width=17\cm]{lilypond}
\new Staff \relative c' {
	\time 2/4 \tempo "Moderato" 4=80

	e32(-> -2 \mf

	fis 	-\markup{ \italic "pesante" } e8 d16) e32( ->fis e8 d16)
	c16(-3 b)-. a-. b-. c4-3--->

	e32(-> -2
	fis e8 d16) e32( ->fis e8 d16)
	c16(-4 b)-. a-. g-. a4-3---> \break

	g8.->(-1 \< a32 b c8-.-4)\! c-.-4
	c16-3( b-.) a-. b-. c32-^(\> d c d \times 4/5 { c[ d c b a]\! }

	g8.->-2)\< ( a32 b c8-.-5)\! c-.-5
	c16-4( b-.)\> a-. g-. a4-.-- \!
	\bar "||"


}
\end{lilypond}

\subsection{Elementos de marcado}
Ya vimos que los textos se pueden adjuntar a una nota como si se
tratase de una articulación. Estos textos simples no admiten ningún
formato, pero los elementos de marcado sí permiten una amplia variedad
de estilos.  Por ahora, tan sólo pondremos como ejemplo un texto en
itálica para expresar un cierto carácter:

\begin[relative=1,verbatim,staffsize=17.5]{lilypond}
c8 -\markup{ \italic "dolce" }
d e f g a b c
\end{lilypond}

\subsection{Reguladores}

Para obtener indicaciones gráficas de matices dinámicos, se marca la
nota de comienzo y la de final con dos símbolos especiales.  La marca
de final para cancelar el regulador sólo es necesaria cuando no ocurre
una indicación dinámica normal.

\begin[relative=1,verbatim,staffsize=17.5]{lilypond}
c8 \p \< d e f g a b c
d \f \> c b a g f e d \! c1
\end{lilypond}

\subsection{Acento}

El acento en forma de 'v' o de 'v invertida' utiliza el símbolo del
acento circunflejo, con un significado distinto al de forzar la
dirección:

\begin[relative=1,verbatim,staffsize=17.5]{lilypond}
c1 -^
c ^^
\end{lilypond}


