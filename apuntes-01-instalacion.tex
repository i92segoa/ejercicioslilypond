% \version "2.17.0"

\section{Uso de LilyPond y Frescobaldi bajo Windows}
\subsection{Descarga e instalación}

Recomendamos utilizar la combinación LilyPond/Frescobaldi para el
trabajo con partituras de LilyPond. Ambos están disponibles para su
descarga gratuita en las respectivas páginas oficiales de los
proyectos, \texttt{lilypond.org} y \texttt{frescobaldi.org}.  Después
de ejecutar los programas instaladores, las aplicaciones están listas
para su uso.

%\subsection{Configuración de Frescobaldi}
%Después de instalar Frescobaldi y LilyPond, es necesario que el
%primero pueda llamar al segundo para procesar las partituras, para lo
%que debe estar configurado con la ruta

% Opcional, no es seguro, comprobar

\subsection{Creación de una partitura sencilla}

Ahora comprobaremos si todo nuestro software funciona, con un ejemplo
absolutamente mínimo. Abrimos el programa Frescobaldi y escribimos lo
siguiente\footnote{Las llaves se consiguen con AltGr pulsando una
  tecla que en los teclados españoles de PC suele estar junto a la
  'Ñ'. Los apóstrofos se consiguen mediante la tecla que está justo a
  la derecha del número 0.}:

\begin{quote}
\begin{verbatim}
 { b }
\end{verbatim}
\end{quote}

Denominamos a este texto \emph{código de entrada}.

\subsection{Procesar el documento}

Los programas Frescobaldi (un editor) y LilyPond (un generador de
partituras a partir de un texto) forman una combinación en la que cada
uno está especializado en una misión.  Si pulsamos en Frescobaldi la
combinación de teclas Control+M, éste llama a LilyPond pasándole el
documento en curso para que lo convierta en una partitura en formato
PDF. Esto se denomina \emph{procesar el código de entrada}, y al PDF
resultante le llamamos \emph{la salida}.

El procesado tarda un par de segundos\footnote{La primera vez después
  de haber instalado LilyPond, el programa tiene que preparar las
  fuentes tipográficas; esto lleva aproximadamente medio minuto, pero
  las ejecuciones posteriores tardan, como se ha dicho, unos
  segundos.}. El resultado es un archivo PDF que puede verse en el
panel derecho de Vista Previa de la música:

\begin[staffsize=15,fragment,quote]{lilypond}
  b
\end{lilypond}

Podemos guardar este texto con un nombre terminado en la extensión
\verb+.ly+, por ejemplo \verb+prueba.ly+.  Pulse la combinación de
teclas Control+S para hacerlo.  Denominaremos a este archivo con la
extensión \verb+.ly+ que contiene el código de entrada, \emph{archivo
  fuente} o \emph{archivo de entrada}.  Si pulsamos de nuevo
Control+M, el PDF resultante se llamará \verb+prueba.pdf+ y estará
localizado en la misma carpeta en que hemos guardado el archivo
fuente.

Antes de haber guardado el documento, tanto el texto de entrada como
el PDF de salida se encuentran en una carpeta temporal, de donde se
borrarán al salir del programa Frescobaldi.

\subsection{Expresiones musicales de LilyPond}

Una partitura completa de LilyPond es un fragmento de música encerrado
entre llaves \verb+{ }+, llamado \emph{expresión musical}.  En los
ejemplos que aparecen en este cuaderno de ejercicios, con frecuencia
se omiten las llaves por sencillez, pero no debe olvidarlas cuando
inserte el código en sus propios ejercicios.

De igual forma que las expresiones matemáticas, en LilyPond podemos
ampliar una expresión añadiéndole operadores a la izquierda o
combinando varias de ellas en una sola; el ejemplo de este primer
ejercicio, sin embargo, es una expresión sencilla.  Veremos casos más
complejos cuando la música lo requiera.

\subsection{Conclusión}

Observamos que nuestro código de entrada minimalista se limita a
declarar la nota \emph{si} por su nombre anglosajón, ``b'', y que el
resultado incluye esta nota (con un valor de negra) y además un
pentagrama, una clave de \emph{sol} y un compás de 4/4.  Son los
valores por omisión, que se dan por supuestos si no los damos
explícitamente.

Le damos la enhorabuena si ha conseguido completar la primera lección
con éxito. En los apartados siguientes vamos a profundizar en la
producción de partituras progresivamente más complejas.
