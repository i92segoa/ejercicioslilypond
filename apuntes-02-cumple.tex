% \version "2.17.0"

\section{\emph{Cumpleaños feliz}}
\subsection{Modelo}

Aprenderemos a tipografiar este ejemplo mediante las indicaciones que
se dan en los apartados siguientes.  

\bigskip

\begin[relative=2,staffsize=13,fragment]{lilypond}
  \time 3/4 \partial 4
g8. g16 a4 g c b2
g8. g16 a4 g d' c2
g8. g16 g'4 e c b a\fermata
f'8. f16 e4 c d c2. \bar "|."
\end{lilypond}

Recordemos que el objetivo de cada uno de los 30 ejercicios es
reproducir el modelo, adaptando por nosotros mismos las indicaciones
de los epígrafes que le siguen.

\subsection{Modo relativo}
Al introducir las notas, si precedemos la expresión entre llaves por
la instrucción \verb+\relative+ seguida de una nota, no tenemos que
especificar la altura de cada nota para saltos de cuarta o menores; a
partir de un salto de quinta hay que añadir un apóstrofo para subir
una octava, y una coma para bajar una octava.

\begin[verbatim,staffsize=13]{lilypond}
\relative c' { c e g c g' a f e d c g c, }
\end{lilypond}


\subsection{Compás}

El compás es de 3/4; escribimos
\begin[verbatim,relative=2,staffsize=13,fragment]{lilypond}
\time 3/4
\end{lilypond}


\subsection{Anacrusa}

El compás inicial está incompleto y sólo tiene un valor de negra; lo
expresamos mediante \verb+\partial+ seguido de una duración, en
nuestro caso el 4 que indica un valor de negra.

\begin[verbatim,relative=2,staffsize=13,fragment]{lilypond}
\time 3/4 \partial 4 g
\end{lilypond}

\subsection{Duraciones. Puntillo}

Los valores de nuestro ejemplo son: blanca, negra, corchea y
semicorchea.  Se escriben como las cifras 2, 4, 8 y 16,
respectivamente, detrás de la nota.

\begin[verbatim,relative=2,staffsize=13,fragment]{lilypond}
g2 g4 g8 g16
\end{lilypond}

El puntillo se consigue mediante el punto ortográfico después del
número de la duración.

\begin[verbatim,notime,relative=2,staffsize=13,fragment]{lilypond}
g2. g4. g8.
\end{lilypond}

\subsection{Calderón y doble barra final}

Colocamos un calderón sobre una nota mediante la instrucción
\verb+\fermata+ y la doble barra mediante la instrucción \verb+\bar+
seguida del tipo de barra deseado, que en nuestro caso es \verb+"|."+,
entre comillas.

\begin[verbatim,notime,relative=2,staffsize=13,fragment]{lilypond}
g2\fermata \bar "|."
\end{lilypond}


