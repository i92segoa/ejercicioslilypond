%\documentclass[12pt,a4paper,oneside]{scrbook} % la clase book del Koma-script bundle
\documentclass[a4paper,10pt,oneside,headinclude,titlepage]{article} % la clase book del Koma-script bundle
%\linespread{1.25}
\usepackage{setspace}
%\usepackage{tikz}
%\usetikzlibrary{fit,shapes}
\usepackage[spanish]{babel}
%\usepackage{verbatim} %para el entorno comment
%\usepackage{moreverb} %para los ejemplos de lilypond, aporta verbatimtabinput
%\usepackage{alltt} %para los ejemplos de lilypond, aporta verbatiminput
%\usepackage{sverb} %para los ejemplos de lilypond, aporta verbinput
%\usepackage{fancyvrb} %para los ejemplos de lilypond, aporta VerbatimInput
%\pagestyle{empty}
\usepackage[utf8]{inputenc}
\usepackage[T1]{fontenc} %posiblemente sirva para eliminar el problema del enguionado de palabras acentuadas. Lo quitamos provisionalmente para evitar un error
\usepackage{textcomp} % recomendación de Javier Bezos para completar la fuente

%\usepackage[left=3cm, right=3cm]{geometry}
\usepackage{graphicx}
%\usepackage{url}

\usepackage[utopia]{mathdesign}
%\usepackage{mathptmx} %mejor que Times    % alternativa a Charter


\typearea[0mm]{13}% same as class options above
%\usepackage{newcent}
%\addtokomafont{part}{\mdseries} %encabezamientos sin negrita
%\addtokomafont{partnumber}{\mdseries} %encabezamientos sin negrita
%\addtokomafont{chapter}{\mdseries} %encabezamientos sin negrita
%\setkomafont{disposition}{\normalcolor\bfseries} %no sans serif
%\setkomafont{disposition}{\normalcolor\mdseries} %no negrita

\parskip=6pt\clubpenalty=10000\widowpenalty=10000

\newcommand{\preLilyPondExample}{\vspace{-10pt}}

\newcommand{\lpversion}{2.13.4}
\newcommand{\defsep}{\textbf{$\|$}}
\newcommand{\software}{\emph{software}}
\newcommand{\negspace}{\vspace{-10pt}}  %{\vspace{-20pt}}
\newcommand{\seppar}{
\bigskip
%\vspace{6pt}
}

%%%%%%%%%%%%%%%%%%%%%%%%%%%%%%%%%%%%%%%%%%%%%%%%%%%%%%%%%%%%%%%%%%%%%%%%%%%%%%%%%%%%%%%%%%%
\begin{document}

\setcounter{section}{1}
\section{Cumpleaños feliz}
\subsection{Modelo}

Aprenderemos a tipografiar este ejemplo mediante las indicaciones que se dan en los apartados siguientes.

\bigskip

\begin[relative=2,staffsize=13,fragment]{lilypond}
  \time 3/4 \partial 4
g8. g16 a4 g c b2
g8. g16 a4 g d' c2
g8. g16 g'4 e c b a\fermata
f'8. f16 e4 c d c2. \bar "|."
\end{lilypond}

\subsection{Modo relativo}
Al introducir las notas, si precedemos la expresión entre llaves por la instrucción \verb+\relative+ seguida de una nota, no tenemos que especificar la altura de cada nota para saltos de cuarta o menores; a partir de un salto de quinta hay que añadir un apóstrofo para subir una octava, y una coma para bajar una octava.

\begin[verbatim,staffsize=13]{lilypond}
\relative c' { c e g c g' a f e d c g c, }
\end{lilypond}


\subsection{Compás}

El compás es de 3/4; escribimos
\begin[verbatim,relative=2,staffsize=13,fragment]{lilypond}
\time 3/4 
\end{lilypond}


\subsection{Anacrusa}

El compás inicial está incompleto y sólo tiene un valor de negra; lo expresamos mediante \verb+\partial+ seguido de una duración, en nuestro caso el 4 que indica un valor de negra.

\begin[verbatim,relative=2,staffsize=13,fragment]{lilypond}
\time 3/4 \partial 4 g
\end{lilypond}

\subsection{Duraciones. Puntillo}

Los valores de nuestro ejemplo son: blanca, negra, corchea y semicorchea.  Se escriben como las cifras 2, 4, 8 y 16, respectivamente, detrás de la nota.

\begin[verbatim,relative=2,staffsize=13,fragment]{lilypond}
g2 g4 g8 g16
\end{lilypond}

El puntillo se consigue mediante el punto ortográfico después del número de la duración.

\begin[verbatim,notime,relative=2,staffsize=13,fragment]{lilypond}
g2. g4. g8.
\end{lilypond}

\subsection{Calderón y doble barra final}

Colocamos un calderón sobre una nota mediante la instrucción \verb+\fermata+ y la doble barra mediante la instrucción \verb+\bar+ seguida del tipo de barra deseado, que en nuestro caso es \verb+"|."+, entre comillas.

\begin[verbatim,notime,relative=2,staffsize=13,fragment]{lilypond}
g2\fermata \bar "|."
\end{lilypond}



\end{document}

