\section{\emph{Esta noche es Nochebuena} (I). Canciones con letra.}


\subsection{Modelo}

A continuación presentamos un villancico del s. XVI, original de
Gales, con título en inglés \emph{Deck the Halls}:

\bigskip

\begin[staffsize=17.5,line-width=17\cm]{lilypond}
\relative c'' { \key f \major
c8. bes16 a8 g
f g a8 f
g16 a bes g a8. g16
f8 e f4
c'8. bes16 a8 g
f g a8( f)
d'16 d d d c8. bes16
a8 g f4
}
\addlyrics { Es -- ta no che~es No -- che -- bue -- na,
	fa la la la la, la la la la.
	Y no~es no  -- che de dor -- mir
	fa la la la la, la la la la. }
\end{lilypond}


\subsection{Contextos de letra}

El contexto de letra se llama Lyrics, y su contenido debe ir precedido
de \verb+\lyricmode+ para que se interprete como letra. Las sílabas se
separan mediante dos guiones.  Una forma de alinear la letra con la
música es expresar la duración de cada sílaba como si fueran notas:

\begin[verbatim,relative=3,staffsize=17.5]{lilypond}
<<
  \new Staff \relative c'' { \time 3/4 \partial 4 g8. g16 a4 g c b }
  \new Lyrics \lyricmode { Cum8. -- ple16 -- a4 -- ños fe -- liz }
>>
\end{lilypond}

Otra manera, más sencilla, es utilizar \verb+\addlyrics+ después de la
música, como aparece en el siguiente ejemplo.  Las sinalefas se
consiguen uniendo las sílabas mediante una tilde curva, el mismo
símbolo que se utiliza para la ligadura de unión.

\begin[verbatim,relative=3,staffsize=17.5]{lilypond}
\relative c' { \partial 4 e8 f g4 c b8 b r4 }
\addlyrics { ¿Dón -- de~es -- tán las lla -- ves? }
\end{lilypond}

