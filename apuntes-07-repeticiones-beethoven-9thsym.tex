\section{Repeticiones. \emph{Novena sinfonía} de Beethoven}


\subsection{Modelo}



He aquí un ejemplo de repetición de primera y segunda vez:

\bigskip

\begin[staffsize=17.5,line-width=17\cm]{lilypond}


\header { title = "Novena Sinfonía" composer = "Beethoven" }

\relative c' { \key c \major

\repeat volta 2 { e2 f4 g g f e d c c d e } \alternative { { e2 d2 } { d2 c2 } }


}


\end{lilypond}


\subsection{Repeticiones sencillas}

En LilyPond, las repeticiones no se hacen definiendo tipos de barra o
dibujando explícitamente puntos de repetición.  En lugar de eso,
definimos el fragmento que se repite y cuáles son los finales
alternativos, como bloques separados dentro de la instrucción
\verb+\repeat+.  Hay varios tipos de repetición; para la primera y
segunda vez, empleamos esta forma:

\verb+\repeat volta veces {trozo que se repite} \alternative{{primera vez}{segunda vez}}+

En este ejemplo, dejamos \verb+volta+ como está, para expresar el tipo
de repetición; sustituimos \verb+veces+ por el número de repeticiones,
y en los bloques de primera y segunda vez escribimos la música que va
dentro de las casillas de primera y segunda.  Los bloques de los
finales alternativos van, a su vez, dentro del bloque
\verb+\alternative{}+ entre llaves.

Si hay más de dos repeticiones, la segunda alternativa se marca para
ejecutarse la última vez.

\begin[verbatim,relative=2,staffsize=13]{lilypond}
\repeat volta 3 { g4 f e d } \alternative{{ g1 } { c,1 }}
\end{lilypond}

El modo relativo sigue funcionando dentro del texto de entrada de
forma normal como si toda la música fuese secuencial, sin
repeticiones.

