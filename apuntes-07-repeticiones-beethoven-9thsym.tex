%\documentclass[12pt,a4paper,oneside]{scrbook} % la clase book del Koma-script bundle
\documentclass[a4paper,10pt,oneside,headinclude,titlepage]{article} % la clase book del Koma-script bundle
%\linespread{1.25}
\usepackage{setspace}
%\usepackage{tikz}
%\usetikzlibrary{fit,shapes}
\usepackage[spanish]{babel}
%\usepackage{verbatim} %para el entorno comment
%\usepackage{moreverb} %para los ejemplos de lilypond, aporta verbatimtabinput
%\usepackage{alltt} %para los ejemplos de lilypond, aporta verbatiminput
%\usepackage{sverb} %para los ejemplos de lilypond, aporta verbinput
%\usepackage{fancyvrb} %para los ejemplos de lilypond, aporta VerbatimInput
\pagestyle{empty}
\usepackage[utf8]{inputenc}
\usepackage[T1]{fontenc} %posiblemente sirva para eliminar el problema del enguionado de palabras acentuadas. Lo quitamos provisionalmente para evitar un error
\usepackage{textcomp} % recomendación de Javier Bezos para completar la fuente

\usepackage[margin=2cm]{geometry}
\usepackage{graphicx}
%\usepackage{url}

\usepackage[utopia]{mathdesign}
%\usepackage{mathptmx} %mejor que Times    % alternativa a Charter


%\typearea[0mm]{13}% same as class options above
%\usepackage{newcent}
%\addtokomafont{part}{\mdseries} %encabezamientos sin negrita
%\addtokomafont{partnumber}{\mdseries} %encabezamientos sin negrita
%\addtokomafont{chapter}{\mdseries} %encabezamientos sin negrita
%\setkomafont{disposition}{\normalcolor\bfseries} %no sans serif
%\setkomafont{disposition}{\normalcolor\mdseries} %no negrita

\parskip=6pt\clubpenalty=10000\widowpenalty=10000

\newcommand{\preLilyPondExample}{\vspace{-10pt}}

\newcommand{\lpversion}{2.13.4}
\newcommand{\defsep}{\textbf{$\|$}}
\newcommand{\software}{\emph{software}}
\newcommand{\negspace}{\vspace{-10pt}}  %{\vspace{-20pt}}
\newcommand{\seppar}{
\bigskip
%\vspace{6pt}
}

%%%%%%%%%%%%%%%%%%%%%%%%%%%%%%%%%%%%%%%%%%%%%%%%%%%%%%%%%%%%%%%%%%%%%%%%%%%%%%%%%%%%%%%%%%%
\begin{document}

\setcounter{section}{6} %para 07repeticiones


\section{Repeticiones. Novena sinfonía de Beethoven}


\subsection{Modelo}



He aquí un ejemplo de repetición de primera y segunda vez:

\bigskip

\begin[staffsize=17.5,no-ragged-right]{lilypond}


\header { title = "Novena Sinfonía" composer = "Beethoven" }  

\relative c' { \key c \major

\repeat volta 2 { e2 f4 g g f e d c c d e } \alternative { { e2 d2 } { d2 c2 } }


}


\end{lilypond}


\subsection{Repeticiones sencillas}

En LilyPond, las repeticiones no se hacen definiendo tipos de barra o
dibujando explícitamente puntos de repetición.  En lugar de eso,
definimos el fragmento que se repite y cuáles son los finales
alternativos, como bloques separados dentro de la instrucción
\verb+\repeat+.  Hay varios tipos de repetición; para la primera y
segunda vez, empleamos esta forma:

\verb+\repeat volta veces {trozo que se repite} \alternative{{primera vez}{segunda vez}}+

En este ejemplo, dejamos \verb+volta+ como está, para expresar el tipo
de repetición; sustituimos \verb+veces+ por el número de repeticiones,
y en los bloques de primera y segunda vez escribimos la música que va
dentro de las casillas de primera y segunda.  Los bloques de los
finales alternativos van, a su vez, dentro del bloque
\verb+\alternative{}+ entre llaves.

Si hay más de dos repeticiones, la segunda alternativa se marca para
ejecutarse la última vez. 

\begin[verbatim,relative=2,staffsize=13]{lilypond}
\repeat volta 3 { g4 f e d } \alternative{{ g1 } { c,1 }}
\end{lilypond}

El modo relativo sigue funcionando dentro del texto de entrada de
forma normal como si toda la música fuese secuencial, sin
repeticiones.


\end{document}

