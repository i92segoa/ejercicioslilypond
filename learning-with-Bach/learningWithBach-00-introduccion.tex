\section*{Introducción}

``Aprende LilyPond con Bach'' añade a la colección de 30 ejercicios
semanales el atractivo, para los amantes de la música de J.S. Bach, de
utilizar esta gran música como vehículo y como hilo conductor del
material de los ejemplos y los retos a que el lector estará habituado.
LilyPond, como sabrá, es una potente herramienta libre de producción
de partituras musicales de alta calidad.  La idea de que practicando
con el lenguaje en un régimen abundante en ejemplos es la mejor forma
de aprenderlo, no es nueva. Confieso que me influyeron un par de
libros de carácter didáctico, uno sobre el lenguaje PostScript y otro
sobre tipografía, a la hora de esctructurar el trabajo de esta
forma. Una vez más debo advertir que esto no es un curso completo de
LilyPond y que en todo caso la fuente más completa de información
sobre el programa es la amplísima documentación oficial.  Si ya siguió
nuestra anterior colección de ejercicios semanales, sabrá que lo único
que hay que hacer es tratar de reproducir el modelo que se ofrece,
utilizando los párrafos explicativos como pistas orientativas en
cuanto a la sintaxis de las instrucciones que es necesario utilizar.

Volvemos a ofrecer como soporte la lista de distribución de correo de
los usuarios hispanohablantes de
LilyPond\footnote{http://lists.gnu.org/mailman/listinfo/lilypond-es}.

Espero que disfrute con LilyPond y con Bach. Creo que no se puede
pedir más.

\section*{Licencia}

Este documento está publicado bajo la licencia (cc)(by)(sa) Creative
Commons Atribución - Compartir igual 3.0 España.  Algunos derechos
reservados.

Usted es libre de copiar, distribuir y comunicar públicamente la obra.
Puede hacer obras derivadas y distribuirlas, siempre que añada el
siguiente texto:

\begin{quote}
\emph{Basado en un trabajo anterior de Francisco Vila,
  \texttt{http://www.paconet.org} }
\end{quote}

visible en una de las dos primeras páginas, o una de las dos últimas.

La obra derivada debe distribuirse bajo la misma licencia.

El texto completo de esta licencia está en
http://creativecommons.org/licenses/by-sa/2.0/es/legalcode.es



