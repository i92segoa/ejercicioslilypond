\documentclass[a4paper,10pt,oneside,headinclude,titlepage]{article} % la clase article estándar
\usepackage{setspace}
\usepackage{pdfpages}
\usepackage[spanish]{babel}
\usepackage[utf8]{inputenc}
\usepackage[T1]{fontenc} %posiblemente sirva para eliminar el problema del enguionado de palabras acentuadas. Lo quitamos provisionalmente para evitar un error
\usepackage{textcomp} % recomendación de Javier Bezos para completar la fuente

\usepackage[margin=2cm]{geometry}
\usepackage{graphicx}
\usepackage[utopia]{mathdesign}
\parskip=0pt\clubpenalty=10000\widowpenalty=10000
\parindent=6mm
\newcommand{\preLilyPondExample}{\vspace{0pt}}
\newcommand{\postLilyPondExample}{\vspace{6pt}}

\newcommand{\lpversion}{2.15.30}
\newcommand{\defsep}{\textbf{$\|$}}
\newcommand{\software}{\emph{software}}
\newcommand{\negspace}{\vspace{-10pt}}
\newcommand{\seppar}{
\bigskip

}
\title{LilyPond}
\author{Francisco Vila}
\date{\today}

\begin{document}

\section{Introducción}

El alemán Johann Sebastian Bach es a la música clásica lo que el
finlandés Linus Torvalds es al software libre.

Torvalds no inventó el software libre, no es el único desarrollador, y
para unos es un genio mientras que otros lo ven como un oportuno
producto de su época. Con su singular trayectoria dentro del mundo de
la computación, no hay duda de que es una referencia, alguien que está
``en el inicio de muchas cosas'', honor que comparte con otros gurús y
que inspira a legiones de amantes del código y las aplicaciones
libres.

Bach es un icono en su campo: su poderosa mente del siglo XVII produjo
una cantidad enciclopédica de música de una misteriosa perfección
formal, verdaderos teoremas del arte de los sonidos. Para muchos el
mejor compositor de todos los tiempos, Bach nos servirá de hilo
conductor para presentar un veterano programa de tipografía musical
libre: GNU LilyPond.

LilyPond produce partituras musicales completas, listas para imprimir,
a partir de documentos de texto plano.  Es una aplicación de consola y
funciona en servidores como servicio\footnote{lilybin} o para
tipografiar bases de datos musicales\footnote{mutopia}, pero lo más
frecuente para un uso personal es utilizarlo a través de un entorno de
desarrollo como Frescobaldi.

\section{Lo instalamos y lo probamos}

Frescobaldi y LilyPond están en los repositorios oficiales de Debian y
Ubuntu. En el caso de Frescobaldi recomendamos la distribución wheezy
de Debian; en Ubuntu se puede encontrar a partir de la versión
12.04. Los podemos instalar mediante

\begin{quote}
\verb+# apt-get install frescobaldi lilypond+
\end{quote}

Lanzamos Frescobaldi desde nuestro menú principal, o bien

\begin{quote}
\verb+$ frescobaldi+
\end{quote}

Ahora comprobaremos si el tinglado completo funciona. Tecleamos en el
panel izquierdo del editor la partitura musical más minimalista que
existe,

\begin{quote}
\verb+{ b }+
\end{quote}

y al pulsar Control+M, Frescobaldi debería llamar a LilyPond y
presentar en el panel derecho la vista previa del PDF.

LilyPond presenta la música dentro de una serie de contextos
(pentagrama, partitura) predeterminados y listos para usar: traza la
pauta, la clave, el compás y lo coloca todo en una hoja A4.  Pasemos a
un ejemplo más completo que incluye un título, el nombre del autor, un
valor para las figuras, una alteración forzada y hasta un texto bajo
las notas.

\section{Un ejemplo completo}

\begin{verbatim}
\header{ title = "The title"            % (1)
         composer = "J. S. Bach" }
\language "deutsch"                     % (2)
music = \relative f { b'1 a c h! }      % (3)
\new Staff { \music }                   % (4)
\addlyrics { B A C H }                  % (5)
\end{verbatim}

Creemos que el bloque \verb+\header+ (1) es autoexplicativo. La
instrucción (2) declara el idioma para el nombre de las notas, en este
caso el alemán, aunque nosotros podemos poner
\verb+\language{espanol}+ o dejar el idioma por defecto, el holandés,
que es perfecto para la mayoría de las partituras. Es igual que el
alemán excepto porque \verb+b+ es \emph{si} natural.

En la instrucción (3) creamos una variable musical que contiene unas
cuantas notas. Esto nos permite utilizar la música más de una vez; de
la forma en que están codificadas estas notas, podemos hablar más
tarde.


\section{Expresiones}

La instrucción (4) introduce la expresión principal de la
partitura. En una partitura completa de LilyPond, debe haber una sola
expresión encerrada entre llaves, o con una o más funciones prefijas o
postfijas que no hacen sino aumentar la expresión.  Matemáticamente,
es como si tenemos un símbolo tal que ``a'' y lo aumentamos
multiplicándolo por un factor, ``3a'', le añadimos una cantidad ``3a +
5'' y le cambiamos el signo a todo el conjunto: ``-(3a + 5)''. Lo que
tenemos al final sigue siendo una única expresión.

En nuestro documento, las instrucciones (4) y (5) son el núcleo de la
partitura. En (4) creamos un pentagrama; cuando declaramos un contexto
\verb+Staff+ de forma explícita para incluir la música dentro de él,
estamos esctructurando nuestro contenido y preparando el documento
para que crezca sin perder legibilidad, manteniendo todo perfectamente
ordenado. En cuanto a su contenido, nos limitamos a usar la variable
que definimos previamente, precediendo su nombre por una barra
invertida.

En (5) añadimos un texto: cada sílaba o carácter suelto se coloca
automáticamente debajo de una nota de la música precedente. Es
perfecto para la letra de las canciones, y de hecho está pensado para
ello.

\section{El modo notas}

La sintaxis de introducción de notas es sencilla. Las siete notas son
\quad\verb+c d e f g a h+\quad en alemán,
\quad\verb+c d e f g a b+\quad en holandés y
\quad\verb+do re mi fa sol la si+\quad en español. Para explicar
nuestro ejemplo completamente, nos basta saber por ahora que un
apóstrofo coloca las notas una octava más aguda, los números a la
derecha como 1, 2, 4, 8, 16 son las duraciones de las figuras, y si no
damos ninguna duración se toma por defecto la de la nota anterior. El
conveniente modo relativo, dado por \verb+\relative <nota>+, establece
que las notas cercanas no requieren ninguna corrección de octava, sólo
los saltos de quinta o mayores lo necesitan.

Finalmente, hemos forzado en \verb+h!+ la alteración natural o
becuadro con el símbolo de admiración. El resultado final, bellamente
tipografiado por LilyPond en la tradición de las mejores ediciones
impresas, puede verse a continuación.

\includegraphics{BACH}

\section{Conclusión}
Lo crea o no, orquestas de todo el mundo interpretan sinfonías y
óperas que se han compuesto tipográficamente con LilyPond. Es muy
rápido para las pequeñas partituras de cada día y muy potente para
grandes proyectos. Pero sólo hemos acariciado la superficie. LilyPond
es a la música lo que \LaTeX\ es a los textos: un motor de tipografía
que compila textos de alta calidad, y que a cambio exige del usuario
que estructure su contenido en una serie de construcciones formales.

\end{document}
