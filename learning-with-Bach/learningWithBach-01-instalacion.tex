\section{Uso de LilyPond y Frescobaldi bajo Windows}
\subsection{Descarga e instalación}

Recomendamos utilizar la combinación LilyPond/Frescobaldi para el
trabajo con partituras de LilyPond. Ambos están disponibles para su
descarga gratuita en las respectivas páginas oficiales de los
proyectos, \texttt{lilypond.org} y \texttt{frescobaldi.org}.  Después
de ejecutar los programas instaladores, las aplicaciones están listas
para su uso.

\subsection{Configuración de Frescobaldi}
Después de instalar Frescobaldi y LilyPond, 

\subsection{Creación de una partitura sencilla}

Para la realización de nuestro primer ejemplo, abrimos el programa
Frescobaldi y escribimos lo siguiente\footnote{Las llaves se consiguen
  con AltGr pulsando una tecla que en los teclados españoles de PC
  suele estar junto a la 'Ñ'. Los apóstrofos se consiguen mediante la
  tecla que está justo a la derecha del número 0.}:

\begin{quote}
\begin{verbatim}
{  c' d' e' f' g'2 e' }
\end{verbatim}
\end{quote}

Denominamos a este texto \emph{código de entrada}.

Guardamos este texto con un nombre terminado en la extensión
\verb+.ly+, por ejemplo \verb+prueba.ly+.  Denominaremos a este
archivo con la extensión \verb+.ly+ que contiene el código de entrada,
\emph{archivo fuente} o \emph{archivo de entrada}.

\subsection{Procesar el documento}

La aplicación Frescobaldi no tiene la misión de hacer una composición
tipográfica de la partitura, y el programa LilyPond no se utiliza para
editar el documento. Los programas Frescobaldi (un editor) y LilyPond
(un generador de partituras a partir de un texto) forman una
combinación en la que cada uno está especializado en una
misión. Cuando pulsemos en Frescobaldi la combinación de teclas
Control+M, éste llama a LilyPond y le ofrece el documento en curso
para que lo convierta en una partitura (o ``salida'') en formato
PDF. Esto se denomina \emph{procesar el código de entrada}.

El procesado tarda un par de segundos\footnote{La primera vez después
  de haber instalado LilyPond, el programa tiene que preparar las
  fuentes tipográficas; esto lleva aproximadamente medio minuto, pero
  las ejecuciones posteriores tardan, como se ha dicho, unos
  segundos.}. El resultado es un archivo PDF que en nuestro caso se
llamará \verb+prueba.pdf+ y que puede verse en el panel derecho de
Vista Previa de la música:

\begin[staffsize=15,fragment]{lilypond}
  c' d' e' f' g'2 e'
\end{lilypond}


\subsection{Expresiones musicales de LilyPond}

Una partitura completa de LilyPond es un fragmento de música encerrado
entre llaves \verb+{ }+, llamado \emph{expresión musical}.  En los
ejemplos que aparecen en este cuaderno de ejercicios, con frecuencia
se omiten las llaves por sencillez, pero no deben olvidarse cuando se
inserte el código en los ejercicios propios.

De igual forma que las expresiones matemáticas, en LilyPond podemos
ampliar una expresión añadiéndole operadores a la izquierda o
combinando varias de ellas en una sola; el ejemplo de este primer
ejercicio, sin embargo, es una expresión sencilla.  Veremos casos más
complejos cuando la música lo requiera.
