\section{Ofrenda Musical, de Bach}


\subsection{Modelo}

Estudiaremos los títulos de cabecera y ejercitaremos las alteraciones
accidentales con este ejemplo de Bach:

\bigskip

\begin[staffsize=17.5,line-width=17\cm]{lilypond}
\header { title="Tema real"
          subtitle="de la \"Ofrenda musical\""
          composer="J.S. Bach"
}
\relative c'' {
\key c \minor
\time 2/2
c2 es g as b, r4
g' fis2 f e es~ es4 d des c b a8 g c4 f es2 d c4 }
\end{lilypond}

Casi todos los elementos de notación de este fragmento ya se han
estudiado.  Veamos, en los apartados siguientes, solamente los que
faltan.

\subsection{Títulos}

El título, subtítulo, autor y otros muchos encabezamientos se
especifican dentro de un bloque \verb+\header { ... }+ en la siguiente
forma:

\begin{verbatim}
\header {
  title    = "Título"
  subtitle = "Subtítulo"
  composer = "Autor"
}
\end{verbatim}

Si el propio encabezamiento contiene comillas, es necesario escribir
\verb+\"+ para imprimir cada una estas comillas.  Por ejemplo:

\begin{verbatim}
\header {
  title="Sonata \"Claro de luna\""
}
\end{verbatim}



\subsection{Compás}
Definimos el tipo de compás mediante la instrucción \verb+\time+ seguida de un quebrado:

\begin[verbatim,relative=2,staffsize=13]{lilypond}
\time 3/4
c4 c c
\time 6/8
c4. c
\time 2/4
c2
\time 2/2
c1
\end{lilypond}


\subsection{Ligadura de unión}

Utilizamos la tilde curva (en la tecla Alt Gr + 4) para unir dos notas
de idéntica altura:

\begin[verbatim,relative=1,staffsize=13,fragment]{lilypond}
c ~ c
\end{lilypond}


