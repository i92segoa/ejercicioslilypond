%\documentclass[12pt,a4paper,oneside]{scrbook} % la clase book del Koma-script bundle
\documentclass[a4paper,10pt,oneside,headinclude,titlepage]{article} % la clase book del Koma-script bundle
%\linespread{1.25}
\usepackage{setspace}
%\usepackage{tikz}
%\usetikzlibrary{fit,shapes}
\usepackage[spanish]{babel}
%\usepackage{verbatim} %para el entorno comment
%\usepackage{moreverb} %para los ejemplos de lilypond, aporta verbatimtabinput
%\usepackage{alltt} %para los ejemplos de lilypond, aporta verbatiminput
%\usepackage{sverb} %para los ejemplos de lilypond, aporta verbinput
%\usepackage{fancyvrb} %para los ejemplos de lilypond, aporta VerbatimInput
\pagestyle{empty}
\usepackage[utf8]{inputenc}
\usepackage[T1]{fontenc} %posiblemente sirva para eliminar el problema del enguionado de palabras acentuadas. Lo quitamos provisionalmente para evitar un error
\usepackage{textcomp} % recomendación de Javier Bezos para completar la fuente

\usepackage[margin=2cm]{geometry}
\usepackage{graphicx}
%\usepackage{url}

\usepackage[utopia]{mathdesign}
%\usepackage{mathptmx} %mejor que Times    % alternativa a Charter


%\typearea[0mm]{13}% same as class options above
%\usepackage{newcent}
%\addtokomafont{part}{\mdseries} %encabezamientos sin negrita
%\addtokomafont{partnumber}{\mdseries} %encabezamientos sin negrita
%\addtokomafont{chapter}{\mdseries} %encabezamientos sin negrita
%\setkomafont{disposition}{\normalcolor\bfseries} %no sans serif
%\setkomafont{disposition}{\normalcolor\mdseries} %no negrita

\parskip=6pt\clubpenalty=10000\widowpenalty=10000

\newcommand{\preLilyPondExample}{\vspace{-10pt}}

\newcommand{\lpversion}{2.13.4}
\newcommand{\defsep}{\textbf{$\|$}}
\newcommand{\software}{\emph{software}}
\newcommand{\negspace}{\vspace{-10pt}}  %{\vspace{-20pt}}
\newcommand{\seppar}{
\bigskip
%\vspace{6pt}
}

%%%%%%%%%%%%%%%%%%%%%%%%%%%%%%%%%%%%%%%%%%%%%%%%%%%%%%%%%%%%%%%%%%%%%%%%%%%%%%%%%%%%%%%%%%%
\begin{document}

\setcounter{section}{3}


\section{Ofrenda Musical, de Bach}


\subsection{Modelo}

Estudiaremos los títulos de cabecera y ejercitaremos las alteraciones
accidentales con este ejemplo de Bach:

\bigskip

\begin[staffsize=17.5,no-ragged-right]{lilypond}
\header { title="Tema real" 
          subtitle="de la \"Ofrenda musical\""
          composer="J.S. Bach"
}
\relative c'' {
\key c \minor
\time 2/2
c2 es g as b, r4
g' fis2 f e es~ es4 d des c b a8 g c4 f es2 d c4 }
\end{lilypond}

Casi todos los elementos de notación de este fragmento ya se han
estudiado.  Veamos, en los apartados siguientes, solamente los que
faltan.

\subsection{Títulos}

El título, subtítulo, autor y otros muchos encabezamientos se
especifican dentro de un bloque \verb+\header { ... }+ en la siguiente
forma:

\begin{verbatim}
\header {
  title    = "Título"
  subtitle = "Subtítulo"
  composer = "Autor"
}
\end{verbatim}

Si el propio encabezamiento contiene comillas, es necesario escribir
\verb+\"+ para imprimir cada una estas comillas.  Por ejemplo:

\begin{verbatim}
\header {
  title="Sonata \"Claro de luna\""
}
\end{verbatim}



\subsection{Compás}
Definimos el tipo de compás mediante la instrucción \verb+\time+ seguida de un quebrado:

\begin[verbatim,relative=2,staffsize=13]{lilypond}
\time 3/4
c4 c c
\time 6/8
c4. c
\time 2/4
c2
\time 2/2
c1
\end{lilypond}


\subsection{Ligadura de unión}

Utilizamos la tilde curva (en la tecla Alt Gr + 4) para unir dos notas
de idéntica altura:

\begin[verbatim,relative=1,staffsize=13,fragment]{lilypond}
c ~ c
\end{lilypond}


\end{document}

