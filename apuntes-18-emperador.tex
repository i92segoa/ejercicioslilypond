%\documentclass[12pt,a4paper,oneside]{scrbook} % la clase book del Koma-script bundle
\documentclass[a4paper,10pt,oneside,headinclude,titlepage]{article} % la clase book del Koma-script bundle
%\linespread{1.25}
\usepackage{setspace}
%\usepackage{tikz}
%\usetikzlibrary{fit,shapes}
\usepackage[spanish]{babel}
%\usepackage{verbatim} %para el entorno comment
%\usepackage{moreverb} %para los ejemplos de lilypond, aporta verbatimtabinput
%\usepackage{alltt} %para los ejemplos de lilypond, aporta verbatiminput
%\usepackage{sverb} %para los ejemplos de lilypond, aporta verbinput
%\usepackage{fancyvrb} %para los ejemplos de lilypond, aporta VerbatimInput
\pagestyle{empty}
\usepackage[utf8]{inputenc}
\usepackage[T1]{fontenc} %posiblemente sirva para eliminar el problema del enguionado de palabras acentuadas. Lo quitamos provisionalmente para evitar un error
\usepackage{textcomp} % recomendación de Javier Bezos para completar la fuente

\usepackage[margin=2cm]{geometry}
\usepackage{graphicx}
%\usepackage{url}

\usepackage[utopia]{mathdesign}
%\usepackage{mathptmx} %mejor que Times    % alternativa a Charter


%\typearea[0mm]{13}% same as class options above
%\usepackage{newcent}
%\addtokomafont{part}{\mdseries} %encabezamientos sin negrita
%\addtokomafont{partnumber}{\mdseries} %encabezamientos sin negrita
%\addtokomafont{chapter}{\mdseries} %encabezamientos sin negrita
%\setkomafont{disposition}{\normalcolor\bfseries} %no sans serif
%\setkomafont{disposition}{\normalcolor\mdseries} %no negrita

\parskip=6pt\clubpenalty=10000\widowpenalty=10000

\newcommand{\preLilyPondExample}{\vspace{-10pt}}

\newcommand{\lpversion}{2.13.4}
\newcommand{\defsep}{\textbf{$\|$}}
\newcommand{\software}{\emph{software}}
\newcommand{\negspace}{\vspace{-10pt}}  %{\vspace{-20pt}}
\newcommand{\seppar}{
\bigskip
%\vspace{6pt}
}

%%%%%%%%%%%%%%%%%%%%%%%%%%%%%%%%%%%%%%%%%%%%%%%%%%%%%%%%%%%%%%%%%%%%%%%%%%%%%%%%%%%%%%%%%%%
\begin{document}

\setcounter{section}{17} %para 18 emperador


\section{Polifonía compleja: la ``Canción del Emperador''.}


\subsection{Modelo}

Esta versión para guitarra de la ``Canción del Emperador'' de Luis de
Narváez, sobre el tema ``Mille Regretz'', es una transcripción de las
tablaturas originales y presenta una polifonía enrevesada porque todas
las voces están contenidas en un solo pentagrama.  Para este ejemplo
hará falta una cuidadosa planificación y el empleo de silencios
ocultos.  Además contiene indicaciones del número de cuerda, silencios
con altura definida y otros ajustes menores.  Por sencillez, el resto
de los ajustes necesarios se omiten por el momento.

\bigskip

%\hspace{3cm}
\begin[staffsize=15,line-width=14\cm,indent=1.5\cm]{lilypond}

% canción del emperador. Narváez
\header { title = "CANCIÓN DEL EMPERADOR" 
	composer= "Luys de Narváez"
	opus =  "(1530-1550)"
} 
\version "2.12.0"

cantus = \relative c'{ 
	<b fis' fis\4 b>1
	<b b' b\3 >
	<c e'>
	<c' e>
	<b d> \break %5
	d2 c4 b
	a2 g
	a1
	r2 <b g'>2
	<b g'> g' \break %10
	fis fis4 g
	e2 e
	r4 g8_( fis) e d e fis
	<g, b g'>2 <g b g'> \break
	fis'8( g) fis g fis e d c
	b2  e  ~ 
	e4 e4 dis cis
	<dis fis,>2 <dis fis,>   
 }


altus = \relative c'' {
	s1
	s1
	g4 a8 b c d c b
	a,8 b c d e4 a,
	b8 cis d e fis4 b, % \break %5
	b4 d a g
	c4\rest dis e2 ^~
	e4 dis e dis
	\stemUp g2 s2
	s2 \voiceThree b4 cis %10
	\shiftOff d1
	c1
}

tenor = \relative c { \voiceTwo 
	s1
	s1
	s1
	s1
	s1 %5
	s1
	fis2 r4 e
	fis1
	<e e'>1
	e2 e'
	d2 b
	c2 a8 b c d
	<e g b>1
	e2 e
	<d a'>1
	<e g>1
	<b fis'>1
	b2 b
} %5

\score {
\new Staff \relative c' { \set Staff.instrumentName = "Guitarra"
    \time 4/4
    \key g \major

    <<
	\new Voice { \voiceOne \cantus }
	\new Voice { \voiceFour \altus } 
	\new Voice { \voiceTwo \tenor }
    >>
  }

\layout { left-margin=0\cm
%, indent=2\cm
}

}
\end{lilypond}




\subsection{Silencios ocultos o de separación}
\label{sub:ocultos}

Será de gran ayuda, para la realización de partituras de polifonía
compleja, la inserción de silencios de separación.  Éstos no se
imprimen pero ocupan el mismo espacio que una figura con la duración
correspondiente.  Para insertarlos se utiliza \verb+s+ como si fuera
una nota; en el siguiente ejemplo hemos rellenado la voz inferior con
un silencio de blanca oculto:

\begin[relative=2,verbatim,staffsize=17.5]{lilypond}
\new Staff <<
  { c4 d e f }
  \\
  { a,4 s2 d4 }
>>
\end{lilypond}


\subsection{Silencios con altura. Ligaduras orientadas}

Los silencios se suelen colocar automáticamente de forma que no haya
colisiones con las notas de las otras voces.  Sin embargo, si queremos
colocar un silencio a la altura de una nota determinada, lo hacemos
mediante \verb+\rest+ que convierte la nota anterior en un silencio.  

\begin[relative=2,verbatim,staffsize=17.5]{lilypond}
c4 g'8\rest g, c2 ^~ c1 _~ c
\end{lilypond}

En este ejemplo, además, hemos utilizado los indicadores de dirección
para orientar la ligadura de unión hacia arriba o hacia abajo.

\subsection{Planificación de las voces en polifonía compleja.}

Existen varias técnicas para resolver el problema de la polifonía en
casos similares al de arriba.  Una solución es preparar una
construcción polifónica \verb+<< { } \\ { } >>+ por cada compás o por
cada pocos compases.  Hoy recomendamos un enfoque orientado a voces
que se extienden a lo largo de toda la pieza, quizá utilizando
silencios ocultos como se explica en el apartado \ref{sub:ocultos}.


No es necesario que los acordes de redonda pertenezcan a distintas
voces.  Aquí usamos, simplemente, un acorde:

\begin[verbatim,staffsize=17.5]{lilypond}

vozUno = \relative c''{ <g e'>1 }
vozDos = \relative c' { r2 c8 d e f }
\new Staff << \vozUno \vozDos >>

\end{lilypond}


Para la orientación adecuada de las plicas y el desplazamiento de las
voces secundarias, tendremos en cuenta que las dos primeras voces
tienen las plicas en direcciones opuestas; las voces tercera y cuarta,
además, llevan un desplazamiento a la derecha.  El mismo efecto puede
conseguirse en cualquier momento gracias a las instrucciones
\verb+\voiceOne+, \verb+\voiceTwo+, \verb+\voiceThree+, y
\verb+\voiceFour+.

En el siguiente ejemplo utilizamos \verb+\voiceOne+ y
\verb+\voiceThree+ para que las dos voces tengan las plicas hacia
arriba, y la voz de contralto tenga un desplazamiento horizontal a la
derecha:

\begin[verbatim,staffsize=17.5]{lilypond}
soprano   = { g''1 }
contralto = { b'2 cis'' }
\new Staff << \new Voice { \voiceOne \soprano }
              \new Voice { \voiceThree \contralto } >>
\end{lilypond}


En caso necesario puede usarse \verb+\shiftOff+ para anular el
desplazamiento de una voz secundaria.  Podemos recurrir a
\verb+\stemUp+ y \verb+\stemDown+ para orientar las plicas hacia
arriba o hacia abajo:

\begin[verbatim,relative=2,staffsize=17.5]{lilypond}
\stemUp c4 b a g \stemDown f e d c
\end{lilypond}



\subsection{Saltos de línea manuales.}

En ocasiones conviene insertar un cambio de línea manual: lo hacemos
con \verb+\break+, aunque sólo se producirá el salto si en el momento
actual es posible saltar.  Lo podemos comprobar aquí:

\begin[relative=2,verbatim,staffsize=17.5]{lilypond}
c4 c c \break c
c1 \break
c1 c
\end{lilypond}

\subsection{Números de cuerda.}

Las cuerdas de la guitarra se indican mediante un número dentro de un
círculo. Las escribimos con \verb+\1+, \verb+\2+, etc.


\end{document}

