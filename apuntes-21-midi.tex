\section{MIDI. Transposición. Il est bel et bon, de P. Passereau}


\subsection{Modelo}

Hoy mostraremos la forma de escuchar las partituras elaboradas con
LilyPond.  A partir de una partitura se puede generar un pequeño
archivo MIDI que contiene las notas, pero no los sonidos: éstos se
generan cuando el archivo se reproduce y pueden sonar algo distintos
en cada sistema.  Sin embargo, podemos elegir los instrumentos que
determinarán el timbre de cada pentagrama.

Aprovecharemos para realizar un transporte sobre el tono original, un
tono y medio hacia arriba.

No todos los elementos de la música se exportan al archivo MIDI. Antes
de introducir el modelo, el comienzo de \emph{Il est bel et bon}, una
la \emph{chanson} de Pierre Passereau (s.XVI), analizaremos qué
elementos merece la pena omitir si solamente queremos producir un
archivo MIDI para escuchar la música.  En primer lugar, el ejemplo
original:


\bigskip

\begin[staffsize=15,
line-width=17\cm,
indent=0
]{lilypond}

%#(set-global-staff-size 14)

%\version "2.13.16"

 \header { title = "Il est bel et bon" composer = "Pierre Passereau" }

sop = \relative c' { %\tempo 1=60
	% \key d \minor
	\time 2/2
	d8 e f g a4 a
	a4 a a a
	g4 e r2
	d8 e f g a4 a
	a4 a a a
	g4 e8 e g4 e8 e
	f4 d d c d2 }

alt = \relative c' { R1
	a8 b c d e4 e
	e4 e e e
	f2 e
	a,8 b c d e4 e
	e4 e b b8 b
	d4 a a a
a2 }

ten = \relative c { \clef "G_8"
	r2 d8 e f g
	a4 a c c
	b4 b g2
	a2 a8 g f g
	a4 a c c
	b2 g4 g
	a4. g8 f4 e
d2 }

baj = \relative c { \clef bass
	R1
	r2 a8 b c d
	e4 e e e
	d2 a~
	a2 a8 b c d
	e4 e e e
	d4 d a a
d2 }


letrasop = \lyricmode { Il est bel et bon, bon,
	bon, bon, bon, com --
	mè -- re,
	Il est bel et bon, bon,
	bon, bon, bon, com --
	mè -- re, com -- mè -- re, com --
	mè -- re mon ma -- ry. }

letraalt = \lyricmode { Il est bel et bon, bon,
	bon, bon, bon, com --
	mè -- re,
	Il est bel et bon, bon,
	bon, com -- mè -- re, com --
	mè -- re, mon ma -- ry. }


letraten = \lyricmode { Il est bel et
	bon, bon, bon, bon,
	bon, com -- mè --
	re, Il est bel et
	bon, bon, bon, com --
	mè -- re, com --
	mè -- re mon ma -- ry. }

letrabaj = \lyricmode { Il est bel et
	bon, bon, bon, com --
	mè -- re,
	Il est bel et
	bon, bon, bon, com --
	mè -- re mon ma -- ry. }


\score {
	% \transpose d f
	\new ChoirStaff
	<<
		\new Staff { \set Staff.instrumentName="S"
                  \set Staff.midiInstrument = "oboe"
		\sop }
		\addlyrics { \letrasop }
		\new Staff { \set Staff.instrumentName="A"
                  \set Staff.midiInstrument = "oboe"
		\alt }
		\addlyrics { \letraalt }
		\new Staff { \set Staff.instrumentName="T"
                  \set Staff.midiInstrument = "bassoon"
		\ten  }
		\addlyrics { \letraten }
		\new Staff { \set Staff.instrumentName="B"
                  \set Staff.midiInstrument = "clarinet"
		\baj }
		\addlyrics { \letrabaj }
>>
%\midi { }
%\layout { }
}

\paper { system-count=2 indent=1\cm }

\end{lilypond}


Ahora bien: los nombres de los pentagramas, la letra de la canción,
las agrupaciones de pentagramas que trazan llaves, los títulos, las
articulaciones y otros elementos no se reflejan el el resultado MIDI.
La indicación metronómica sí se refleja; así pues, bastaría con
dejarlo de esta manera:

\bigskip

\begin[staffsize=15,
line-width=17\cm,
indent=0
]{lilypond}

%#(set-global-staff-size 14)

%\version "2.13.16"

% \header { title = "Il es bel et bon" composer = "Pierre Passereau" }

sop = \relative c' { \tempo 1=60
	% \key d \minor
	\time 2/2
	d8 e f g a4 a
	a4 a a a
	g4 e r2
	d8 e f g a4 a
	a4 a a a
	g4 e8 e g4 e8 e
	f4 d d c d2 }

alt = \relative c' { R1
	a8 b c d e4 e
	e4 e e e
	f2 e
	a,8 b c d e4 e
	e4 e b b8 b
	d4 a a a
a2 }

ten = \relative c { \clef "G_8"
	r2 d8 e f g
	a4 a c c
	b4 b g2
	a2 a8 g f g
	a4 a c c
	b2 g4 g
	a4. g8 f4 e
d2 }

baj = \relative c { \clef bass
	R1
	r2 a8 b c d
	e4 e e e
	d2 a~
	a2 a8 b c d
	e4 e e e
	d4 d a a
d2 }


\score {
	% \transpose d f
	% \new ChoirStaff
	<<
		\new Staff { \set Staff.midiInstrument = "oboe"
		\sop }
		%\addlyrics { \letrasop }
		\new Staff { \set Staff.midiInstrument = "oboe"
		\alt }
		%\addlyrics { \letraalt }
		\new Staff { \set Staff.midiInstrument = "bassoon"
		\ten  }
		%\addlyrics { \letraten }
		\new Staff { \set Staff.midiInstrument = "clarinet"
		\baj }
		%\addlyrics { \letrabaj }
>>
%\midi { }
%\layout { }
}

\paper { system-count=1 }


\end{lilypond}


\subsection{El bloque midi}

Para producir un archivo MIDI, debe existir un bloque \verb+\midi{ }+,
posiblemente vacío, dentro de un bloque \verb+\score+ explícito que
contiene la música.  En nuestro ejemplo, después de definir las
variables (en su caso), podemos hacerlo así:

\begin{verbatim}
\score {
  <<
     \new Staff { \soprano }
     \new Staff { \alto }
     \new Staff { \tenor }
     \new Staff { \bajo }
  >>
  \midi { }
}
\end{verbatim}

Si existe un bloque \verb+\midi{}+ y no hay un bloque \verb+layout{}+, el proceso será muy rápido pero no habrá ninguna salida en PDF.  Para obtener MIDI y PDF, deben estar los dos bloques, así:

\begin{verbatim}
\score {
    ...

  \midi { }
  \layout{ }
}
\end{verbatim}

El archivo MIDI tiene la extensión \verb+.mid+ (en UNIX: \verb+.midi+)
y se creará en el mismo directorio que el archivo fuente.  En teoría,
debería poder reproducirse fácilmente mediante un doble click.

\subsection{Instrumentos MIDI}

 Es posible (y así lo haremos) determinar un sonido MIDI por cada voz
 y cada pentagrama.  En esta ocasión lo haremos estableciendo el valor
 de la propiedad \verb+midiInstrument+ del contexto Score, a un texto
 que corresponderá exactamente al nombre oficial del instrumento según
 el estándar General Midi, que puede consultarse en las tablas de la
 documentación de LilyPond.

 Para \emph{Il est bel et bon} vamos a asignar a las voces de soprano
 y contralto un sonido de oboe; a la de tenor, de fagot (``bassoon'');
 y a la de bajo, un sonido de clarinete (``clarinet'') (aunque es muy
 grave para el clarinete, podría ser un clarinete bajo).

\begin{verbatim}
\new Staff {
  \set Staff.midiInstrument = "oboe"
  \soprano
}
\end{verbatim}




\subsection{Transposición}

Ahora queremos que la música suene un tono y medio más alta.
Utilizamos para ello la instrucción \verb+\transpose+ previa a la
expresión musical.  La instrucción admite dos alturas de nota que se
toman como referencia del tono de partida y del de destino de la
transposición.  Lo vemos en este ejemplo donde hemos introducido
música en Sol mayor y la hemos impreso en Fa mayor (un tono más baja):

\begin[verbatim,staffsize=17.5]{lilypond}
\transpose g f
  \relative c'' { \key g \major
     g4 a b c d1
  }
\end{lilypond}

Se transporta la armadura solamente si está establecida dentro de la
expresión que se transporta.
