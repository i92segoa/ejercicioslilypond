\section{Contextos explícitos. Música simultánea}


\subsection{Modelo}

El presente ejemplo contiene música a dos voces en dos pentagramas:

\bigskip

\begin[staffsize=17.5,line-width=17\cm]{lilypond}

<<

\new Staff \relative c''' {
\time 12/8 \key f\major
    c4. ~ c8 b a g4. ~ g8 f e |
    d b c f4. ~ f8 e g c4 es,8

}

\new Staff \relative c'' {
\time 12/8 \key f\major
    e16( d )e c g c f( e )f d b d g( f )g e c e a( g )a f g a |
    b,8 g' c, ~ c a b!-\trill c16( b )c g e g f( g a bes )c a }

>>


\end{lilypond}


\subsection{Contextos explícitos}

La construcción

\verb+{ música }+

es una abreviatura de

\verb+\new Staff { \new Voice { música } }+

y es suficiente para la mayoría de las aplicaciones sencillas. Staff
(pentagrama) y Voice (Voz) son contextos; los contextos contienen
música.  Muchas veces un contexto se crea de forma implícita allí
donde se necesita.  Sin embargo, es conveniente declarar de forma
explícita al menos el contexto de pentagrama (la parte
\verb+\new Staff+) para tener un mayor control sobre los pentagramas
que se crean.

\subsection{Música simultánea}

Dos o más expresiones encerradas entre ángulos dobles, \verb+<< >>+,
se imprimen como música simultánea.  La tonalidad no se hereda de una
expresión a otra, pero la indicación de compás es común:

\begin[verbatim,relative=2,staffsize=13]{lilypond}
<<
 \relative c' {
   \key f \major
   \time 2/4
   c d e g }
 \relative c' {
   e d c b }
>>
\end{lilypond}

Observemos que los dos pentagramas están en compás de 2/4 pero sólo el
de arriba está en Fa mayor.

\subsection{Trinos}

La instrucción \verb+\trill+ después de una nota, unida mediante un
guión, produce una indicación de trino:

\begin[verbatim,relative=1,staffsize=13]{lilypond}
{ c-\trill }
\end{lilypond}


