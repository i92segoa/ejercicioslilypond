
\section{Uso de LilyPond bajo Windows}
\subsection{Descarga e instalación}

LilyPond está disponible para su descarga gratuita en la página
oficial del proyecto, \texttt{lilypond.org}.  Después de
ejecutar el instalador, la aplicación está lista para su uso.

\subsection{Creación de una partitura sencilla}

Elegiremos el Escritorio para la realización de nuestro primer
ejemplo. Para ello, abrimos el accesorio «Bloc de notas» de Windows y
escribimos lo siguiente\footnote{Las llaves se consiguen con AltGr
  pulsando una tecla que en los teclados de PC suele estar junto a la
  'Ñ'. Los apóstrofos se consiguen mediante la tecla que está justo a
  la derecha del número 0.}:

\begin{quote}
\begin{verbatim}
{  c' d' e' f' g'2 e' }
\end{verbatim}
\end{quote}

Denominamos a este texto \emph{código de entrada}.

Guardamos este texto con un nombre terminado en la extensión
\verb+.ly+, por ejemplo \verb+prueba.ly+, con las siguientes
precauciones:

\begin{enumerate}
\item En la lista desplegable «Guardar como archivo de tipo\ldots» del
  diálogo de Guardar, elegimos «Todos los archivos (*.*)»
\item En la lista desplegable «Codificación» es necesario seleccionar
  «UTF-8».
\end{enumerate}

Denominaremos a este archivo con la extensión \verb+.ly+ que
contiene el código de entrada, \emph{archivo fuente} o
\emph{archivo de entrada}.

\subsection{Procesar el documento}

El programa LilyPond no se utiliza para editar la partitura, sino para
producir una salida en formato PDF a partir del documento de texto que
hemos preparado. Esto se denomina \emph{procesar el código de
  entrada}. Para procesar el código de entrada, si el icono del
documento está seleccionado, pulsamos la tecla Enter. Si no, podemos
hacer doble click sobre él con el ratón. También podemos pulsar con el
botón derecho y elegir del menú «Procesar documento». En cualquier
caso LilyPond hará su trabajo: interpretar el código de entrada y
producir una salida.

El procesado tarda un par de segundos\footnote{La primera vez, el
  programa tiene que preparar las fuentes tipográficas; esto lleva
  aproximadamente medio minuto, pero las ejecuciones posteriores
  tardan, como se ha dicho, unos segundos.}. El resultado es un
archivo PDF que en nuestro caso se llamará \verb+prueba.pdf+ y que
puede examinarse con cualquier visor de documentos en este formato,
como por ejemplo Acrobat Reader. El resultado es el siguiente:

\begin[staffsize=15,fragment]{lilypond}
  c' d' e' f' g'2 e' 
\end{lilypond}


\subsection{Expresiones musicales de LilyPond}

Una partitura completa de LilyPond es un fragmento de música encerrado
entre llaves \verb+{ }+, llamado \emph{expresión musical}.  En los
ejemplos que aparecen en este cuaderno de ejercicios, con frecuencia
se omiten las llaves por sencillez, pero no deben olvidarse cuando se
inserte el código en los ejercicios propios.

De igual forma que las expresiones matemáticas, en LilyPond podemos
ampliar una expresión añadiéndole operadores a la izquierda o
combinando varias de ellas en una sola; el ejemplo de este primer
ejercicio, sin embargo, es una expresión sencilla.  Veremos casos más
complejos cuando la música lo requiera.
