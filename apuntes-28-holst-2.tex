\section{Sobreescritura de propiedades: Marte, de Holst (II)}


\subsection{Modelo}

Para completar el ejemplo orquestal, hoy aprenderemos a mover
objetos para ajustar su posición; en el caso que nos ocupa, esto
ahorra espacio y permite un tamaño de los pentagramas algo mayor,
sin que se produzcan colisiones entre los objetos de los distintos
pentagramas, y todo ello de forma que la música quepa en una sola
página.  En el título usaremos el efecto \emph{smallcaps} de
mayúsculas pequeñas para ``Mars''.

En el fragmento aparecen los pentagramas de los fagotes y el
contrafagot, con el matiz \emph{mezzopiano} y la indicación
``III'' del tercer fagot desplazadas a la izquierda y hacia
arriba.  Se ha enmascarado en blanco el pentagrama detrás de estas
indicaciones para evitar la superposición.  Las pautas de
percusión se han acercado entre sí para ahorrar espacio. La
indicación \emph{piano} de los violines está también desplazada
para hacer sitio al texto \emph{col legno} del siguiente
pentagrama.  Para finalizar, hemos reducido el grosor de las
líneas de pauta para suavizar el aspecto demasiado negro de una
partitura orquestal a tamaño reducido.

\bigskip

% Aumentar la separación entre sistemas
\def\betweenLilyPondSystem#1{\vspace{0.4cm}\linebreak}

\begin[line-width=13cm]{lilypond}

\version "2.13.0"

juntaPentagrama = \with { \override VerticalAxisGroup #'next-staff-spacing =
                   #'((space . 6) (padding . 0)) 
		   }


		bassoonsI =  \relative  g, {
			\clef bass
			\oneVoice R1*5/4 R1*5/4
			\voiceOne g2. ^"I II a2" ~ ( \p  g2 ~
			g2. ^\< d'2 ) \!  des2. ^\> ~ des2 \! \laissezVibrer % ~ des
		}

		bassoonsIII =  \relative  d, {
			\clef bass
			s1*5/4 s1*5/4 R1*5/4*2
			%  \once \override Voice.DynamicText #'extra-offset = #'(-2.9 . 2.9) 
			\override TextScript #'whiteout = ##t
			\override DynamicText #'whiteout = ##t
			\once \override DynamicText #'X-offset = #-4.5
			\once \override DynamicText #'extra-offset = #'(-0.1 . 2.3)
			\once \override TextScript #'outside-staff-priority = ##f
			\once \override TextScript #'X-offset = #-4
			des2.
			-"III"
			 \mp
			 \>  ~ des2 \! \laissezVibrer % ~ des
		}


%%%%%%%%%%%%%%%%%%%%%%%%%%%%%%%%%%%%%%%%%%%%%%%%%%%%%%%%%%%%%%%%%%%%%%%%%%%%%%%%%%%%%%%%%%%%%%

             doble = \relative  g, { \key c \major
			\clef bass
			R1*5/4 R1*5/4
			g2.\p ~ ( g2 ~ g2. \< d'2 \! ) des2. \> ~ des2 \! \laissezVibrer % ~ des
	     }

%%%%%%%%%%%%%%%%%%%%%%%%%%%%%%%%%%%%%%%%%%%%%%%%%%%%%%%%%%%%%%%%%%%%%%%%%%%%%%%%%%%%%%%%%%%%%%

%%%%%%%%%%%%%%%%%%%%%%%%%%%%%%%%%%%%%%%%%%%%%%%%%%%%%%%%%%%%%%%%%%%%%%%%%%%%%%%%%%%%%%%%%%%%%%%%%%%

	side  = { R1*5/4*5 }

%%%%%%%%%%%%%%%%%%%%%%%%%%%%%%%%%%%%%%%%%%%%%%%%%%%%%%%%%%%%%%%%%%%%%%%%%%%%%%%%%%%%%%%%%%%%%%%%%%%

	cymbals  = { R1*5/4*5 }

%%%%%%%%%%%%%%%%%%%%%%%%%%%%%%%%%%%%%%%%%%%%%%%%%%%%%%%%%%%%%%%%%%%%%%%%%%%%%%%%%%%%%%%%%%%%%%%%%%%

	drum  = { R1*5/4*5 }


       violinI = \relative g {
	\once \override DynamicText #'extra-offset = #'(-0.8 . 1)
	\once \override DynamicText #'X-offset = #-2.5
	\times 2/3 { g8\p ^"col legno" g g }  g4 g g8 g g4
	\times 2/3 { g8 g g } g4 g g8 g g4
	\times 2/3 { g8 g g } g4 g g8 g g4
	\times 2/3 { g8 \< g g } g4 g g8 g g4\!
	\times 2/3 { g8 \> g g } g4 g g8 g g4\! }

%%%%%%%%%%%%%%%%%%%%%%%%%%%%%%%%%%%%%%%%%%%%%%%%%%%%%%%%%%%%%%%%%%%%%%%%%%%%%%%%%%%%%%%%%%%%%%%%%%%%%%

	violinII = \relative g {
		\once \override DynamicText #'extra-offset = #'(-0.8 . 1)
		\once \override DynamicText #'X-offset = #-2.5
		\times 2/3 { g8 \p ^"col legno" g g }  g4 g g8 g g4
		\times 2/3 { g8 g g } g4 g g8 g g4
		\times 2/3 { g8 g g } g4 g g8 g g4
		\times 2/3 { g8 \< g g } g4 g g8 g g4\!
		\times 2/3 { g8 \> g g } g4 g g8 g g4\! }


        #(set-global-staff-size 10.5)  % antes 15.5 para a3
	#(set-default-paper-size "a4") % antes a3

\header {
	title = \markup { \fontsize #6 { \smallCaps {  "I. Mars, " } "the Bringer of War" } }
	tagline=##f
}


\score {

    % main
    \new StaffGroup <<   \tempo "Allegro"
	\time 5/4

    %bassoons
    \new PianoStaff  <<
	\new Staff  \with { instrumentName = "3 Bassoons" } { << \bassoonsI \\ \bassoonsIII >> }
	\new Staff  \with { instrumentName = "Double Bassoon" } { \doble }  >>

    %side drum
    \new RhythmicStaff
         \with { \juntaPentagrama
                 instrumentName = "Side Drum" }
               { \side }

    % cymbals
    \new RhythmicStaff
         \with { \juntaPentagrama
                 instrumentName = "Cymbals" }
	       { \cymbals  }

    %bass drum
    \new RhythmicStaff
         \with { \juntaPentagrama
                 instrumentName = "Bass Drum" }
	{ \drum }


    %violins
    \new PianoStaff  <<
	\new Staff \with { instrumentName = "1st Violins" }
		{ \violinI }
	\new Staff \with { instrumentName = "2nd Violins" }
		{ \violinII } >>

>> %main

   \layout { indent=1.5\cm %era 4 para a3

	      \context { \Score
	      \override StaffSymbol #'thickness = #(magstep -3)

	      }
   }


} %score


\paper { ragged-right=##f
       line-width=16.5\cm
	 page-count=1
	 system-count=1
}

\end{lilypond}


\subsection{Sobreescritura de propiedades}
\label{override}

Es importante aprovechar al máximo las posibilidades de tipografiado
automático de partituras que LilyPond ofrece, sin ninguna intervención
manual.  Sin ambargo, en el apartado \ref{tamano-global}
(pág. \pageref{tamano-global}) utilizamos tímidamente la
sobreescritura de propiedades para modificar el tamaño de un
pentagrama.  Las propiedades de los objetos gráficos tienen un valor
determinado que se usa para especificar la forma en que el objeto se
imprime.  Hay varias instrucciones que hacen posible la modificación
de estos valores, y la más frecuente es \verb+\override+.  Los valores
exactos son algo que se puede determinar mediante ensayo y error,
aunque existen ayudas muy valiosas como la herramienta Regla de
LilyPondTool (que no explicaremos aquí).  La
instrucción \verb+\override+ se utiliza de la siguiente manera:

\begin{verbatim}
\override contexto.objeto #'propiedad = #valor
\end{verbatim}

Que significa: asignar el \emph{valor} a la \emph{propiedad}
del \emph{objeto} dentro del \emph{contexto}.  El contexto
predeterminado es Voice y muchas veces se puede dejar sin
especificar.  Veamos a continuación un ejemplo del uso de la
sobreescritura de propiedades para mover objetos.


\subsection{Mover objetos}

Los matices dinámicos son objetos llamados internamente
DynamicText, que se imprimen en el lugar determinado por una serie
de variables.  Apliquemos la formulación general de la
instrucción \verb+\override+ que acabamos de mostrar, y
consignemos lo siguiente para cada uno de los apartados:

\medskip

\begin{tabular}{c|c|c|c}
Contexto & Objeto & Propiedad & Valor \\ \hline
Voice & DynamicText & 'extra-offset & '(-0.8 . 1) \\
Voice & DynamicText & 'X-offset & -2.5
\end{tabular}

\medskip

Estas medidas están expresadas en espacios de pentagrama, por lo que
(afortunadamente) no dependen del tamaño de éste.  Los dos números
entre paréntesis se refieren a las dimensiones X e Y.  El efecto de la
sobreescritura permanece hasta que se vuelva a sobreescribir o hasta
que se encuentre una instrucción \verb+\revert+ con el nombre del
objeto y la propiedad.  En el ejemplo se ve que las tres indicaciones
están afectadas por una sola sobreescritura:

\begin[fragment,verbatim]{lilypond}
    \override DynamicText #'extra-offset = #'(-0.8 . 1)
    \override DynamicText #'X-offset = #-2.5
    g1\p g\p g\p
\end{lilypond}


\subsection{Aplicación por una sola vez}

Las sobrreescrituras permanecen hasta nueva orden, pero por
comodidad, en caso de que sólo se necesite una vez, podemos
preceder la instrucción de sobreescritura por la palabra
clave \verb+\once+.  Aquí podemos ver que sólo la primera
indicación dinámica está afectada por \verb+\once \override+:

\begin[fragment,verbatim]{lilypond}
    \once \override DynamicText #'extra-offset = #'(-0.8 . 1)
    \once \override DynamicText #'X-offset = #-2.5
    g1\p g\p  g\p
\end{lilypond}


\subsection{Enmascarar en blanco}

Cuando se quieren tapar las líneas que caen detrás de una
indicación dinámica o textual, se le da un valor verdadero a la
propiedad \verb+whiteout+.

\medskip
\begin{tabular}{c|c|c|c}
Contexto & Objeto & Propiedad & Valor \\ \hline
Voice & TextScript, DynamicText  & 'whiteout & verdadero (\#t) o falso (\#f) \\
\end{tabular}
\medskip

Por ejemplo:
\begin{verbatim}
\override TextScript #'whiteout = ##t
\end{verbatim}

\begin{lilypond}
\new PianoStaff <<\new Staff 
	{ 
	\voiceTwo c'1 
	\override TextScript #'whiteout = ##t
	\override DynamicText #'whiteout = ##t
	
	\once \override TextScript #'outside-staff-priority = ##f
	\once \override TextScript #'X-offset = #-3

	\once \override DynamicText #'X-offset = #-4.5
	\once \override DynamicText #'extra-offset = #'(0 . 1.4)
	 c'2  
	\mp

	-"III"
	c'2 
}

\new Staff { c'1 c' }
>>
\end{lilypond}


\subsection{Grosor de las líneas del pentagrama}

Pruebe la siguiente sobreescritura para conseguir líneas más
delgadas en pautas sueltas o en toda la partitura:

\medskip
\begin{tabular}{c|c|c|c}
Contexto & Objeto & Propiedad & Valor \\ \hline
Staff, Score & StaffSymbol  & 'thickness & \#(magstep -3) \\
\end{tabular}
\medskip

Por ejemplo:
\begin{verbatim}
\new Staff \with { \override StaffSymbol #'thickness = #(magstep -3) }
\end{verbatim}

En el ejemplo que aparece a continuación podemos ver dos aplicaciones
de sentido opuesto, y el aspecto predeterminado en segundo lugar.


\begin[staffsize=10]{lilypond}
<<
  \new Staff \with { \override StaffSymbol #'thickness = #(magstep -6) } { s1 -"-6" }
  \new Staff \with { \override StaffSymbol #'thickness = #(magstep 0) } { s1 -"0"}
  \new Staff \with { \override StaffSymbol #'thickness = #(magstep 6) } { s1 -"+6" }
>>
\end{lilypond}


\subsection{Separación de pautas}

El espaciado vertical es un asunto delicado.  El ajuste de la
separación entre cada pauta y la siguiente se hace también
mediante sobreescritura de propiedades, pero desde la versión de
desarrollo 2.13 de LilyPond los objetos y propiedades han cambiado
de nombre y la sobreescritura tiene otro formato.  En cualquier
caso aquí tenemos las que valdrán para la futura versión estable
2.14:

\medskip
\begin{tabular}{c|c|c|c}
Contexto & Objeto & Propiedad & Valor \\ \hline
Staff & VerticalAxisGroup  & 'next-staff-spacing & \#((space . 6) (padding . 0)) \\
\end{tabular}
\medskip

Esta sobreescritura se puede almacenar en una variable para utilizarla
repetidas veces (sólo para 2.13):

\begin[verbatim]{lilypond}
juntaPauta = \with {
                    \override VerticalAxisGroup #'next-staff-spacing =
                      #'((space . 3) (padding . 0))
                  }
<<
  \new RhythmicStaff
    \with {
      instrumentName= "Side Drum"
      \juntaPauta
    }
    { c4 c c8 c c4 }
  \new RhythmicStaff
    \with {
      instrumentName= "Cymbals"
      \juntaPauta
    }
    { c4 c c8 c c4 }
>>
\end{lilypond}


\subsection{Notas}

\begin{itemize}
\item Para los nombres de instrumentos que contienen un bemol, use \verb+\flat+ dentro del elemento de marcado.

\begin[verbatim]{lilypond}
\new Staff
  \with {
    instrumentName= \markup {  "3 Clarinets in B" \flat }
  }
  s1
\end{lilypond}

\item La instrucción de marcado \verb+\smallCaps+ produce un estilo ``versalitas'' en que las minúsculas son mayúsculas pequeñas:

\begin[verbatim]{lilypond}
\markup { \smallCaps "Marte" }
\end{lilypond}


\end{itemize}
