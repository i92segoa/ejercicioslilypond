%\documentclass[12pt,a4paper,oneside]{scrbook} % la clase book del Koma-script bundle
\documentclass[a4paper,10pt,oneside,headinclude,titlepage]{article} % la clase article estándar
%\linespread{1.25}
\usepackage{setspace}
\usepackage{pdfpages}
%\usepackage{tikz}
%\usetikzlibrary{fit,shapes}
\usepackage[spanish]{babel}
%\usepackage{verbatim} %para el entorno comment
%\usepackage{moreverb} %para los ejemplos de lilypond, aporta verbatimtabinput
%\usepackage{alltt} %para los ejemplos de lilypond, aporta verbatiminput
%\usepackage{sverb} %para los ejemplos de lilypond, aporta verbinput
%\usepackage{fancyvrb} %para los ejemplos de lilypond, aporta VerbatimInput
%\pagestyle{empty}
\usepackage[utf8]{inputenc}
\usepackage[T1]{fontenc} %posiblemente sirva para eliminar el problema del enguionado de palabras acentuadas. Lo quitamos provisionalmente para evitar un error
\usepackage{textcomp} % recomendación de Javier Bezos para completar la fuente

\usepackage[margin=2cm]{geometry}
\usepackage{graphicx}
%\usepackage{url}

\usepackage[utopia]{mathdesign}
%\usepackage{mathptmx} %mejor que Times    % alternativa a Charter


%\typearea[0mm]{13}% same as class options above
%\usepackage{newcent}
%\addtokomafont{part}{\mdseries} %encabezamientos sin negrita
%\addtokomafont{partnumber}{\mdseries} %encabezamientos sin negrita
%\addtokomafont{chapter}{\mdseries} %encabezamientos sin negrita
%\setkomafont{disposition}{\normalcolor\bfseries} %no sans serif
%\setkomafont{disposition}{\normalcolor\mdseries} %no negrita

\parskip=0pt\clubpenalty=10000\widowpenalty=10000
\parindent=6mm
\newcommand{\preLilyPondExample}{\vspace{0pt}}
\newcommand{\postLilyPondExample}{\vspace{6pt}}

\newcommand{\lpversion}{2.13.4}
\newcommand{\defsep}{\textbf{$\|$}}
\newcommand{\software}{\emph{software}}
\newcommand{\negspace}{\vspace{-10pt}}  %{\vspace{-20pt}}
\newcommand{\seppar}{
\bigskip
%\vspace{6pt}
}



%%%%%%%%%%%%%%%%%%%%%%%%%%%%%%%%%%%%%%%%%% TÍTULOS
\title{LilyPond\\30 Ejercicios semanales}
%\subtitle{}
% necesita Koma-script 2.98 2007/12/24
\author{Francisco Vila}
\date{\today}
%\date{}
%%%%%%%%%%%%%%%%%%%%%%%%%%%%%%%%%%%%%%%%%% FIN TÍTULOS

% \includeonly{apuntes-29-espanol}

\begin{document}

\nonfrenchspacing

\begin{titlepage} %%%%%%%%%%%% PORTADA
  \makeatletter
  \begin{center}
    %\textbf{\Large Universidad de Extremadura}\par
    \vfill
    \includegraphics[width=60mm]{lily-logo.png}\par
    \vfill
    \textbf{\huge\@title}\par
%    \textbf{\large\@subtitle}\par
    {\@date}
    \vfill
    \textbf{\large\@author}
    \vfill
  \end{center}
  \makeatother
\end{titlepage}

%\singlespace

\begin{singlespace} % los listados van a un espacio. Necesita el paquete setspace
  \tableofcontents
 % \listoffigures
 % \listoftables
\end{singlespace}
 
 \section*{Introducción}

Esto no es un curso completo de LilyPond ni creo que sustituya a unas
clases directas; tan sólo pretende servir como material de apoyo. La
forma de utilizar estos ejercicios es bastante obvia si se examina uno
cualquiera de ellos:

\begin{enumerate}
\item Observar el modelo y buscar los elementos desconocidos.

\item Leer cuidadosamente el texto para aprender a realizar estos
  elementos.  Habrá que adaptarlos para recrear el modelo.

\item Tratar de tipografiar el modelo exactamente.

\end{enumerate}

El orden de los ejercicios es importante porque los elementos nuevos,
necesarios para el modelo, siempre están explicados en el mismo
apartado.

Dos son los complementos necesarios para aprender a tipografiar música
con LilyPond.  El primero es la documentación oficial que está en
lilypond.org; el segundo es la comunidad de usuarios, que está a su
disposición en la dirección de la lista en
español\footnote{http://lists.gnu.org/mailman/listinfo/lilypond-es} y
en la lista general en
inglés\footnote{http://lists.gnu.org/mailman/listinfo/lilypond-user}.

El estado actual de este documento es ``borrador incompleto''.  Dado
que los ejercicios son semanales, su número al término de la colección
será de 30, justo el número de semanas de un curso académico.  El
estilo de las indicaciones es muy escueto porque en su origen estas
``lecciones'' estaban pensadas para entregar a los alumnos en mano, en
una hoja o dos como máximo.  Dado que ésta ya no es la situación
actual, la vía está abierta a un estilo más ``de libro'' que suavice
la severidad que el escrito presenta por el momento.

\section*{LilyPond ¿por qué?}

Todo el que aún no ha quedado atrapado en el irresistible atractivo de
LilyPond como idea y como solución al problema de la
tipografía\footnote{El llamado ``problema de la tipografía musical''
  se expresa como el dilema a que se enfrenta un usuario cuando tiene
  que elegir entre usar copias no autorizadas de software privativo o
  pagar el alto precio (en dinero y en libertades) que cuesta una
  licencia. Como probablemente ya sepa, LilyPond ofrece la tercera
  vía: la del software libre.}, se preguntará, y con razón, ¿por qué
embarcarse en la incómoda y tediosa tarea de aprender un
\emph{lenguaje de programación} estricto, poco intuitivo, no visual,
lleno de reglas imposibles de recordar, pudiendo simplemente
\emph{tomar prestada} una copia de \emph{ese otro programa que todo el
  mundo usa}? La respuesta es simple: todos estos inconvenientes son
relativos y el proceso en su conjunto merece la pena.  No aburriré al
lector con la lista de ventajas que el software libre ofrece frente a
los productos comerciales privativos y cerrados; Internet ofrece
amplísima información sobre este punto.  Es cierto que la adopción de
LilyPond supone un cambio de paradigma, bastante radical, respecto a
la típica aplicación que nos permite editar partituras colocando notas
con el ratón sobre un pentagrama vacío. Sin embargo, al mismo tiempo
que reconocemos que llegar a dominar este gran programa requiere un
cierto esfuerzo, debemos resaltar la gran potencia de sus conceptos
principales:

\begin{itemize}
\item documentos de texto que podemos leer sin el programa
\item la robustez de no depender de hacer \emph{clic} en el lugar equivocado
\item la posibilidad de reutilizar el material de forma consistente
\item gratuidad y disponibilidad del programa y su documentación
\item calidad final muy elevada, sin compromiso
\item adaptación automática del material a la página
\item solución automática de las colisiones entre elementos
\item escalabilidad: idoneidad para proyectos de todos los tamaños
\end{itemize}

%Antes de publicar este borrador en una forma definitiva, posiblemente
%haya que emplear fragmentos alternativos de ciertas obras protegidas.

LilyPond está hecho por un grupo de desarrolladores en su tiempo
libre. Es un ejemplo de desarrollo colaborativo. Pertenecer a este
equipo me llena de orgullo.  Mi deseo es que esta pequeña recopilación
sea de utilidad a alguien en algún lugar.  Gracias por leerla.


\section*{Licencia}

Esta recopilación de ejercicios semanales ha nacido de la necesidad de
tener un material de trabajo para mis clases de Edición de partituras
en el Conservatorio Superior de Badajoz.  Esta asignatura optativa ya
no existe, pero en cambio la comunidad hispanohablante de usuarios de
LilyPond ha ido creciendo poco a poco y quizá el uso de estos apuntes
encuentre un hueco en un ámbito menos local.

Este documento está publicado bajo la licencia (cc)(by)(sa) Creative
Commons Atribución - Compartir igual 3.0 España.  Algunos derechos
reservados.

Usted es libre de copiar, distribuir y comunicar públicamente la obra.
Puede hacer obras derivadas y distribuirlas, siempre que añada el
siguiente texto:

\begin{quote}
\emph{Basado en un trabajo anterior de Francisco Vila,
  \texttt{http://www.paconet.org} }
\end{quote}

visible en una de las dos primeras páginas, o una de las dos últimas.

La obra derivada debe distribuirse bajo la misma licencia.

El texto completo de esta licencia está en
http://creativecommons.org/licenses/by-sa/2.0/es/legalcode.es




 % \version "2.17.0"

\section{Uso de LilyPond y Frescobaldi bajo Windows}
\subsection{Descarga e instalación}

Recomendamos utilizar la combinación LilyPond/Frescobaldi para el
trabajo con partituras de LilyPond. Ambos están disponibles para su
descarga gratuita en las respectivas páginas oficiales de los
proyectos, \texttt{lilypond.org} y \texttt{frescobaldi.org}.  Después
de ejecutar los programas instaladores, las aplicaciones están listas
para su uso.

%\subsection{Configuración de Frescobaldi}
%Después de instalar Frescobaldi y LilyPond, es necesario que el
%primero pueda llamar al segundo para procesar las partituras, para lo
%que debe estar configurado con la ruta

% Opcional, no es seguro, comprobar

\subsection{Creación de una partitura sencilla}

Ahora comprobaremos si todo nuestro software funciona, con un ejemplo
absolutamente mínimo. Abrimos el programa Frescobaldi y escribimos lo
siguiente\footnote{Las llaves se consiguen con AltGr pulsando una
  tecla que en los teclados españoles de PC suele estar junto a la
  'Ñ'. Los apóstrofos se consiguen mediante la tecla que está justo a
  la derecha del número 0.}:

\begin{quote}
\begin{verbatim}
 { b }
\end{verbatim}
\end{quote}

Denominamos a este texto \emph{código de entrada}.

\subsection{Procesar el documento}

Los programas Frescobaldi (un editor) y LilyPond (un generador de
partituras a partir de un texto) forman una combinación en la que cada
uno está especializado en una misión.  Si pulsamos en Frescobaldi la
combinación de teclas Control+M, éste llama a LilyPond pasándole el
documento en curso para que lo convierta en una partitura en formato
PDF. Esto se denomina \emph{procesar el código de entrada}, y al PDF
resultante le llamamos \emph{la salida}.

El procesado tarda un par de segundos\footnote{La primera vez después
  de haber instalado LilyPond, el programa tiene que preparar las
  fuentes tipográficas; esto lleva aproximadamente medio minuto, pero
  las ejecuciones posteriores tardan, como se ha dicho, unos
  segundos.}. El resultado es un archivo PDF que puede verse en el
panel derecho de Vista Previa de la música:

\begin[staffsize=15,fragment,quote]{lilypond}
  b
\end{lilypond}

Podemos guardar este texto con un nombre terminado en la extensión
\verb+.ly+, por ejemplo \verb+prueba.ly+.  Pulse la combinación de
teclas Control+S para hacerlo.  Denominaremos a este archivo con la
extensión \verb+.ly+ que contiene el código de entrada, \emph{archivo
  fuente} o \emph{archivo de entrada}.  Si pulsamos de nuevo
Control+M, el PDF resultante se llamará \verb+prueba.pdf+ y estará
localizado en la misma carpeta en que hemos guardado el archivo
fuente.

Antes de haber guardado el documento, tanto el texto de entrada como
el PDF de salida se encuentran en una carpeta temporal, de donde se
borrarán al salir del programa Frescobaldi.

\subsection{Expresiones musicales de LilyPond}

Una partitura completa de LilyPond es un fragmento de música encerrado
entre llaves \verb+{ }+, llamado \emph{expresión musical}.  En los
ejemplos que aparecen en este cuaderno de ejercicios, con frecuencia
se omiten las llaves por sencillez, pero no debe olvidarlas cuando
inserte el código en sus propios ejercicios.

De igual forma que las expresiones matemáticas, en LilyPond podemos
ampliar una expresión añadiéndole operadores a la izquierda o
combinando varias de ellas en una sola; el ejemplo de este primer
ejercicio, sin embargo, es una expresión sencilla.  Veremos casos más
complejos cuando la música lo requiera.

\subsection{Conclusión}

Observamos que nuestro código de entrada minimalista se limita a
declarar la nota \emph{si} por su nombre anglosajón, ``b'', y que el
resultado incluye esta nota (con un valor de negra) y además un
pentagrama, una clave de \emph{sol} y un compás de 4/4.  Son los
valores por omisión, que se dan por supuestos si no los damos
explícitamente.

Le damos la enhorabuena si ha conseguido completar la primera lección
con éxito. En los apartados siguientes vamos a profundizar en la
producción de partituras progresivamente más complejas.

 % \version "2.17.0"

\section{\emph{Cumpleaños feliz}}
\subsection{Modelo}

Aprenderemos a tipografiar este ejemplo mediante las indicaciones que
se dan en los apartados siguientes.  

\bigskip

\begin[relative=2,staffsize=13,fragment]{lilypond}
  \time 3/4 \partial 4
g8. g16 a4 g c b2
g8. g16 a4 g d' c2
g8. g16 g'4 e c b a\fermata
f'8. f16 e4 c d c2. \bar "|."
\end{lilypond}

Recordemos que el objetivo de cada uno de los 30 ejercicios es
reproducir el modelo, adaptando por nosotros mismos las indicaciones
de los epígrafes que le siguen.

Ahora se trata en primer lugar de introducir las notas: podemos
hacerlo en modo absoluto o en modo relativo.

\subsection{Modo absoluto}
Hemos visto en la lección primera que un conjunto de nombres de notas,
en notación inglesa\footnote{Holandesa, en realidad; la diferencia
  está en las alteraciones.}, encerrados en un par de llaves, dan
directamente una partitura. Esta forma de nombrar las notas se llama
\emph{modo absoluto} y requiere un apóstrofo para las notas desde el
Do central hasta el Si de la tercera línea, en clave de Sol, así:
\verb+c'+ \verb+d'+ \verb+e'+ \verb+f'+ \verb+g'+ \verb+a'+
\verb+b'+. Añadiremos otro apóstrofo para nombrar las notas de la
octava superior: \verb+c''+ \verb+d''+ \verb+e''+ etc. Las notas sin
apóstrofo \verb+c+ \verb+d+ \verb+e+ \verb+f+ \verb+g+ \verb+a+
\verb+b+ dan la octava inferior al Do central. Al añadir una coma
obtenemos la octava inferior \verb+c,+ \verb+d,+ \verb+e,+ \verb+f,+
\verb+g,+ \verb+a,+ \verb+b,+ y otra octava por debajo si añadimos
otra coma: \verb+c,,+ \verb+d,,+ \verb+e,,+ etc.

Este modo es adecuado para la introducción de música que tenga muchos
saltos. Cuando hay voces melódicas con pocos saltos, es más cómodo el
modo relativo, que nos puede ahorrar muchas comas y apóstrofos.

\subsection{Modo relativo}
Al introducir las notas, si precedemos la expresión entre llaves por
la instrucción \verb+\relative+ seguida de una nota, no tenemos que
especificar la altura de cada nota para saltos de cuarta o menores; a
partir de un salto de quinta hay que añadir un apóstrofo para subir
una octava, y una coma para bajar una octava.

\begin[verbatim,staffsize=13]{lilypond}
\relative c' { c e g c g' a f e d c g c, }
\end{lilypond}


\subsection{Compás}

El compás es de 3/4; escribimos
\begin[verbatim,relative=2,staffsize=13,fragment]{lilypond}
\time 3/4
\end{lilypond}


\subsection{Anacrusa}

El compás inicial está incompleto y sólo tiene un valor de negra; lo
expresamos mediante \verb+\partial+ seguido de una duración, en
nuestro caso el 4 que indica un valor de negra.

\begin[verbatim,relative=2,staffsize=13,fragment]{lilypond}
\time 3/4 \partial 4 g
\end{lilypond}

\subsection{Duraciones. Puntillo}

Los valores de nuestro ejemplo son: blanca, negra, corchea y
semicorchea.  Se escriben como las cifras 2, 4, 8 y 16,
respectivamente, detrás de la nota.

\begin[verbatim,relative=2,staffsize=13,fragment]{lilypond}
g2 g4 g8 g16
\end{lilypond}

El puntillo se consigue mediante el punto ortográfico después del
número de la duración.

\begin[verbatim,notime,relative=2,staffsize=13,fragment]{lilypond}
g2. g4. g8.
\end{lilypond}

\subsection{Calderón y doble barra final}

Colocamos un calderón sobre una nota mediante la instrucción
\verb+\fermata+ y la doble barra mediante la instrucción \verb+\bar+
seguida del tipo de barra deseado, que en nuestro caso es \verb+"|."+,
entre comillas.

\begin[verbatim,notime,relative=2,staffsize=13,fragment]{lilypond}
g2\fermata \bar "|."
\end{lilypond}



 %\documentclass[12pt,a4paper,oneside]{scrbook} % la clase book del Koma-script bundle
\documentclass[a4paper,10pt,oneside,headinclude,titlepage]{article} % la clase book del Koma-script bundle
%\linespread{1.25}
\usepackage{setspace}
%\usepackage{tikz}
%\usetikzlibrary{fit,shapes}
\usepackage[spanish]{babel}
%\usepackage{verbatim} %para el entorno comment
%\usepackage{moreverb} %para los ejemplos de lilypond, aporta verbatimtabinput
%\usepackage{alltt} %para los ejemplos de lilypond, aporta verbatiminput
%\usepackage{sverb} %para los ejemplos de lilypond, aporta verbinput
%\usepackage{fancyvrb} %para los ejemplos de lilypond, aporta VerbatimInput
\pagestyle{empty}
\usepackage[utf8]{inputenc}
\usepackage[T1]{fontenc} %posiblemente sirva para eliminar el problema del enguionado de palabras acentuadas. Lo quitamos provisionalmente para evitar un error
\usepackage{textcomp} % recomendación de Javier Bezos para completar la fuente

\usepackage[margin=2cm]{geometry}
\usepackage{graphicx}
%\usepackage{url}

\usepackage[utopia]{mathdesign}
%\usepackage{mathptmx} %mejor que Times    % alternativa a Charter


%\typearea[0mm]{13}% same as class options above
%\usepackage{newcent}
%\addtokomafont{part}{\mdseries} %encabezamientos sin negrita
%\addtokomafont{partnumber}{\mdseries} %encabezamientos sin negrita
%\addtokomafont{chapter}{\mdseries} %encabezamientos sin negrita
%\setkomafont{disposition}{\normalcolor\bfseries} %no sans serif
%\setkomafont{disposition}{\normalcolor\mdseries} %no negrita

\parskip=6pt\clubpenalty=10000\widowpenalty=10000

\newcommand{\preLilyPondExample}{\vspace{-10pt}}

\newcommand{\lpversion}{2.13.4}
\newcommand{\defsep}{\textbf{$\|$}}
\newcommand{\software}{\emph{software}}
\newcommand{\negspace}{\vspace{-10pt}}  %{\vspace{-20pt}}
\newcommand{\seppar}{
\bigskip
%\vspace{6pt}
}

%%%%%%%%%%%%%%%%%%%%%%%%%%%%%%%%%%%%%%%%%%%%%%%%%%%%%%%%%%%%%%%%%%%%%%%%%%%%%%%%%%%%%%%%%%%
\begin{document}

\setcounter{section}{2}
\section{Serenata nocturna, de Mozart}
\subsection{Modelo}

Esta vez utilizaremos como modelo el conocido comienzo de la
``serenata nocturna'' de Mozart.

\bigskip

\begin[relative=2,staffsize=18,fragment]{lilypond}
\key g \major \tempo "Allegro"
<g d' g>4 \f r8 d' g4 r8 d
g8 d g b d4 r
c4 r8 a c4 r8 a
c8 a fis a d,4 r4 \bar "||"
\end{lilypond}

Este fragmento contiene los elementos nuevos de notación musical
que iremos revisando en los apartados siguientes.

\subsection{Tonalidad}
Podemos definir la armadura de la tonalidad mediante la
instrucción \verb+\key+ seguida del nombre de una nota y de la
instrucción \verb+\major+ (para mayor) o \verb+\minor+ (para
menor):

\begin[verbatim,relative=2,staffsize=13]{lilypond}
\key f \minor
f g a
\end{lilypond}

Observemos que '\verb+a+' produce un La natural aunque la armadura
es de Fa menor.

\subsection{Indicación de tempo}

Mediante \verb+\tempo+ seguido de una expresión entre comillas,
colocamos en el lugar adecuado una indicación de tempo.

\begin[verbatim,relative=1,staffsize=13,fragment]{lilypond}
\tempo "Allegro con fuoco" c1 c c
\end{lilypond}

\subsection{Acordes}

Los acordes se introducen escribiendo las notas entre ángulos, en
cualquier orden.  La duración se coloca después del ángulo de
cierre.
\begin[verbatim,relative=1,staffsize=13,fragment]{lilypond}
< c e g >2
\end{lilypond}

%\begin{minipage}{\textwidth}
El modo relativo funciona dentro de un acorde, pero es la primera
nota del acorde la que se tiene en cuenta para las notas que
siguen.  El último Do del siguiente ejemplo no es relativo a la
tercera nota del acorde, sino a la primera.

\begin[verbatim,relative=1,staffsize=13,fragment]{lilypond}
< c e g > c
\end{lilypond}
%\end{minipage}

\newpage
\subsection{Notas alteradas}

El sostenido se obtiene añadiendo ``\verb+is+'' al nombre de la nota, y el bemol añadiendo ``\verb+es+'':

\begin[verbatim,relative=2,staffsize=13,fragment]{lilypond}
c cis a aes
\end{lilypond}


En LilyPond se introduce siempre la altura real de las notas,
naturales o alteradas, aunque no presenten una alteración
accidental.  Por ejemplo, en la tonalidad de Fa mayor es necesario
escribir ``\verb+bes+'' para obtener el Si bemol, aunque la
armadura ya contiene esta alteración.

\begin[verbatim,relative=2,staffsize=13,fragment]{lilypond}
\key f \major bes1
\end{lilypond}

\subsection{Matices}

Para imprimir una indicación de dinámica podemos escribir
\verb+\p+, \verb+\mf+, \verb+\f+, etc. después de una nota.

\begin[verbatim,relative=2,staffsize=13,fragment]{lilypond}
c2 \pp c \mf  c \f  c \ff
\end{lilypond}

\subsection{Silencios}

Los silencios se escriben como si fueran notas con el nombre
'\verb+r+':

\begin[verbatim,relative=2,staffsize=13,fragment]{lilypond}
r2 r4 r8 r16
\end{lilypond}




\subsection{Barra doble}

La barra introducida mediante \verb+\bar "||"+ produce una doble
barra simple, distinta a la doble barra final que se obtiene
mediante \verb+\bar "|."+

\begin[verbatim,relative=1,staffsize=13,fragment]{lilypond}
c1 \bar "||"
\end{lilypond}

En el teclado español, el signo de barra '|' está en AltGr + 1.


\end{document}


 % \version "2.17.0"

\section{La \emph{Ofrenda Musical}, de Bach}


\subsection{Modelo}

Estudiaremos los títulos de cabecera y ejercitaremos las alteraciones
accidentales con este ejemplo de Bach:

\bigskip

\begin[staffsize=17.5,line-width=17\cm]{lilypond}
\header { title="Tema real"
          subtitle="de la \"Ofrenda musical\""
          composer="J.S. Bach"
}
\relative c'' {
\key c \minor
\time 2/2
c2 es g as b, r4
g' fis2 f e es~ es4 d des c b a8 g c4 f es2 d c4 }
\end{lilypond}

Casi todos los elementos de notación de este fragmento ya se han
estudiado.  Veamos, en los apartados siguientes, solamente los que
faltan.

\subsection{Títulos}

El título, subtítulo, autor y otros muchos encabezamientos se
especifican dentro de un bloque \verb+\header { ... }+ en la siguiente
forma:

\begin{verbatim}
\header {
  title    = "Título"
  subtitle = "Subtítulo"
  composer = "Autor"
}
\end{verbatim}

Si el propio encabezamiento contiene comillas, es necesario escribir
\verb+\"+ para imprimir cada una estas comillas.  Por ejemplo:

\begin{verbatim}
\header {
  title="Sonata \"Claro de luna\""
}
\end{verbatim}



\subsection{Compás}
Definimos el tipo de compás mediante la instrucción \verb+\time+
seguida de un quebrado:

\begin[verbatim,relative=2,staffsize=13]{lilypond}
\time 3/4
c4 c c
\time 6/8
c4. c
\time 2/4
c2
\time 2/2
c1
\end{lilypond}


\subsection{Ligadura de unión}

Utilizamos la tilde curva (en la tecla Alt Gr + 4) para unir dos notas
de idéntica altura:

\begin[verbatim,relative=1,staffsize=13,fragment]{lilypond}
c ~ c
\end{lilypond}



 \section{``La llamada del destino'' (Quinta sinfonía de Beethoven)}


\subsection{Modelo}

En este modelo que reproduce el tema del primer movimiento de la
5ª sinfonía de Beethoven, vemos un caso de barrado manual:

\bigskip

\begin[staffsize=17.5,no-ragged-right]{lilypond}
\version "2.11.63"

\relative c''{
 \key c \minor
 \time 2/4
 r8 g8[ g g]
 ees2 \fermata
 r8 f8[  f f]
 d2 ~
 d \fermata
}
\end{lilypond}


\subsection{Número de la versión}

Los archivos de entrada de LilyPond siguen una sintaxis estricta.
Los desarrolladores del programa LilyPond tratan de mantener lo
más estable posible esta sintaxis, pero de vez en cuando se
producen cambios que hacen incompatibles los archivos de entrada
antiguos con las versiones de LilyPond recientes.  Existe un
programa convertidor que no usaremos aún, pero que requiere que
especifiquemos el número de la versión del programa para la que se
escribió la partitura; de esa forma, será posible convertir
automáticamente los archivos para actualizarlos.  El número de la
versión debe escribirse siempre al principio del texto, en la
forma \verb+\version "2.12.0"+, donde aparece entrecomillado el
número de la versión actual del programa.

Si no especificamos ningún número de versión, el programa
registrará una advertencia en el archivo de salida \verb+.log+.


\begin{verbatim}GNU LilyPond 2.14.0
Procesando «05-barras-beethoven-5thsym.ly»
Analizando...
05-barras-beethoven-5thsym.ly:0:
warning: no se ha encontrado ninguna instrucción \version, escriba

\version "2.14.0"

para disponer de compatibilidad en el futuro
\end{verbatim}

\subsection{Barrado manual}
Las barras de corchea, semicorchea y figuras de menor duración se
imprimen automáticamente; sin embargo, en ciertos casos debemos
especificarlas manualmente, por ejemplo en el siguiente caso:

\begin[verbatim,relative=2,staffsize=13]{lilypond}
\time 2/4
r8 g a b
c r r4
\end{lilypond}


Si queremos que las tres primeras corcheas estén unidas mediante
una barra, marcamos la primera con un corchete recto de apertura
'\verb+[+' y la última con un corchete recto de cierre '\verb+]+',
de la siguiente forma:

\begin[verbatim,relative=2,staffsize=13]{lilypond}
\time 2/4
r8 g[ a b]
c r r4
\end{lilypond}

Es importante observar que los corchetes \textbf{no encierran
  conjuntos de notas}, sino que marcan las notas primera y última
de una barra colocándose cada uno \textbf{detrás} de la nota
correspondiente.


 \section{``Suite para cello número 1'', de Bach}


\subsection{Modelo}

En este fragmento se utilizan ligaduras de expresión:

\bigskip

\begin[staffsize=17.5,no-ragged-right]{lilypond}
\version "2.11.63"

\relative g, {

	\clef bass
	\time 4/4
	\key g \major
%	\set Staff.midiInstrument = "cello"

	% 1
	g16(d') b' a b( d, b' d,) g,(d') b' a b( d, b' d,) |
	g,(e') c' b c( e, c' e,) g,(e') c' b c( e, c' e,) |
%	g,(e') c' b c( e, c') e, g,(e') c' b c( e, c' e,) |

}
\end{lilypond}


\subsection{Ligaduras de expresión}

De la misma forma que en el caso de las barras manuales (que se
indican mediante corchetes de manera que no encierran conjuntos de
notas, sino que los corchetes de apertura y cierre marcan las
notas primera y última que pertenecen a la barra), las ligaduras
de expresión se indican mediante paréntesis de apertura y cierre
que marcan por la derecha las notas primera y última de una ligadura de expresión.

\begin[verbatim,relative=1,staffsize=13]{lilypond}
c( d e f g a b c)
\end{lilypond}


 % \version "2.17.0"

\section{Repeticiones. \emph{Novena sinfonía} de Beethoven}


\subsection{Modelo}



He aquí un ejemplo de repetición de primera y segunda vez:

\bigskip

\begin[staffsize=17.5,line-width=17\cm]{lilypond}


\header { title = "Novena Sinfonía" composer = "Beethoven" }

\relative c' { \key c \major

\repeat volta 2 { e2 f4 g g f e d c c d e } \alternative { { e2 d2 } { d2 c2 } }


}


\end{lilypond}


\subsection{Repeticiones sencillas}

En LilyPond, las repeticiones no se hacen definiendo tipos de barra o
dibujando explícitamente puntos de repetición.  En lugar de eso,
definimos el fragmento que se repite y cuáles son los finales
alternativos, como bloques separados dentro de la instrucción
\verb+\repeat+.  Hay varios tipos de repetición; para la primera y
segunda vez, empleamos esta forma:

\verb+\repeat volta veces {trozo que se repite} \alternative{{primera vez}{segunda vez}}+

En este ejemplo, dejamos \verb+volta+ como está, para expresar el tipo
de repetición; sustituimos \verb+veces+ por el número de repeticiones,
y en los bloques de primera y segunda vez escribimos la música que va
dentro de las casillas de primera y segunda.  Los bloques de los
finales alternativos van, a su vez, dentro del bloque
\verb+\alternative{}+ entre llaves.

Si hay más de dos repeticiones, la segunda alternativa se marca para
ejecutarse la última vez.

\begin[verbatim,relative=2,staffsize=13]{lilypond}
\repeat volta 3 { g4 f e d } \alternative{{ g1 } { c,1 }}
\end{lilypond}

El modo relativo sigue funcionando dentro del texto de entrada de
forma normal como si toda la música fuese secuencial, sin
repeticiones.


 % \version "2.17.0"

\section{Contextos explícitos. Música simultánea}


\subsection{Modelo}

El presente ejemplo contiene música a dos voces en dos pentagramas:

\bigskip

\begin[staffsize=17.5,line-width=17\cm]{lilypond}

<<

\new Staff \relative c''' {
\time 12/8 \key f\major
    c4. ~ c8 b a g4. ~ g8 f e |
    d b c f4. ~ f8 e g c4 es,8

}

\new Staff \relative c'' {
\time 12/8 \key f\major
    e16( d )e c g c f( e )f d b d g( f )g e c e a( g )a f g a |
    b,8 g' c, ~ c a b!-\trill c16( b )c g e g f( g a bes )c a }

>>


\end{lilypond}


\subsection{Contextos explícitos}

La construcción

\verb+{ música }+

es una abreviatura de

\verb+\new Staff { \new Voice { música } }+

y es suficiente para la mayoría de las aplicaciones sencillas. Staff
(pentagrama) y Voice (Voz) son contextos; los contextos contienen
música.  Muchas veces un contexto se crea de forma implícita allí
donde se necesita.  Sin embargo, es conveniente declarar de forma
explícita al menos el contexto de pentagrama (la parte
\verb+\new Staff+) para tener un mayor control sobre los pentagramas
que se crean.

\subsection{Música simultánea}

Dos o más expresiones encerradas entre ángulos dobles, \verb+<< >>+,
se imprimen como música simultánea.  La tonalidad no se hereda de una
expresión a otra, pero la indicación de compás es común:

\begin[verbatim,relative=2,staffsize=13]{lilypond}
<<
 \relative c' {
   \key f \major
   \time 2/4
   c d e g }
 \relative c' {
   e d c b }
>>
\end{lilypond}

Observemos que los dos pentagramas están en compás de 2/4 pero sólo el
de arriba está en Fa mayor.

\subsection{Trinos}

La instrucción \verb+\trill+ después de una nota, unida mediante un
guión, produce una indicación de trino:

\begin[verbatim,relative=1,staffsize=13]{lilypond}
{ c-\trill }
\end{lilypond}



 %\documentclass[12pt,a4paper,oneside]{scrbook} % la clase book del Koma-script bundle
\documentclass[a4paper,10pt,oneside,headinclude,titlepage]{article} % la clase book del Koma-script bundle
%\linespread{1.25}
\usepackage{setspace}
%\usepackage{tikz}
%\usetikzlibrary{fit,shapes}
\usepackage[spanish]{babel}
%\usepackage{verbatim} %para el entorno comment
%\usepackage{moreverb} %para los ejemplos de lilypond, aporta verbatimtabinput
%\usepackage{alltt} %para los ejemplos de lilypond, aporta verbatiminput
%\usepackage{sverb} %para los ejemplos de lilypond, aporta verbinput
%\usepackage{fancyvrb} %para los ejemplos de lilypond, aporta VerbatimInput
\pagestyle{empty}
\usepackage[utf8]{inputenc}
\usepackage[T1]{fontenc} %posiblemente sirva para eliminar el problema del enguionado de palabras acentuadas. Lo quitamos provisionalmente para evitar un error
\usepackage{textcomp} % recomendación de Javier Bezos para completar la fuente

\usepackage[margin=2cm]{geometry}
\usepackage{graphicx}
%\usepackage{url}

\usepackage[utopia]{mathdesign}
%\usepackage{mathptmx} %mejor que Times    % alternativa a Charter


%\typearea[0mm]{13}% same as class options above
%\usepackage{newcent}
%\addtokomafont{part}{\mdseries} %encabezamientos sin negrita
%\addtokomafont{partnumber}{\mdseries} %encabezamientos sin negrita
%\addtokomafont{chapter}{\mdseries} %encabezamientos sin negrita
%\setkomafont{disposition}{\normalcolor\bfseries} %no sans serif
%\setkomafont{disposition}{\normalcolor\mdseries} %no negrita

\parskip=6pt\clubpenalty=10000\widowpenalty=10000

\newcommand{\preLilyPondExample}{\vspace{-10pt}}

\newcommand{\lpversion}{2.13.4}
\newcommand{\defsep}{\textbf{$\|$}}
\newcommand{\software}{\emph{software}}
\newcommand{\negspace}{\vspace{-10pt}}  %{\vspace{-20pt}}
\newcommand{\seppar}{
\bigskip
%\vspace{6pt}
}

%%%%%%%%%%%%%%%%%%%%%%%%%%%%%%%%%%%%%%%%%%%%%%%%%%%%%%%%%%%%%%%%%%%%%%%%%%%%%%%%%%%%%%%%%%%
\begin{document}

\setcounter{section}{8} %para 09 Polifonía


\section{Polifonía en un pentagrama}


\subsection{Modelo}

En la jerga de LilyPond, ``polifonía'' significa más de una voz en el mismo pentagrama.

El siguiente ejemplo puede obtenerse a partir del ejercicio anterior
sin alterar la música:

\bigskip

\begin[staffsize=17.5]{lilypond}
\score{
\new Staff <<

 \relative c''' {
\time 12/8 \key f\major
    c4. ~ c8 b a g4. ~ g8 f e |
    d b c f4. ~ f8 e g c4 es,8 

}
\\
 \relative c'' {
\time 12/8 \key f\major
    e16( d )e c g c f( e )f d b d g( f )g e c e a( g )a f g a |
    b,8 g' c, ~ c a b!-\trill c16( b )c g e g f( g a bes )c a }

>>
\layout{ system-count=1 }
}

\end{lilypond}


\subsection{La construcción de voces polifónicas}

Supongamos que tenemos música simultánea en dos pentagramas:

\begin[verbatim,relative=2,staffsize=17.5]{lilypond}
<<
  \new Staff { e4 f g2 e4 f g2 g8 a g f e4 c4 g'8 a g f e4 c4 }
  \new Staff { c,4 d e c c d e c e f g2 e4 f g2 }
>> 
\end{lilypond}

\bigskip

La construcción

\begin{quote}
  \verb+<< { música } \\ { música } >>+
\end{quote}

permite crear dos voces dentro de un pentagrama; partiendo del ejemplo
anterior es fácil hacer lo siguiente:

\begin[verbatim,relative=2,staffsize=17.5]{lilypond}
\new Staff
  <<
    { e4 f g2 e4 f g2 g8 a g f e4 c4 g'8 a g f e4 c4 }
      \\
    { c,4 d e c c d e c e f g2 e4 f g2 }
  >> 
\end{lilypond}


La primera expresión es la voz 1 y tiene las plicas hacia arriba; la
segunda expresión es la voz 2 y tiene las plicas hacia abajo.

\end{document}


 
\setcounter{section}{9} %para 10 tresillos


\section{Sistemas de piano. Tresillos}


\subsection{Modelo}

Este fragmento de música para piano tiene una llave que une los dos
pentagramas.  En él hay tresillos y dos voces en el pentagrama
inferior.

\bigskip

\begin[staffsize=17.5]{lilypond}
\new PianoStaff <<
\new Staff \relative c' { \time 2/4
	\times 2/3 { c8 e g } d4
	e8 c d4
	\times 2/3 { c8 e g } \times 2/3 { f e d }
	c4 d
}
\new Staff \relative c { \clef bass
	<< {
	c4 fis
	g4 fis
	e4 fis
	e d }
	\\
	{ c2 ~ c ~ c ~ c } >>
}
>>
\end{lilypond}


\subsection{Tresillos y otros grupos de valoración especial}

He aquí cómo se pueden expresar los tresillos del Bolero de M. Ravel:


\begin[verbatim,relative=3,staffsize=17.5]{lilypond}
\time 3/4 g8[ \times 2/3 { g16 g g] } g8[ \times 2/3 { g16 g g] } g8 g
\end{lilypond}


Para componer tipográficamente un grupo de valoración especial se usa
la instrucción \verb+\times+ \emph{fracción} \verb+{ ... }+, que
multiplica la expresión entre llaves por la fracción expresada.

Por ejemplo, el siguiente grupo vale como 6 corcheas:

\begin[verbatim,relative=2,staffsize=17.5]{lilypond}
\time 3/4 \times 6/7 { ees8( f ees d ees ges8. f16) }
\end{lilypond}


\subsection{Sistemas de piano}

Declarando el contexto explícito \verb+PianoStaff+ podemos dibujar un
sistema de piano e introducir dentro de él los pentragramas superior e
inferior:

\begin[verbatim,staffsize=17.5]{lilypond}
\new PianoStaff <<
  \new Staff \relative c' { c4 c c c }
  \new Staff \relative c  { \clef bass c4 c c c }
>>
\end{lilypond}


 \section{\emph{Esta noche es Nochebuena} (I). Canciones con letra.}


\subsection{Modelo}

A continuación presentamos un villancico del s. XVI, original de
Gales, con título en inglés \emph{Deck the Halls}:

\bigskip

\begin[staffsize=17.5,line-width=17\cm]{lilypond}
\relative c'' { \key f \major
c8. bes16 a8 g
f g a8 f
g16 a bes g a8. g16
f8 e f4
c'8. bes16 a8 g
f g a8( f)
d'16 d d d c8. bes16
a8 g f4
}
\addlyrics { Es -- ta no che~es No -- che -- bue -- na,
	fa la la la la, la la la la.
	Y no~es no  -- che de dor -- mir
	fa la la la la, la la la la. }
\end{lilypond}


\subsection{Contextos de letra}

El contexto de letra se llama Lyrics, y su contenido debe ir precedido
de \verb+\lyricmode+ para que se interprete como letra. Las sílabas se
separan mediante dos guiones.  Una forma de alinear la letra con la
música es expresar la duración de cada sílaba como si fueran notas:

\begin[verbatim,relative=3,staffsize=17.5]{lilypond}
<<
  \new Staff \relative c'' { \time 3/4 \partial 4 g8. g16 a4 g c b }
  \new Lyrics \lyricmode { Cum8. -- ple16 -- a4 -- ños fe -- liz }
>>
\end{lilypond}

Otra manera, más sencilla, es utilizar \verb+\addlyrics+ después de la
música, como aparece en el siguiente ejemplo.  Las sinalefas se
obtienen uniendo las sílabas mediante una tilde curva, el mismo
símbolo que se utiliza para la ligadura de unión.

\begin[verbatim,relative=3,staffsize=17.5]{lilypond}
\relative c' { \partial 4 e8 f g4 c b8 b r4 }
\addlyrics { ¿Dón -- de~es -- tán las lla -- ves? }
\end{lilypond}

La construcción

\begin{verbatim}
{ música }\addlyrics{ letra }
\end{verbatim}

es una abreviatura de otra construcción más compleja en la que se
utiliza un contexto de voz explícito y con un nombre, y un contexto de
letra que se asigna a éste gracias a la instrucción
\verb+\lyricsto+. Equivale a lo siguiente:

\begin{verbatim}
<<
  \new Voice = "nombre" { música }
  \new Lyrics \lyricsto "nombre" { letra }
>>
\end{verbatim}

donde se puede observar que \verb+\lyricsto+ implica a \verb+\lyricmode+.

 %\documentclass[12pt,a4paper,oneside]{scrbook} % la clase book del Koma-script bundle
\documentclass[a4paper,10pt,oneside,headinclude,titlepage]{article} % la clase book del Koma-script bundle
%\linespread{1.25}
\usepackage{setspace}
%\usepackage{tikz}
%\usetikzlibrary{fit,shapes}
\usepackage[spanish]{babel}
%\usepackage{verbatim} %para el entorno comment
%\usepackage{moreverb} %para los ejemplos de lilypond, aporta verbatimtabinput
%\usepackage{alltt} %para los ejemplos de lilypond, aporta verbatiminput
%\usepackage{sverb} %para los ejemplos de lilypond, aporta verbinput
%\usepackage{fancyvrb} %para los ejemplos de lilypond, aporta VerbatimInput
\pagestyle{empty}
\usepackage[utf8]{inputenc}
\usepackage[T1]{fontenc} %posiblemente sirva para eliminar el problema del enguionado de palabras acentuadas. Lo quitamos provisionalmente para evitar un error
\usepackage{textcomp} % recomendación de Javier Bezos para completar la fuente

\usepackage[margin=2cm]{geometry}
\usepackage{graphicx}
%\usepackage{url}

\usepackage[utopia]{mathdesign}
%\usepackage{mathptmx} %mejor que Times    % alternativa a Charter


%\typearea[0mm]{13}% same as class options above
%\usepackage{newcent}
%\addtokomafont{part}{\mdseries} %encabezamientos sin negrita
%\addtokomafont{partnumber}{\mdseries} %encabezamientos sin negrita
%\addtokomafont{chapter}{\mdseries} %encabezamientos sin negrita
%\setkomafont{disposition}{\normalcolor\bfseries} %no sans serif
%\setkomafont{disposition}{\normalcolor\mdseries} %no negrita

\parskip=6pt\clubpenalty=10000\widowpenalty=10000

\newcommand{\preLilyPondExample}{\vspace{-10pt}}

\newcommand{\lpversion}{2.13.4}
\newcommand{\defsep}{\textbf{$\|$}}
\newcommand{\software}{\emph{software}}
\newcommand{\negspace}{\vspace{-10pt}}  %{\vspace{-20pt}}
\newcommand{\seppar}{
\bigskip
%\vspace{6pt}
}

%%%%%%%%%%%%%%%%%%%%%%%%%%%%%%%%%%%%%%%%%%%%%%%%%%%%%%%%%%%%%%%%%%%%%%%%%%%%%%%%%%%%%%%%%%%
\begin{document}

\setcounter{section}{11} %para 12 acordes


\section{Esta noche es Nochebuena (2). Acordes.}


\subsection{Modelo}

En esta ocasión hemos añadido al villancico ``Deck the Halls'' unos acordes en cifrado americano:

\bigskip

\begin[staffsize=17.5]{lilypond}

<<
\new ChordNames \chordmode { f2 c4:7 f c:7 f c:7 f
f2 c4:7 f bes f/c c:7 f }
\relative c'' { \key f \major
c8. bes16 a8 g
f g a8 f
g16 a bes g a8. g16
f8 e f4
c'8. bes16 a8 g
f g a8( f)
d'16 d d d c8. bes16
a8 g f4
}
\addlyrics { Es -- ta no che~es No -- che -- bue -- na,
	fa la la la la, la la la la.
	Y no~es no  -- che de dor -- mir
	fa la la la la, la la la la. }
	
>>
\end{lilypond}


\subsection{Contextos de acordes}

El contexto de nombres de acorde se llama ChordNames, y su contenido
debe ir precedido de \verb+\chordmode+ para que se interprete como
acordes. Se escriben las fundamentales de los acordes con sus
duraciones, y si no hay acorde se escribe ''r'' como silencio, así:

\begin[verbatim,relative=3,staffsize=17.5]{lilypond}
<<
  \new ChordNames \chordmode { r4 c2. g4 }
  \new Staff \relative c'' { \time 3/4 \partial 4 g8. g16 a4 g c b }
  \new Lyrics \lyricmode { Cum8. -- ple16 -- a4 -- ños fe -- liz }
>>
\end{lilypond}

Las variantes como séptima dominante se escriben después de los dos
puntos, y las inversiones se indican escribiendo una barra inclinada y
luego la nota del bajo.

\begin[verbatim,relative=3,staffsize=17.5]{lilypond}
<<
  \new ChordNames \chordmode { r4 c4 c/e g:7 }
  \relative c' { \partial 4 e8 f g4 c b8 b r4 }
  \addlyrics { ¿Dón -- de~es -- tán las lla -- ves? }
>>
\end{lilypond}

\end{document}


 %\documentclass[12pt,a4paper,oneside]{scrbook} % la clase book del Koma-script bundle
\documentclass[a4paper,10pt,oneside,headinclude,titlepage]{article} % la clase book del Koma-script bundle
%\linespread{1.25}
\usepackage{setspace}
%\usepackage{tikz}
%\usetikzlibrary{fit,shapes}
\usepackage[spanish]{babel}
%\usepackage{verbatim} %para el entorno comment
%\usepackage{moreverb} %para los ejemplos de lilypond, aporta verbatimtabinput
%\usepackage{alltt} %para los ejemplos de lilypond, aporta verbatiminput
%\usepackage{sverb} %para los ejemplos de lilypond, aporta verbinput
%\usepackage{fancyvrb} %para los ejemplos de lilypond, aporta VerbatimInput
\pagestyle{empty}
\usepackage[utf8]{inputenc}
\usepackage[T1]{fontenc} %posiblemente sirva para eliminar el problema del enguionado de palabras acentuadas. Lo quitamos provisionalmente para evitar un error
\usepackage{textcomp} % recomendación de Javier Bezos para completar la fuente

\usepackage[margin=2cm]{geometry}
\usepackage{graphicx}
%\usepackage{url}

\usepackage[utopia]{mathdesign}
%\usepackage{mathptmx} %mejor que Times    % alternativa a Charter


%\typearea[0mm]{13}% same as class options above
%\usepackage{newcent}
%\addtokomafont{part}{\mdseries} %encabezamientos sin negrita
%\addtokomafont{partnumber}{\mdseries} %encabezamientos sin negrita
%\addtokomafont{chapter}{\mdseries} %encabezamientos sin negrita
%\setkomafont{disposition}{\normalcolor\bfseries} %no sans serif
%\setkomafont{disposition}{\normalcolor\mdseries} %no negrita

\parskip=6pt\clubpenalty=10000\widowpenalty=10000

\newcommand{\preLilyPondExample}{\vspace{-10pt}}

\newcommand{\lpversion}{2.13.4}
\newcommand{\defsep}{\textbf{$\|$}}
\newcommand{\software}{\emph{software}}
\newcommand{\negspace}{\vspace{-10pt}}  %{\vspace{-20pt}}
\newcommand{\seppar}{
\bigskip
%\vspace{6pt}
}

%%%%%%%%%%%%%%%%%%%%%%%%%%%%%%%%%%%%%%%%%%%%%%%%%%%%%%%%%%%%%%%%%%%%%%%%%%%%%%%%%%%%%%%%%%%
\begin{document}

\setcounter{section}{12} %para 13 variables


\section{Variables. Reutilización del código.}


\subsection{Modelo}

Para este ejercicio de procedente de la Corrente de la partita para
flauta, BWV 1030, de Bach, debe escribir la música dentro de una
variable, y emplear la variable más tarde dentro de un contexto de
pentagrama:

\bigskip

\begin[staffsize=17.5]{lilypond}
corrente =  {  
\time 3/4
    \partial 8
    e''8 |
    a'16 ( b'16 c''16 d''16 e''8 fis''16  gis''16) a''8 b''8 |
    c'''8 a'8 g'4 b''4 |
f'8 a''16 gis''16 a''8 e'8 d'8 b''8 
gis''4.\trill b''16 a''16 gis''16 fis''16 e''16 d''16 |}

\new Staff { \corrente }


\end{lilypond}


\subsection{Definición y utilización de variables}

Dando nombre a una expresión, podemos reutilizar la expresión
escribiendo su nombre precedido de una barra invertida.

\begin[verbatim,staffsize=17.5]{lilypond}
musica = \relative c' { c1 d e }

<<
  \new Staff { \musica }
  \new Staff { \musica }
>>
\end{lilypond}

Estas expresiones con un nombre se llaman \textbf{variables}.  Los
nombres de variable no pueden contener números, aunque sí vocales
acentuadas y 'ñ'.  El problema es que se deben utilizar con el mismo
nombre exacto que se les dio al crearlas, por lo que se recomienda
utilizar nombres fáciles de escribir sin errores.

También se pueden almacenar en variables expresiones de letra:

\begin[verbatim,staffsize=17.5]{lilypond}
musicaUno = \relative c' { f1 e d c }
musicaDos = \relative c { \clef bass d1 g g, c }
letra = \lyricmode { La, la, la, la. }

<<
  \new Staff { \musicaUno } \addlyrics { \letra }
  \new Staff { \musicaDos } \addlyrics { \letra }
>>
\end{lilypond}

\end{document}


 %\documentclass[12pt,a4paper,oneside]{scrbook} % la clase book del Koma-script bundle
\documentclass[a4paper,10pt,oneside,headinclude,titlepage]{article} % la clase book del Koma-script bundle
%\linespread{1.25}
\usepackage{setspace}
%\usepackage{tikz}
%\usetikzlibrary{fit,shapes}
\usepackage[spanish]{babel}
%\usepackage{verbatim} %para el entorno comment
%\usepackage{moreverb} %para los ejemplos de lilypond, aporta verbatimtabinput
%\usepackage{alltt} %para los ejemplos de lilypond, aporta verbatiminput
%\usepackage{sverb} %para los ejemplos de lilypond, aporta verbinput
%\usepackage{fancyvrb} %para los ejemplos de lilypond, aporta VerbatimInput
\pagestyle{empty}
\usepackage[utf8]{inputenc}
\usepackage[T1]{fontenc} %posiblemente sirva para eliminar el problema del enguionado de palabras acentuadas. Lo quitamos provisionalmente para evitar un error
\usepackage{textcomp} % recomendación de Javier Bezos para completar la fuente

\usepackage[margin=2cm]{geometry}
\usepackage{graphicx}
%\usepackage{url}

\usepackage[utopia]{mathdesign}
%\usepackage{mathptmx} %mejor que Times    % alternativa a Charter


%\typearea[0mm]{13}% same as class options above
%\usepackage{newcent}
%\addtokomafont{part}{\mdseries} %encabezamientos sin negrita
%\addtokomafont{partnumber}{\mdseries} %encabezamientos sin negrita
%\addtokomafont{chapter}{\mdseries} %encabezamientos sin negrita
%\setkomafont{disposition}{\normalcolor\bfseries} %no sans serif
%\setkomafont{disposition}{\normalcolor\mdseries} %no negrita

\parskip=6pt\clubpenalty=10000\widowpenalty=10000

\newcommand{\preLilyPondExample}{\vspace{-10pt}}

\newcommand{\lpversion}{2.13.4}
\newcommand{\defsep}{\textbf{$\|$}}
\newcommand{\software}{\emph{software}}
\newcommand{\negspace}{\vspace{-10pt}}  %{\vspace{-20pt}}
\newcommand{\seppar}{
\bigskip
%\vspace{6pt}
}

%%%%%%%%%%%%%%%%%%%%%%%%%%%%%%%%%%%%%%%%%%%%%%%%%%%%%%%%%%%%%%%%%%%%%%%%%%%%%%%%%%%%%%%%%%%
\begin{document}

\setcounter{section}{13} %para 14 articulaciones


\section{Articulaciones y digitaciones: Sonatina de Bartok (I)}


\subsection{Modelo}

Este ejercicio procede de la Sonatina para piano de Bela
Bartok. Contiene una indicación metronómica, digitaciones, acentos y
otras articulaciones.

\bigskip

\begin[staffsize=17.5]{lilypond}
\new Staff \relative c' { \time 2/4 \tempo "Moderato" 4=80
	 
	e32(-> -2 \mf 
	%-"pesante"
	fis e8 d16) e32( ->fis e8 d16)
	c16(-3 b)-. a-. b-. c4-3--->
	
	e32(-> -2 
	fis e8 d16) e32( ->fis e8 d16)
	c16(-4 b)-. a-. g-. a4-3--->
}
\end{lilypond}

\subsection{Tempo con indicación metronómica}
Además de la instrucción normal de tempo del tipo
\verb+\tempo "Allegro"+, podemos añadir un valor de figura, seguido de
un signo igual y un número, que se imprimirán entre paréntesis como
indicación metronómica. La indicación metronómica aparecerá sola, si
no se escribe ningún texto dentro de las comillas. También puede
aparecer sin los paréntesis, quitando el texto y las comillas.

\begin[relative=1,verbatim,staffsize=17.5]{lilypond}
\tempo "Allegro" 1=120 c1 c c \tempo "" 1=80 c c c \tempo 1=40 c c c
\end{lilypond}

\subsection{Digitaciones y articulaciones}

Mediante el guión podemos adjuntar a una nota articulaciones,
digitaciones o textos:

\begin[relative=1,verbatim,staffsize=17.5]{lilypond}
 c4-> c-- c-. c-2 c1-"texto"
\end{lilypond}

En general se recomienda dejar la situación automática que LilyPond da
a las articulaciones, pero también se puede forzar su posición encima
o debajo de la nota sustituyendo el guión por un circunflejo o una
barra baja, respectivamente:

\begin[relative=2,verbatim,staffsize=17.5]{lilypond}
 b4-> b-- b-. b-2 b1-"texto"   % automático
 b4^> b^- b^. b^2 b1^"texto"   % siempre arriba
 b4_> b_- b_. b_2 b1_"texto"   % siempre abajo
\end{lilypond}



\end{document}


 % \version "2.17.0"

\section{Reguladores.  Elementos de marcado. \emph{Sonatina} de Bartok (II)}


\subsection{Modelo}

Para este ejercicio podemos reutilizar la parte hecha en el anterior.
Aquí hemos incorporado una inscripción textual en tipo itálica, en el
primer compás, y hemos añadido otros cuatro compases que contienen
reguladores.

\bigskip

\begin[staffsize=17.5,line-width=17\cm]{lilypond}
\new Staff \relative c' {
	\time 2/4 \tempo "Moderato" 4=80

	e32(-> -2 \mf

	fis 	-\markup{ \italic "pesante" } e8 d16) e32( ->fis e8 d16)
	c16(-3 b)-. a-. b-. c4-3--->

	e32(-> -2
	fis e8 d16) e32( ->fis e8 d16)
	c16(-4 b)-. a-. g-. a4-3---> \break

	g8.->(-1 \< a32 b c8-.-4)\! c-.-4
	c16-3( b-.) a-. b-. c32-^(\> d c d \times 4/5 { c[ d c b a]\! }

	g8.->-2)\< ( a32 b c8-.-5)\! c-.-5
	c16-4( b-.)\> a-. g-. a4-.-- \!
	\bar "||"


}
\end{lilypond}

\subsection{Elementos de marcado}
Ya vimos que los textos se pueden adjuntar a una nota como si se
tratase de una articulación. Estos textos simples no admiten ningún
formato, pero los elementos de marcado sí permiten una amplia variedad
de estilos.  Por ahora, tan sólo pondremos como ejemplo un texto en
itálica para expresar un cierto carácter:

\begin[relative=1,verbatim,staffsize=17.5]{lilypond}
c8 -\markup{ \italic "dolce" }
d e f g a b c
\end{lilypond}

\subsection{Reguladores}

Para obtener indicaciones gráficas de matices dinámicos, se marca la
nota de comienzo y la de final con dos símbolos especiales.  La marca
de final para cancelar el regulador sólo es necesaria cuando no ocurre
una indicación dinámica normal.

\begin[relative=1,verbatim,staffsize=17.5]{lilypond}
c8 \p \< d e f g a b c
d \f \> c b a g f e d \! c1
\end{lilypond}

\subsection{Acento}

El acento en forma de 'v' o de 'v invertida' utiliza el símbolo del
acento circunflejo, con un significado distinto al de forzar la
dirección:

\begin[relative=1,verbatim,staffsize=17.5]{lilypond}
c1 -^
c ^^
\end{lilypond}



 \section{Ornamentos barrocos: Aria de las Variaciones Goldberg.}


\subsection{Modelo}

El siguiente fragmento es el comienzo del Aria de las ``Variaciones
Goldberg'' BWV 988 de Bach.  Contiene abundantes apoyaturas y
ornamentos barrocos, y nos servirá para introducir las notas de adorno
en general.

\bigskip

\begin[staffsize=17.5,line-width=17\cm]{lilypond}
\relative c''' {
	\key g \major \time 3/4
        g4 g( a8.\prallmordent) b16
        a8 \appoggiatura g16 fis8 \appoggiatura e16 d2
        g,4\prallmordent g4.\downprall fis16 g
        a32( g fis16) g32( fis e16) \appoggiatura e8 d2
        d'4 d( e8.\prallmordent) f16
        e8 \appoggiatura d16 c8 \appoggiatura b16
        a4.
        fis'8 \turn
        g32( fis16.) a32( g16.) fis32( e16.) d32( c16.)
        \appoggiatura c8 a'8. c,16
        b32( g16.) fis8
        \appoggiatura fis8 g2\prallmordent
}
\end{lilypond}

\subsection{Notas de adorno}
Para conseguir un mordente de una nota (que está tachado por una línea
inclinada y se ejecuta rápidamente) o una apoyatura (que tiene el
valor que representa) empleamos las instrucciones \verb+\appoggiatura+
y \verb+\acciaccatura+, respectivamente, como prefijos:

\begin[relative=2,verbatim,staffsize=17.5]{lilypond}
g2 \acciaccatura b8 a8 g a b
\appoggiatura gis4 a2 r
\end{lilypond}

Estas notas se dibujan con una ligadura que las une a la nota
principal.  Al utilizar \verb+\grace+ como prefijo de una expresión
obtenemos mordentes de varias notas, pero es necesario escribir la
ligadura explícitamente:

\begin[relative=0,verbatim,staffsize=17.5]{lilypond}
\clef bass
\grace { a32[( c e] } a8) a a a
\end{lilypond}


\subsection{Algunas abreviaturas y otros ornamentos barrocos}

Nuestro modelo no utiliza acciaccaturas pero sí emplea grupos
abreviados de notas de adorno muy utilizados en el barroco; las
palabras clave se emplean como sufijos, a modo de articulaciones, pero
sin el guión de éstas.  Usaremos \verb+\prallmordent+ para el
semitrino largo con resolución descendente, \verb+\downprall+ para el
semitrino con preparación descendente y \verb+\turn+ para el grupeto
circular.

\begin[relative=3,verbatim,staffsize=17.5]{lilypond}
a2 \prallmordent
g  \downprall
f1 \turn
\end{lilypond}



 %\documentclass[12pt,a4paper,oneside]{scrbook} % la clase book del Koma-script bundle
\documentclass[a4paper,10pt,oneside,headinclude,titlepage]{article} % la clase book del Koma-script bundle
%\linespread{1.25}
\usepackage{setspace}
%\usepackage{tikz}
%\usetikzlibrary{fit,shapes}
\usepackage[spanish]{babel}
%\usepackage{verbatim} %para el entorno comment
%\usepackage{moreverb} %para los ejemplos de lilypond, aporta verbatimtabinput
%\usepackage{alltt} %para los ejemplos de lilypond, aporta verbatiminput
%\usepackage{sverb} %para los ejemplos de lilypond, aporta verbinput
%\usepackage{fancyvrb} %para los ejemplos de lilypond, aporta VerbatimInput
\pagestyle{empty}
\usepackage[utf8]{inputenc}
\usepackage[T1]{fontenc} %posiblemente sirva para eliminar el problema del enguionado de palabras acentuadas. Lo quitamos provisionalmente para evitar un error
\usepackage{textcomp} % recomendación de Javier Bezos para completar la fuente

\usepackage[margin=2cm]{geometry}
\usepackage{graphicx}
%\usepackage{url}

\usepackage[utopia]{mathdesign}
%\usepackage{mathptmx} %mejor que Times    % alternativa a Charter


%\typearea[0mm]{13}% same as class options above
%\usepackage{newcent}
%\addtokomafont{part}{\mdseries} %encabezamientos sin negrita
%\addtokomafont{partnumber}{\mdseries} %encabezamientos sin negrita
%\addtokomafont{chapter}{\mdseries} %encabezamientos sin negrita
%\setkomafont{disposition}{\normalcolor\bfseries} %no sans serif
%\setkomafont{disposition}{\normalcolor\mdseries} %no negrita

\parskip=6pt\clubpenalty=10000\widowpenalty=10000

\newcommand{\preLilyPondExample}{\vspace{-10pt}}

\newcommand{\lpversion}{2.13.4}
\newcommand{\defsep}{\textbf{$\|$}}
\newcommand{\software}{\emph{software}}
\newcommand{\negspace}{\vspace{-10pt}}  %{\vspace{-20pt}}
\newcommand{\seppar}{
\bigskip
%\vspace{6pt}
}

%%%%%%%%%%%%%%%%%%%%%%%%%%%%%%%%%%%%%%%%%%%%%%%%%%%%%%%%%%%%%%%%%%%%%%%%%%%%%%%%%%%%%%%%%%%
\begin{document}

\setcounter{section}{16} %para 17 set


\section{Cuarteto de cuerda. La instrucción \texttt{set}.}


\subsection{Modelo}

Presentamos el comienzo de un cuarteto de Beethoven en el que puede
verse el nombre de los instrumentos, clave de Do en la viola y una
serie de títulos adicionales.  El contexto que engloba a los
pentagramas es \verb+StaffGroup+.

\bigskip

\begin[staffsize=12.5]{lilypond}
%#(set-global-staff-size 12.5)


\header {
title = "SECHS QUARTETTE"
subtitle = "für 2 Violinen, Bratsche und Violoncell"
composer = "L. VAN BEETHOVEN"
opus = "Opus 18. nº1."
piece = "Quartett nº1."
%subsubtitle = "."
dedication = "Dem Fürsten von Lobkowitz gewidmet."}

violinUno = \relative c' { \set Staff.instrumentName = #"Violino I "
			\key f \major 	\time 3/4
			\tempo "Allegro con brio"
			f4\p ~
			 f8( g16 f) e8_. f_. 
			c4 r4 r4 
			f4~ f8( g16 f) e8_. f_. 
			d4 r4 r4
			f'4~ \< f8( g16 f) e8-. f-.
			g2(\> bes,4)
			a2(\! d8. bes16) 
			a2( g4) \break
			
}
			
			
violinDos = \relative c' { \set Staff.instrumentName = #"Violino II " \key f \major 	\time 3/4 
			f4~\p f8( g16 f) e8_. f_. 
			c4 r4 r4 
			f4~ f8( g16 f) e8_. f_. 
			d4 r4 r4
			bes'2.(\<
			bes2)\>( g4)
			f2(\! bes8. g16)
			f2( e4)
}
			

viola = \relative c { \set Staff.instrumentName = #"Viola" \key f \major \time 3/4  \clef alto
			f4~ \p f8( g16 f) e8_. f_. 
			c4 r4 r4 
			f4~ f8( g16 f) e8_. f_. 
			d4 r4 r4
			d'2.(\<
			c2.)(\>
			c4)(\! d g,)
			c8( b c b c4)
}
			
cello = \relative c { \set Staff.instrumentName = #"Violoncello" \key f \major \time 3/4  \clef bass
			f4~ \p f8( g16 f) e8^. f^. 
			c4 r4 r4 
			f4~ f8( g16 f) e8^. f^. 
			d4 r4 r4
			d2.(\<
			e2.)(\>
			f4)(\! d bes) 
			c2.
			
}


\score {
\new StaffGroup {
<<	\new Staff {\violinUno}
    \new Staff {\violinDos}
    \new Staff {\viola}
    \new Staff {\cello}  >>
    }
    \layout{indent=1.5\cm}
    \midi{}
}
\end{lilypond}

\subsection{Establecer el nombre del instrumento con \texttt{set}.}

En LilyPond, los contextos tienen una serie de propiedades que podemos
modificar mediante la instrucción \verb+\set+ indicando el nombre del
contexto y de la propiedad que se quiere modificar, separados mediante
un punto, después un signo igual ``\verb+=+'' y finalmente el valor
deseado para la propiedad.  Por ejemplo, si queremos establecer la
propiedad \verb+instrumentName+ (nombre del instrumento) del contexto
\verb+Staff+ al valor ``Flauta'', escribimos lo siguiente:

\begin[relative=2,verbatim,staffsize=17.5]{lilypond}
\set Staff.instrumentName = #"Flauta"
f2.
\end{lilypond}

\subsection{Clave de viola: Do en tercera}

Podemos aplicar la clave de Do en tercera línea que utiliza la viola
con la abreviatura ``alto'' como argumento para la instrucción
\verb+\clef+.  Como siempre, las notas se deben introducir en su
altura real, independientemente de la clave:

\begin[relative=1,verbatim,staffsize=17.5]{lilypond}
\set Staff.instrumentName = #"Viola"
\clef alto
c2.
\end{lilypond}


\subsection{Títulos adicionales}

Para el ejemplo hemos cumplimentado algunos títulos adicionales en el
bloque \verb+\header+.  Ya conocemos \verb+title+ (título),
\verb+substitle+ (subtítulo) y \verb+composer+ (autor).  Ahora
añadimos los siguientes: \verb+opus+, \verb+piece+ y \verb+dedication+
para el número de Opus, denominación de la pieza y dedicatoria,
respectivamente.

\begin{verbatim}
\header{ title="Título"
 dedication="Dedicatoria"
 opus="Número de Opus"
 piece="Pieza"
} 
\end{verbatim}


\end{document}


 % \version "2.17.0"

\section{Polifonía compleja: \emph{Canción del Emperador}.}


\subsection{Modelo}

Esta versión para guitarra de la ``Canción del Emperador'' de Luis de
Narváez, sobre el tema \emph{Mille Regretz}, es una transcripción de
las tablaturas originales y presenta una polifonía enrevesada porque
todas las voces están contenidas en un solo pentagrama.  Para este
ejemplo hará falta una cuidadosa planificación y el empleo de
silencios ocultos.  Además contiene indicaciones del número de cuerda,
silencios con altura definida y otros ajustes menores.  Por sencillez,
el resto de los ajustes necesarios se omiten por el momento.

\bigskip

%\hspace{3cm}
\begin[staffsize=15,line-width=14\cm,indent=1.5\cm]{lilypond}

% canción del emperador. Narváez
\header { title = "CANCIÓN DEL EMPERADOR"
	composer= "Luys de Narváez"
	opus =  "(1530-1550)"
}
\version "2.17.0"

cantus = \relative c'{
	<b fis' fis\4 b>1
	<b b' b\3 >
	<c e'>
	<c' e>
	<b d> \break %5
	d2 c4 b
	a2 g
	a1
	r2 <b g'>2
	<b g'> g' \break %10
	fis fis4 g
	e2 e
	r4 g8_( fis) e d e fis
	<g, b g'>2 <g b g'> \break
	fis'8( g) fis g fis e d c
	b2  e  ~
	e4 e4 dis cis
	<dis fis,>2 <dis fis,>
 }


altus = \relative c'' {
	s1
	s1
	g4 a8 b c d c b
	a,8 b c d e4 a,
	b8 cis d e fis4 b, % \break %5
	b4 d a g
	c4\rest dis e2 ^~
	e4 dis e dis
	\stemUp g2 s2
	s2 \voiceThree b4 cis %10
% En este ejemplo chocan algunas notas en la misma columna a causa de \shiftOff. Compases 11 y 12
	\shiftOff d1
	c1
}

tenor = \relative c { \voiceTwo
	s1
	s1
	s1
	s1
	s1 %5
	s1
	fis2 r4 e
	fis1
	<e e'>1
	e2 e'
	d2 b
	c2 a8 b c d
	<e g b>1
	e2 e
	<d a'>1
	<e g>1
	<b fis'>1
	b2 b
} %5

\score {
\new Staff \relative c' { \set Staff.instrumentName = "Guitarra"
    \time 4/4
    \key g \major

    <<
	\new Voice { \voiceOne \cantus }
	\new Voice { \voiceFour \altus }
	\new Voice { \voiceTwo \tenor }
    >>
  }

\layout { left-margin=0\cm
%, indent=2\cm
}

}
\end{lilypond}




\subsection{Silencios ocultos o de separación}
\label{sub:ocultos}

Será de gran ayuda, para la realización de partituras de polifonía
compleja, la inserción de silencios de separación.  Éstos no se
imprimen pero ocupan el mismo espacio que una figura con la duración
correspondiente.  Para insertarlos se utiliza \verb+s+ como si fuera
una nota; en el siguiente ejemplo hemos rellenado la voz inferior con
un silencio de blanca oculto:

\begin[relative=2,verbatim,staffsize=17.5]{lilypond}
\new Staff <<
  { c4 d e f }
  \\
  { a,4 s2 d4 }
>>
\end{lilypond}


\subsection{Silencios con altura. Ligaduras orientadas}

Los silencios se suelen colocar automáticamente de forma que no haya
colisiones con las notas de las otras voces.  Sin embargo, si queremos
colocar un silencio a la altura de una nota determinada, lo hacemos
mediante \verb+\rest+ que convierte la nota anterior en un silencio.

\begin[relative=2,verbatim,staffsize=17.5]{lilypond}
c4 g'8\rest g, c2 ^~ c1 _~ c
\end{lilypond}

En este ejemplo, además, hemos utilizado los indicadores de dirección
para orientar la ligadura de unión hacia arriba o hacia abajo.

\subsection{Planificación de las voces en polifonía compleja.}

Existen varias técnicas para resolver el problema de la polifonía en
casos similares al de arriba.  Una solución es preparar una
construcción polifónica \verb+<< { } \\ { } >>+ por cada compás o por
cada pocos compases.  Hoy recomendamos un enfoque orientado a voces
que se extienden a lo largo de toda la pieza, quizá utilizando
silencios ocultos como se explica en el apartado \ref{sub:ocultos}.


No es necesario que los acordes de redonda pertenezcan a distintas
voces.  Aquí usamos, simplemente, un acorde:

\begin[verbatim,staffsize=17.5]{lilypond}

vozUno = \relative c''{ <g e'>1 }
vozDos = \relative c' { r2 c8 d e f }
\new Staff << \vozUno \vozDos >>

\end{lilypond}


Para la orientación adecuada de las plicas y el desplazamiento de las
voces secundarias, tendremos en cuenta que las voces 1 y 3 tienen las
plicas en direcciones opuestas a las voces 2 y 4; las voces tercera y
cuarta, además, llevan un desplazamiento a la derecha.  El mismo
efecto puede conseguirse en cualquier momento gracias a las
instrucciones \verb+\voiceOne+, \verb+\voiceTwo+, \verb+\voiceThree+,
y \verb+\voiceFour+.

En el siguiente ejemplo utilizamos \verb+\voiceOne+ y
\verb+\voiceThree+ para que las dos voces tengan las plicas hacia
arriba, y la voz de contralto tenga un desplazamiento horizontal a la
derecha:

\begin[verbatim,staffsize=17.5]{lilypond}
soprano   = { g''1 }
contralto = { b'2 cis'' }
\new Staff << \new Voice { \voiceOne \soprano }
              \new Voice { \voiceThree \contralto } >>
\end{lilypond}


En caso necesario puede usarse \verb+\shiftOff+ para anular el
desplazamiento de una voz secundaria.  Podemos recurrir a
\verb+\stemUp+ y \verb+\stemDown+ para orientar las plicas hacia
arriba o hacia abajo:

\begin[verbatim,relative=2,staffsize=17.5]{lilypond}
\stemUp c4 b a g \stemDown f e d c
\end{lilypond}



\subsection{Saltos de línea manuales.}

En ocasiones conviene insertar un cambio de línea manual: lo hacemos
con \verb+\break+, aunque sólo se producirá el salto si en el momento
actual es posible saltar.  Lo podemos comprobar aquí:

\begin[relative=2,verbatim,staffsize=17.5,line-width=6\cm]{lilypond}
c4 c c \break c
c1 \break
c1 c
\end{lilypond}

\subsection{Números de cuerda.}

Las cuerdas de la guitarra se indican mediante un número dentro de un
círculo. Las escribimos con \verb+\1+, \verb+\2+, etc.



 \include{apuntes-19-bajocifrado}
 \section{Grabadores. \emph{Misa de Notre Dame}, de Machaut}


\subsection{Modelo}


Esta transcripción moderna del \emph{Ite missa est} de la \emph{Misa
  de Notre Dame}, de Guillaume de Machaut (s.XIV) contiene
indicaciones de tesitura y omite la indicación de compás.  Esto se
hace añadiendo o retirando los complementos grabadores o ``plug-ins''
encargados de hacer esta tarea, en todos los contextos de pentagrama.
Además, la letra contiene apenas dos sílabas, por lo que es necesario
saltar muchas notas de una sílaba a la siguiente.  También utiliza un
sistema especial para coro, sin líneas divisorias entre los
pentagramas.

\bigskip

\begin[staffsize=15,
line-width=17\cm,
indent=0
]{lilypond}
#(set-global-staff-size 14)

%{

\header { title = "Ite missa est"
	subtitle = \markup {
		\score { { c'8 c c c c c c c c c c c c } \layout { indent=0  \context { \Staff \remove "Time_signature_engraver"
	\remove Bar_engraver } } }
	}


}



%}

triplum = \relative c'' { \time 3/2 c1.
g1.
e2 bes'8 a g f e4 g
f1.
r4 g e  r4 r8  a4 g8
f4. a4 c a8 a g a f
e2 fis4. g4 a fis8
g1.
}

motetus = \relative c' { f1.
d1.
c2 d4 f e d
c1.
e2 r d
f4 c2  d4 r8  c4 b8
cis2. d4 r8  d cis d
e1.
}

tenor = \relative c { \clef "G_8" f1.
g1.
a1.
f1.
g1.
r2 f f
e2 d1(
c1.)
}

contratenor = \relative c' { \clef "G_8" c1.
r2 bes1
a2 d, e
f2 a1
g4 c,2 d e4
f2 r g
a2 b1
c1.
}

\score {
\new ChoirStaff <<
\new Staff { \triplum } \addlyrics { De -- \repeat unfold 26 { \skip 8 } o }
\new Staff { \motetus } \addlyrics { De -- \repeat unfold 14 { \skip 8 } o  gra- }
\new Staff { \tenor } \addlyrics { De -- \repeat unfold 7 { \skip 8 } o }
\new Staff { \contratenor } \addlyrics { De -- \repeat unfold 14 { \skip 8 } o }
>>

\layout {
   \context { \Staff \remove Time_signature_engraver
	            \consists Ambitus_engraver }
           }
        }

\end{lilypond}


\subsection{Skips}

Cuando se escribe la letra de un pasaje melismático (que tiene muchas
notas para una sílaba), es frecuente recurrir al empleo del salto o
``skip'', que se inserta en la letra una vez por cada nota que
queremos saltar.  La instrucción requiere una duración como argumento,
aunque esta duración se desprecia y no influye en el resultado.

\begin[verbatim,relative=1,staffsize=17.5]{lilypond}
{ c d e f } \addlyrics { De -- \skip 4 \skip 4 o }
\end{lilypond}

Se recomienda el empleo de repeticiones desplegadas para insertar
múltiples saltos seguidos.
\label{unfold}
\begin[verbatim,relative=1,staffsize=17.5]{lilypond}
{ \time 3/4 f16 g f e f e d e d c b d c4 }
\addlyrics { Pen -- \repeat unfold 11  \skip 4  sier! }
\end{lilypond}


\subsection{Añadir y quitar grabadores}

Existen más de 120 grabadores o complementos distintos, que se agrupan
en contextos como los conocidos \verb+Staff+ o \verb+Voice+.  Si un
grabador se suprime de un contexto, no se trazan aquellos elementos
que estaban a su cargo; si se añade un grabador nuevo a un contexto,
podrán aparecer elementos nuevos.

Los contextos se añaden o se suprimen con las instrucciones
\verb+\consists+ y \verb+\remove+.  Existen dos formas de hacerlo:
para cada contexto uno a uno, o para todos los contextos de una
clase al mismo tiempo.
\label{consists}

\textbf{1.} Para añadir un grabador a todos los contextos de una
clase, abrimos un bloque \verb+\layout+ fuera de la expresión
principal de nuestra partitura.  La sintaxis para añadir un
grabador (aquí el grabador \verb+Ambitus_engraver+ que se encarga
de la indicación de tesitura) es la siguiente:


\begin[verbatim,staffsize=15]{lilypond}
{ c'2 c'' }
\layout {
  \context { \Staff \consists Ambitus_engraver }
}

\end{lilypond}

Se pueden introducir varias instrucciones \verb+\consists+ o
\verb+\remove+ dentro del mismo bloque \verb+\context+.

\textbf{2.}\label{with} Para gestionar los grabadores en un solo contexto,
introducimos las instrucciones \verb+\consists+ y \verb+\remove+
en el momento de la creación del contexto, dentro de un bloque
\verb+\with+.  En el siguiente ejemplo vamos a crear dos contextos
de pentagrama y en el segundo de ellos quitaremos los grabadores
de la indicación de compás y de la clave.

\begin[verbatim,relative=2,staffsize=15]{lilypond}
<<
  \new Staff { c1 }
  \new Staff \with { \remove Time_signature_engraver
                     \remove Clef_engraver
                   } { c1 }
>>
\end{lilypond}


\subsection{Sistemas de coro}

La música vocal polifónica suele emplear sistemas de pentagramas sin
líneas divisorias entre ellos, para no entorpecer al texto.  LilyPond
lo hace mediante el contexto \verb+ChoirStaff+.  Aquí vemos un ejemplo
con dos pentagramas:



\begin[verbatim,staffsize=15]{lilypond}
letra = \lyricmode { Hal -- le -- lu -- jah! }
\new ChoirStaff <<
  \time 2/4 \partial 8
  \new Staff \relative c'' { c16 c c8 c }            \addlyrics { \letra }
  \new Staff \relative c   { \clef bass e16 e f8 c } \addlyrics { \letra }
>>
\end{lilypond}

 %\setcounter{section}{20} %para 21 MIDI


\section{MIDI. Transposición. Il est bel et bon, de P. Passereau}


\subsection{Modelo}

Hoy mostraremos la forma de escuchar las partituras elaboradas con
LilyPond.  A partir de una partitura se puede generar un pequeño
archivo MIDI que contiene las notas, pero no los sonidos: éstos se
generan cuando el archivo se reproduce y pueden sonar algo distintos
en cada sistema.  Sin embargo, podemos elegir los instrumentos que
determinarán el timbre de cada pentagrama.

Aprovecharemos para realizar un transporte sobre el tono original, un
tono y medio hacia arriba.

No todos los elementos de la música se exportan al archivo MIDI. Antes
de introducir el modelo, el comienzo de \emph{Il est bel et bon}, una
la \emph{chanson} de Pierre Passereau (s.XVI), analizaremos qué
elementos merece la pena omitir si solamente queremos producir un
archivo MIDI para escuchar la música.  En primer lugar, el ejemplo
original:


\bigskip

\begin[staffsize=15,
line-width=17\cm,
indent=0
]{lilypond}

%#(set-global-staff-size 14)

%\version "2.13.16"

 \header { title = "Il est bel et bon" composer = "Pierre Passereau" }

sop = \relative c' { %\tempo 1=60
	% \key d \minor
	\time 2/2
	d8 e f g a4 a
	a4 a a a
	g4 e r2
	d8 e f g a4 a
	a4 a a a
	g4 e8 e g4 e8 e
	f4 d d c d2 }

alt = \relative c' { R1
	a8 b c d e4 e
	e4 e e e
	f2 e
	a,8 b c d e4 e
	e4 e b b8 b
	d4 a a a
a2 }

ten = \relative c { \clef "G_8"
	r2 d8 e f g
	a4 a c c
	b4 b g2
	a2 a8 g f g
	a4 a c c
	b2 g4 g
	a4. g8 f4 e
d2 }

baj = \relative c { \clef bass
	R1
	r2 a8 b c d
	e4 e e e
	d2 a~
	a2 a8 b c d
	e4 e e e
	d4 d a a
d2 }


letrasop = \lyricmode { Il est bel et bon, bon,
	bon, bon, bon, com --
	mè -- re,
	Il est bel et bon, bon,
	bon, bon, bon, com --
	mè -- re, com -- mè -- re, com --
	mè -- re mon ma -- ry. }

letraalt = \lyricmode { Il est bel et bon, bon,
	bon, bon, bon, com --
	mè -- re,
	Il est bel et bon, bon,
	bon, com -- mè -- re, com --
	mè -- re, mon ma -- ry. }


letraten = \lyricmode { Il est bel et
	bon, bon, bon, bon,
	bon, com -- mè --
	re, Il est bel et
	bon, bon, bon, com --
	mè -- re, com --
	mè -- re mon ma -- ry. }

letrabaj = \lyricmode { Il est bel et
	bon, bon, bon, com --
	mè -- re,
	Il est bel et
	bon, bon, bon, com --
	mè -- re mon ma -- ry. }


\score {
	% \transpose d f
	\new ChoirStaff
	<<
		\new Staff { \set Staff.instrumentName="S"
                  \set Staff.midiInstrument = "oboe"
		\sop }
		\addlyrics { \letrasop }
		\new Staff { \set Staff.instrumentName="A"
                  \set Staff.midiInstrument = "oboe"
		\alt }
		\addlyrics { \letraalt }
		\new Staff { \set Staff.instrumentName="T"
                  \set Staff.midiInstrument = "bassoon"
		\ten  }
		\addlyrics { \letraten }
		\new Staff { \set Staff.instrumentName="B"
                  \set Staff.midiInstrument = "clarinet"
		\baj }
		\addlyrics { \letrabaj }
>>
%\midi { }
%\layout { }
}

\paper { system-count=2 indent=1\cm }

\end{lilypond}


Ahora bien: los nombres de los pentagramas, la letra de la canción,
las agrupaciones de pentagramas que trazan llaves, los títulos, las
articulaciones y otros elementos no se reflejan el el resultado MIDI.
La indicación metronómica sí se refleja; así pues, bastaría con
dejarlo de esta manera:

\bigskip

\begin[staffsize=15,
line-width=17\cm,
indent=0
]{lilypond}

%#(set-global-staff-size 14)

%\version "2.13.16"

% \header { title = "Il es bel et bon" composer = "Pierre Passereau" }

sop = \relative c' { \tempo 1=60
	% \key d \minor
	\time 2/2
	d8 e f g a4 a
	a4 a a a
	g4 e r2
	d8 e f g a4 a
	a4 a a a
	g4 e8 e g4 e8 e
	f4 d d c d2 }

alt = \relative c' { R1
	a8 b c d e4 e
	e4 e e e
	f2 e
	a,8 b c d e4 e
	e4 e b b8 b
	d4 a a a
a2 }

ten = \relative c { \clef "G_8"
	r2 d8 e f g
	a4 a c c
	b4 b g2
	a2 a8 g f g
	a4 a c c
	b2 g4 g
	a4. g8 f4 e
d2 }

baj = \relative c { \clef bass
	R1
	r2 a8 b c d
	e4 e e e
	d2 a~
	a2 a8 b c d
	e4 e e e
	d4 d a a
d2 }


\score {
	% \transpose d f
	% \new ChoirStaff
	<<
		\new Staff { \set Staff.midiInstrument = "oboe"
		\sop }
		%\addlyrics { \letrasop }
		\new Staff { \set Staff.midiInstrument = "oboe"
		\alt }
		%\addlyrics { \letraalt }
		\new Staff { \set Staff.midiInstrument = "bassoon"
		\ten  }
		%\addlyrics { \letraten }
		\new Staff { \set Staff.midiInstrument = "clarinet"
		\baj }
		%\addlyrics { \letrabaj }
>>
%\midi { }
%\layout { }
}

\paper { system-count=1 }


\end{lilypond}


\subsection{El bloque midi}

Para producir un archivo MIDI, debe existir un bloque \verb+\midi{ }+,
posiblemente vacío, dentro de un bloque \verb+\score+ explícito que
contiene la música.  En nuestro ejemplo, después de definir las
variables (en su caso), podemos hacerlo así:

\begin{verbatim}
\score {
  <<
     \new Staff { \soprano }
     \new Staff { \alto }
     \new Staff { \tenor }
     \new Staff { \bajo }
  >>
  \midi { }
}
\end{verbatim}

Si existe un bloque \verb+\midi{}+ y no hay un bloque \verb+layout{}+, el proceso será muy rápido pero no habrá ninguna salida en PDF.  Para obtener MIDI y PDF, deben estar los dos bloques, así:

\begin{verbatim}
\score {
    ...

  \midi { }
  \layout{ }
}
\end{verbatim}

El archivo MIDI tiene la extensión \verb+.mid+ (en UNIX: \verb+.midi+)
y se creará en el mismo directorio que el archivo fuente.  En teoría,
debería poder reproducirse fácilmente mediante un doble click.

\subsection{Instrumentos MIDI}

 Es posible (y así lo haremos) determinar un sonido MIDI por cada voz
 y cada pentagrama.  En esta ocasión lo haremos estableciendo el valor
 de la propiedad \verb+midiInstrument+ del contexto Score, a un texto
 que corresponderá exactamente al nombre oficial del instrumento según
 el estándar General Midi, que puede consultarse en las tablas de la
 documentación de LilyPond.

 Para \emph{Il est bel et bon} vamos a asignar a las voces de soprano
 y contralto un sonido de oboe; a la de tenor, de fagot (``bassoon'');
 y a la de bajo, un sonido de clarinete (``clarinet'') (aunque es muy
 grave para el clarinete, podría ser un clarinete bajo).

\begin{verbatim}
\new Staff {
  \set Staff.midiInstrument = "oboe"
  \soprano
}
\end{verbatim}




\subsection{Transposición}

Ahora queremos que la música suene un tono y medio más alta.
Utilizamos para ello la instrucción \verb+\transpose+ previa a la
expresión musical.  La instrucción admite dos alturas de nota que se
toman como referencia del tono de partida y del de destino de la
transposición.  Lo vemos en este ejemplo donde hemos introducido
música en Sol mayor y la hemos impreso en Fa mayor (un tono más baja):

\begin[verbatim,staffsize=17.5]{lilypond}
\transpose g f
  \relative c'' { \key g \major
     g4 a b c d1
  }
\end{lilypond}

Se transporta la armadura solamente si está establecida dentro de la
expresión que se transporta.

 %\setcounter{section}{20} %para 21 MIDI


\section{Trabajo colaborativo: quinteto ``La trucha'' de Schubert.}


\subsection{Modelo}

He aquí el comienzo del Tema con variaciones del quinteto D.667 para
piano, violín, viola, violoncello y contrabajo, ``La trucha'', de
F. Schubert.  La realización de este ejercicio puede hacerse en grupos
de dos a cuatro personas.  Aprenderemos a incluir el contenido de
distintos documentos dentro de uno solo, y a variar el tamaño de los
pentagramas.


\bigskip

\begin[staffsize=15,
line-width=17\cm,
indent=0
]{lilypond}
#(set-global-staff-size 17)


global = { \key d \major \time 2/4 \tempo "Andantino" \partial 8 }


violin = \relative c'' { \global a8 \pp
	d8.-. ( d16-. fis8-. fis-.)
	d4( a)
	a8.. a32 e'16.( d32 cis16. b32)
	a4. a8
	d8.-. ( d16-. fis8-. fis-.)
	d4( a8) d
	cis8[( \grace{ d16[ cis] } b16.) cis32] d8(-> gis,)
	a4. a8
	}

viola = \relative c' { \global \clef alto r8 \pp
	<a fis'>4-.( q8-. q-.)
	fis'2
	<g a>4-. ( q8-. q-.)
	q8-. e( fis g)
	<a, fis'>4-.( q8-. q-.)
	fis'4. fis8
	e4.  e8
	e8-. g( fis e)
	}

cello = \relative c' { \global \clef bass r8 ^\pp
	d4-.( d8-. d-.)
	a8( b16 cis d4)
	cis4 cis16.( d32 e16. d32)
	cis8-. cis( d e)
	d4-.( d8-. d-.)
	a8( b16. cis32 d8 a)
	a8( gis16.) a32 b8( d)
	cis8-. e( d cis)
	}

contrabajo = \relative c {
\global \clef bass r8 \pp
	d4-.( d8-. d-.)
	d2
	a'4-.( a8-. a-.)
	a4 r
	d,4-.( d8-. d-.)
	d4. d8
	e4. e8 a,4 r
	}

pianoManoDerecha = \relative c'' { \global \clef treble r8
	R2*8
	}

pianoManoIzquierda = \relative c'' { \global \clef bass r8
	R2*8
	}

<<
	\new Staff \with { fontSize = #-3
		\override StaffSymbol #'staff-space = #(magstep -3)
	%	\override StaffSymbol #'thickness = #(magstep -3)
	}
		{  \violin }
	\new Staff \with { fontSize = #-3
		\override StaffSymbol #'staff-space = #(magstep -3)
	%	\override StaffSymbol #'thickness = #(magstep -3)
	}
		{ \viola }
	\new Staff \with { fontSize = #-3
		\override StaffSymbol #'staff-space = #(magstep -3)
	%	\override StaffSymbol #'thickness = #(magstep -3)
	}
		{ << \cello \\ \contrabajo >> }
	\new PianoStaff <<
	  \new Staff { \pianoManoDerecha }
	  \new Staff { \pianoManoIzquierda }
	>>
>>


\paper { indent=0 system-count =1 }

\end{lilypond}


\subsection{Inclusión de documentos}

La inclusión de archivos externos nos será de utilidad para mantener
la independencia entre el contenido musical y la estructura de una
partitura.  Mediante esta técnica podemos crear documentos que
dependen de otros archivos, quizá realizados por otras personas.  De
esa manera, un equipo puede trabajar de forma colaborativa sobre un
proyecto común.

La inclusión de archivos externos funciona de la siguiente manera:
supongamos que el archivo \verb+violin.ly+ contiene solamente lo
siguiente:

\begin{verbatim}
violin = \relative c'' { \key d \major \time 2/4 \partial 8
  a8 \pp d8.-. ( d16-. fis8-. fis-.)
  }
\end{verbatim}

Un archivo diferente, llamado \verb+parte-violin.ly+, puede incluirlo
especificando su nombre:

\begin{verbatim}
\include "violin.ly"

\score {
   \new Staff { \violin }
}
\end{verbatim}

Entonces, todo el contenido del archivo especificado se inserta en
sustitución de la instrucción \verb+\include+, cuando se procesa el
archivo \verb+parte-violin.ly+.

\begin[relative=2,staffsize=17.5]{lilypond}
\key d \major \time 2/4 \partial 8
a8 \pp d8.-. ( d16-. fis8-. fis-.)
\end{lilypond}

Otro documento puede contener una estructura distinta e incluir el
mismo archivo que contiene la música, por ejemplo
\verb+piano-general.ly+ podría ser algo así:

\begin{verbatim}
\include "violin.ly"
\include "viola.ly"
\include "cello.ly"
\include "contrabajo.ly"
\include "piano.ly"

<<
  \new Staff { \violin }
  \new Staff { \viola }
  \new Staff { << \cello \\ \contrabajo >> }
  \new PianoStaff <<
    \new Staff { \pianoManoDerecha }
    \new Staff { \pianoManoIzquierda }
  >>
>>
\end{verbatim}

De esa forma estamos produciendo la partitura general del pianista, y
la particella de los instrumentos, a partir de la misma fuente y en
archivos independientes.  Para este ejercicio pediremos a cada miembro
de un grupo, que se encargue de elaborar una parte de los
instrumentos, en archivos separados, tales que cuando se inserten en
la estructura que hemos dado arriba, produzcan la partitura general de
piano.  Para probar el resultado de su trabajo parcial, puede crear un
archivo de particella que incluya solamente su parte.

Dado que el nombre del archivo tiene que coincidir exactamente con el
argumento de la instrucción \verb+\include+, se recomienda elegirlo de
tal forma que contenga solamente letras minúsculas, y no espacios o
vocales acentuadas.

\subsection{Tamaño de la partitura y de los pentagramas}
\label{tamano-global}

El tamaño normal de una partitura de LilyPond es ``20'', pero puede
cambiarse de forma global con una instrucción del lenguaje interno
``Scheme'' \verb+set-global-staff-size+, como en este ejemplo que
produce música en miniatura, de tamaño 10:


\begin[verbatim,relative=2]{lilypond}
#(set-global-staff-size 10)
\key d \major \time 2/4 \partial 8
a8 \pp d8.-. ( d16-. fis8-. fis-.)
\end{lilypond}

Las instrucciones del lenguaje Scheme van precedidas del carácter de
almohadilla \verb+#+.  Esta instrucción tiene un efecto general sobre
la partitura; para pentagramas sueltos debe especificarse dentro de
una cláusula \verb+\with{}+ al crear el contexto, con las siguientes
instrucciones:

\begin{verbatim}
\new Staff \with {
  fontSize = #-3
  \override StaffSymbol #'staff-space = #(magstep -3)
}
\end{verbatim}

Aquí hemos usado la sobreescritura de propiedades que se estudiará con
detalle en un ejercicio posterior.  La primera instrucción reduce el
tamaño de la fuente de tipografía musical y la segunda el tamaño del
pentagrama.  Una especificación de -3 para el tamaño, en ambos casos,
establece una reducción como la que hemos aplicado en el ejemplo de
Schubert para los instrumentos de cuerda.

 \include{apuntes-23-despertad}
 % \version "2.17.0"

\section{Varias partituras en un documento: libros.  Marcados de alto nivel}


\subsection{Modelo}

Cuando se quiere elaborar un documento de varias páginas que contiene
distintas piezas (p.ej. los movimientos de una sonata) o distintas
obras completas que deben ir en páginas separadas, se hace necesario
utilizar los conceptos de libro, parte y pieza.

Para este ejercicio debemos elegir \emph{ocho} piezas ya realizadas
que se insertarán en el texto del documento.  Produciremos, a partir
de un solo documento de LilyPond, dos archivos de salida PDF o
\emph{libros}, cada uno de los cuales tendrá dos partes en páginas
separadas, y cada parte de cada libro contendrá dos partituras con
especificación de la pieza en la cabecera correspondiente.

\bigskip
\parindent=0mm
\fbox{\includegraphics[width=39mm]{outA1}}
\fbox{\includegraphics[width=39mm]{outA2}}
\fbox{\includegraphics[width=39mm]{outB1}}
\fbox{\includegraphics[width=39mm]{outB2}}
\parindent=6mm
\bigskip

Se deberá tener cuidado en que los nombres de variable,
procedentes de distintas piezas, no se confundan al incluirlos en
el mismo texto.

% Aumentar la separación entre sistemas
%\def\betweenLilyPondSystem#1{\vspace{0.2cm}\linebreak}

\subsection{Bloques de partitura: distintas piezas}

Si en un documento aparece más de un bloque \verb+\score+, una pieza
aparecerá a continuación de la otra, en la misma página. Sólo se
imprimen, para cada una, las cabeceras \verb+piece+ y \verb+opus+
(véase el apartado \ref{pieceopus}, pág. \pageref{pieceopus}). Se
pueden volver a definir los títulos de cabecera para cada partitura.

Para imprimir el título de cada partitura, se requiere un bloque
\verb+\paper+ (fuera de todas las partituras) que contenga el
siguiente ajuste:

\begin{verbatim}
\paper{
  print-all-headers = ##t
}
\end{verbatim}

En nuestro ejemplo esto no es necesario, pero debe tenerse en cuenta
que las cabeceras de cada partitura deben ir \emph{después} de la
música dentro del bloque \verb+\score+, así:

\begin{verbatim}
\score{ \cumple
  \header { piece = "Cumpleaños feliz" }
}
\end{verbatim}


\subsection{Elementos de marcado del nivel superior}

Si se escribe un elemento \verb+\markup+ fuera de cualquier partitura,
aparecerá en su lugar correspondiente.  Estos elementos se denominan
``Marcados de alto nivel'' porque se encuentran en lo alto de la
jerarquía de la sintaxis de un documento, es decir, fuera de cualquier
bloque.  Los elementos de marcado del nivel superior se pueden emplear
para escribir las letras de las estrofas de una canción.  También
puede insertarse una separación para distanciar dos partituras entre
sí, o antes de una partitura para separarla del borde superior:

\begin{verbatim}
\score { ... }
\markup{ \vspace #5 }
\score { ... }
\end{verbatim}

\subsection{Bloques de libro: distintos archivos de salida}

Cuando en un documento aparecen bloques \verb+\book+, se produce un
archivo de salida PDF distinto para cada bloque, con todas las
partituras que contenga.  El primer archivo tiene el mismo nombre del
documento, como es usual, y a partir del segundo libro se añade un
número consecutivo al final del nombre.

Es preciso tener en cuenta la siguiente salvedad: los documentos de
partitura completos, que pueden contener indicaciones de alto nivel
como definiciones de variables, no pueden incluirse mediante la
instrucción \verb+\include+ dentro de un bloque \verb+\book+.  Las
definiciones de variables son instrucciones del nivel sintáctico
superior y por tanto deben ir fuera de cualquier bloque.

\subsection{Bloques de parte: distintas páginas en un libro}

Es sencillo mantener en páginas separadas dos o más conjuntos de
partituras dentro de un libro: tan sólo hay que encerrar cada conjunto
en un bloque \verb+\bookpart+.  Las cabeceras de título de la primera
parte serán las que se impriman al principio del libro.

\subsection{Pie de página}
\label{tagline}

La cabecera \verb+tagline+ contiene las indicaciones que aparecen en
la última página de cada libro.  Si no se especifica una línea de pie,
la línea predeterminada contiene una nota de autopromoción del
programa y la versión del mismo con que se hizo el archivo de salida.

\subsection{Ejemplo}

La siguiente estructura es un ejemplo que resume lo explicado más
arriba.

\begin{verbatim}
musicaUno = {...}
musicaDos = {...}
...

\book{                          % primer libro
  \bookpart{                      % primera parte
    \header{ title = ... }          % cabeceras de este libro y parte  
    \score{ \músicaUno...             % primera partitura
      \header { piece = ... }         % cabeceras de esta partitura  
    }
    \score{ ... }                   % segunda partitura
  }                               % fin de la parte y salto de página
  \bookpart{ ... }                % segunda parte
}
\book{ ... }                    % segundo libro
\end{verbatim}

 % \version "2.17.0"

\section{Instrumentos transpositores. Cambio de pentagrama}


\subsection{Modelo}

La música para instrumentos transpositores requiere particellas
preparadas para ellos con la transposición inversa, y las partes en
tono de concierto (p.ej. para el piano) no llevan esta transposición.
Nuestro ejemplo de hoy procede de la segunda pieza de concierto para
dos clarinetes y piano, de Mendelssohn.  Podemos transcribir la música
de los clarinetes a partir de las particellas o a partir de la
partitura de conjunto; en cualquier caso, el MIDI de la partitura de
conjunto debe sonar a la altura correcta.

Debemos producir, a partir de la misma fuente, un documento con la
partitura de piano en tono de concierto, y en páginas o en documentos
separados las particellas de los clarinetes preparadas para un
instrumento en Si$\flat$.

La parte de piano necesita un cambio de pentagrama en el tercer
compás.

\begin{lilypond}
clarinete = \relative c''' { \tempo "Presto"
	\key e \minor \time 2/2  \partial 4. c8( \ff b) ais-.
b4.-> fis8-. g4.-> dis8-.
e4-. e,-. r8 e'8( g) fis-. e( d) c-. b-. a-. g-. fis-. e-.
c'4-. fis,-.
}

clarineteUno = { \transposition bes \clarinete }

clarineteDos = \transpose c c, { \transposition bes \clarinete }
pianoMD = \relative c'' { %\tempo "Presto"
	\key d \minor \time 2/2 \partial 4. bes8( \ff  a) gis-.
	a4.-> e8 f4.-> cis8
	d4-. <f, a>-. r8 d'( f) e-.
	d8( c) bes-. a-. \change Staff = "abajo" \voiceOne g-. f-. e-. d-.
	\change Staff ="arriba" <bes' d e>4-. q-.
	}
pianoMI = \relative c { \key d \minor \clef bass \time 2/2 \partial 4. bes8( a) gis-.
	a4.-> <e e'>8 <f f'>4.-> <cis cis'>8
	<d d'>4-> q-> r8 d'( f) e-.
	d8-. c-. bes-. a-.  \voiceTwo g-. f-. e-. d-.
	g4-. g-.
}


\score {
<< 	\new Staff \with { fontSize = #-3 
		\override StaffSymbol #'staff-space = #(magstep -3)
		instrumentName="Clarinete I" 
	%	\override StaffSymbol #'thickness = #(magstep -3)
	} \transpose c bes, { \clarineteUno }
	\new Staff \with { fontSize = #-3 
		\override StaffSymbol #'staff-space = #(magstep -3)
		instrumentName="Clarinete II"
	%	\override StaffSymbol #'thickness = #(magstep -3)
	} \transpose c bes, { \clarineteDos }
	\new PianoStaff \with { instrumentName="Piano" } <<
		\new Staff ="arriba" { \pianoMD }
		\new Staff ="abajo" { \pianoMI }
	>>
>>
\layout{}
\midi{}
}


\score {
	\new Staff \with { instrumentName="Clarinete I" } { \clarineteUno }
\layout{}
\midi{}
}

\score {
	\new Staff \with { instrumentName="Clarinete II" } { \clarineteDos }
\layout{}
\midi{}
}

\end{lilypond}

% Aumentar la separación entre sistemas
%\def\betweenLilyPondSystem#1{\vspace{0.2cm}\linebreak}

\subsection{Cambios de pentagrama}

Podemos cambiar la música de un pentagrama a otro si en la creación de
los contextos hemos dado un nombre a cada uno.  Después, la
instrucción \verb+\change+ nos permite especificar el contexto al que
deseamos cambiar, llamándolo por su nombre.  Por ejemplo:

\begin[verbatim]{lilypond}
musica = \relative c''{ g16 e c \change Staff = "abajo" g c,4 }
<< \new Staff = "arriba" { \musica }
   \new Staff = "abajo"   { \clef bass s2 }
>>
\end{lilypond}

El contexto que recibe la música debe existir mientras duran las notas
cambiadas, lo que podemos hacer con silencios de separación como puede
verse en el ejemplo.  Después del cambio, puede ser necesario orientar
las plicas mediante una instrucción del tipo \verb+\voiceOne+.


\subsection{Instrumentos transpositores}

Para fijar el intervalo de transposición de un instrumento, se usa la
instrucción \verb+\transposition+ seguida de una nota en altura
absoluta.  El intervalo que forma la nota con el Do central, \verb+c'+,
corresponde al intervalo transpositor del instrumento.  Por ejemplo,
si un instrumento como el clarinete en Si$\flat$ transporta un tono
hacia abajo, debemos poner lo siguiente:

\begin{verbatim}
\transposition bes
\end{verbatim}

A continuación podemos escribir la música tal y como aparece en su
particella.

\begin[verbatim]{lilypond}
\relative c''' { \transposition bes
  \key e \minor
  \time 2/2
  \partial 4. c8( \ff b) ais-. }

\end{lilypond}



Más tarde podemos imprimir la partitura en tono de concierto aplicando
a esta música el transporte del instrumento, en este caso un tono
hacia abajo:

\begin[verbatim]{lilypond}
clarinete = \relative c''' { \transposition bes
  \key e \minor
  \time 2/2
  \partial 4. c8( \ff b) ais-. }

\transpose c bes, { \clarinete }
\end{lilypond}


Si no nos preocupa la producción de MIDI, para producir una
particella con destino a un instrumento transpositor a partir de
música en tono de concierto, sólo hay que aplicar el transporte
inverso en el momento de imprimir la particella.

\subsection{Notas}
\begin{itemize}
\item El nombre del sistema de piano puede establecerse con una de
  las dos instrucciones siguientes:

\begin{tabular}{l     r}
\verb+\set PianoStaff.instrumentName = "Piano"+ & (dentro de la música) \\
\verb+\new PianoStaff \with{ instrumentName = "Piano" }+ & (al crear el contexto)
\end{tabular}

\medskip

Véase la sección \ref{instrumentname}
(pág. \pageref{instrumentname}).  Vimos también el uso de
\verb+\with+ en el apartado \ref{with} (pág. \pageref{with}).

\item La literatura en inglés se refiere a los cambios de
  pentagrama como \emph{cross-staff beams}, es decir, barras de
  pentagrama cruzado.

\item Algunas ediciones de música para instrumentos transpositores
  imprimen las partes de éstos en la partitura general con el
  transporte aplicado, igual que en las particellas, lo que puede
  ser confuso para el pianista (pues no son los sonidos reales)
  pero tiene la ventaja de que los nombres de las notas son los
  mismos en todos los papeles.
\end{itemize}

 % \version "2.17.0"

\section{Márgenes. \emph{Frère Jacques}}


\subsection{Modelo}

Es fácil establecer los márgenes de la partitura y el número de
sistemas, así como otros retoques de aspecto final. En este ejemplo
hemos establecido unos márgenes izquierdo y derecho de 3 cm., (sin
sangrado en la primera línea), un margen superior de 4 cm. y un margen
inferior de 10 cm.  Hemos aumentado el tamaño global de la partitura a
26 puntos y hemos distribuido los compases en cuatro sistemas que
ocupen toda la página hasta el margen inferior.  Se han suprimido la
línea informativa final y los números de compás.  Hemos marcado las
entradas del canon con números en negrita.  Por último, añadimos
indicaciones de verso para señalar el idioma de cada uno de los
textos.

\bigskip
\begin{center}
\fbox{\includegraphics[width=100mm]{frerejacques}}
\end{center}
\bigskip


% Aumentar la separación entre sistemas
%\def\betweenLilyPondSystem#1{\vspace{0.2cm}\linebreak}

\subsection{Márgenes.  Sangrado de la primera línea}

Es bastante amplio el abanico de variables que se pueden ajustar
dentro del bloque \verb+\layout+ para controlar la forma en que se
imprime la partitura.  Cada bloque \verb+\layout+ dentro de una
partitura controla los ajustes para esa partitura; para ajustar todas
las partituras de un documento al mismo tiempo, se usa de la misma
forma el bloque \verb+\paper+ fuera de los bloques \verb+\score+.

La variable que controla el sangrado de la primera línea es
\verb+indent+, y aquellas que controlan la medida de los márgenes son
\verb+left-margin+, \verb+right-margin+, \verb+top-margin+ y
\verb+bottom-margin+, para los márgenes izquierdo, derecho, superior e
inferior, respectivamente.

Para expresar las medidas en milímetros o centímetros, se utiliza
\verb+\mm+ ó \verb+\cm+ después del valor; no es necesario si el valor
es cero.  Por ejemplo, aquí hemos suprimido el sangrado de la primera
línea y establecido el margen izquierdo a 2 centímetros:

\begin{verbatim}
\paper {
  indent = 0
  left-margin = 2\cm
}
\end{verbatim}


\subsection{Número de sistemas}

Aunque es posible insertar saltos de línea en los lugares deseados
mediante la instrucción \verb+\break+, puede ser más cómodo
especificar simplemente cuántos sistemas queremos.  Lo hacemos dando
un valor a la variable \verb+system-count+ en el bloque \verb+\layout+
o \verb+\paper+.


\subsection{Ocupación de la página}

Si se da un valor falso a la variable \verb+ragged-last-bottom+, el
último sistema llegará hasta el final de la última página y la
partitura ocupará todo el papel hasta el margen inferior.

\begin{verbatim}
\paper {
  ragged-last-bottom = ##f
}
\end{verbatim}


\subsection{Quitar los números de compás}

Para suprimir los números de compás, se quita el grabador
\verb+Bar_number_engraver+ del contexto \verb+\Score+, véase el
apartado \ref{consists} de la página \pageref{consists}.


\subsection{Negrita}

Dentro de un elemento de marcado, la instrucción \verb+\bold+ establece un estilo negrita:

\begin[verbatim,fragment]{lilypond}
c'^\markup{ Normal, \bold Negrita. }
\end{lilypond}


\subsection{Verso de la letra}

La propiedad de contexto ``\verb+stanza+'' fija un texto que aparece antes
de la primera sílaba de la letra.  Se puede establecer con \verb+\set+
dentro de la expresión que contiene la letra, o dentro de un bloque
\verb+\with+ al crear el contexto \verb+\Lyrics+.

\begin[verbatim,fragment]{lilypond}
\new Staff { c'1 }
\addlyrics { \set stanza = "1." Dooo }
\end{lilypond}


\subsection{Notas}

\begin{itemize}
\item Consulte cómo cambiar el tamaño global de la partitura en el apartado \ref{tamano-global} de la página \pageref{tamano-global}.

\item Suprima la línea informativa final definiendo \verb+tagline = ##f+
  dentro del bloque \verb+\header+.

\item Pruebe cómo suena el canon con este bloque de ejemplo: \footnotesize
\begin{verbatim}
\score { << \new Staff \with { midiInstrument="trumpet" }{ \tempo 4=150 \canon }
            \new Staff \with { midiInstrument="muted trumpet" }  { R1*2 \canon }
            \new Staff \with { midiInstrument="accordion" }      { R1*4 \canon }
            \new Staff \with { midiInstrument="choir aahs" }     { R1*6 \canon } >>
	  \midi{} \layout{} }
\end{verbatim}
\normalsize
\end{itemize}


 \includepdf{holst-marte}

\section{Un gran ejemplo orquestal: Marte, de Holst (1)}


\subsection{Modelo}

Para esta gran partitura orquestal perteneciente a la suite ``Los
planetas'' de Gustav Holst, vamos a trabajar en varias fases. Por
ahora añadiremos dos instrumentos: los timbales y el gong. El
ejemplo contiene repeticiones de trémolo, silencios de compás
completo en 5/4 y una partitura completa dentro de un marcado.

\bigskip

% Aumentar la separación entre sistemas
\def\betweenLilyPondSystem#1{\vspace{0.4cm}\linebreak}

\begin[line-width=17cm]{lilypond}

\version "2.13.18"

           timpaniI = \relative g, { \clef bass
                         \key c \major
   \times 2/3 { g8\p ^"wooden sticks" g g }  g4 g g8 g g4
   \times 2/3 { g8 g g } g4 g g8 g g4
   \times 2/3 { g8 g g } g4 g g8 g g4
   \times 2/3 { g8 \< g g } g4 g g8 g g4\!
   \times 2/3 { g8 \> g g } g4 g g8 g g4\! }

           timpaniII = { \clef bass
                         \key c \major
	   R1*5/4 R1*5/4 R1*5/4 R1*5/4 R1*5/4 }

	gong = { g2.:32 \pp  g2:32 g2.:32 g2:32 g2.:32 g2:32 g2.\< :32 g2\!:32 g2.\>:32 g2:32 \! }


incipitTimpaniGroup = \markup {
	\score{
		 \new PianoStaff  <<
                    \set PianoStaff.instrumentName= \markup {
				\center-column {"6 Timpani" "(two players)"}
			}
			\new Staff { \set Staff.instrumentName = "I"
				\clef bass
				\time 3/2
				\cadenzaOn g,2 d2 bes,2
			}
			\new Staff { \set Staff.instrumentName = "II"
				\clef bass
				\time 3/2
				\cadenzaOn c2 es2 a,2
			}
		>>

	\layout {
		\context { \Staff
                         \remove "Time_signature_engraver"
                         \remove "Clef_engraver"
		}
		line-width=2.5\cm
                indent=1\cm
		%margin-left=0\cm
	} %layout
  } %score
} %markup

        #(set-global-staff-size 15.5)
	% #(set-default-paper-size "a3")

   \paper {  ragged-right=##f
             system-count=1
             }


%\header { title =  "I. Mars, the Bringer of War"
% title = \markup { \fontsize #6 { \smallCaps {  "I. Mars, " } "the Bringer of War" } }
	%	copyright = "Francisco Vila, sobre un trabajo de Guadalupe Cuevas Piñero"
%}


\score {
    \new StaffGroup <<   %\tempo "Allegro"       % main
	\time 5/4

    \new PianoStaff \with { systemStartDelimiter=#'SystemStartBar } <<  %timpani
         \set PianoStaff.instrumentName =
		\markup {
			\incipitTimpaniGroup
		}
	    \new Staff  {  \timpaniI }
	    \new Staff  { \timpaniII } >>

    \new RhythmicStaff		%gong
	{ \set Staff.instrumentName = "Gong"
	\gong }

>> %main
   \layout { indent=3.5\cm
	   \context { \Staff
%                        \override InstrumentName #'padding = #-50
%              \override VerticalAxisGroup #'minimum-Y-extent = #'(-3 . 3)
%	       \override instrumentName #'font-size = #8.0
	   }
   }%layout

} %score

\end{lilypond}

\subsection{Preparación del incipit: partitura dentro de un marcado}

A veces se denomina ``incipit'' a una pequeña partitura previa al
sistema que contiene la música real, y que tiene una finalidad
informativa.  La parte de timbales contiene un incipit que
prepararemos como una partitura normal y luego la insertaremos
dentro del nombre del sistema de piano de los timbales.

Haremos un elemento de marcado que contendrá una partitura
completa en la forma de un bloque \verb+\score+, y que deberá
incluir un bloque \verb+\layout{}+. Este elemento de marcado se
asignará a una variable que luego podrá utilizarse en cualquier
lugar donde se admita un marcado.  Aquí vemos un ejemplo sencillo:

\begin[verbatim]{lilypond}
partiturita=\markup{
  \score {
    \new RhythmicStaff { \time 2/4
      \times 2/3 { c'4 c'8 }
    }
    \layout{
      \context { \RhythmicStaff
               \remove "Time_signature_engraver"
      }
    }
  }
}

\new Staff \relative c' {
  c8_\partiturita c c c c c c c
}
\end{lilypond}

\subsection{Cadenza}

El fragmento del incipit aparece sin compasear, lo haremos de la
siguiente manera:

\begin[verbatim,relative=2]{lilypond}
\new Staff { \cadenzaOn c2 d e }
\end{lilypond}

Además no tiene indicación de compás ni clave, para ello hay que
eliminar los grabadores correspondientes del contexto de Staff: véase
el apartado \ref{consists} de la página \pageref{consists}.

\subsection{Columnas en el marcado}

Son muy numerosas las instrucciones de marcado y es recomendable
consultar la documentación oficial para aprender a usarlas.  La
instrucción de marcado que utilizaremos para el nombre de dos
líneas del sistema de timbales (en sustitución de una simple
línea) es \verb+\center-column+.  Esta instrucción admite un
bloque con varias expresiones textuales, que colocará centradas y
superpuestas:

\begin[verbatim]{lilypond}
\markup{
  \center-column {"6 Timpani" "(two players)"}
}

\end{lilypond}

Después emplearemos este marcado como el valor de la propiedad
\verb+PianoStaff.instrumentName+ de nuestra pequeña partitura.

Así pues, tenemos un marcado con dos líneas que es el nombre del
sistema de piano de una pequeña partitura, que a su vez está
dentro de un marcado que es el nombre del sistema de piano de los
timbales de la partitura grande.

\subsection{Figuras escaladas}

El escalado de valores se efectúa gracias al asterisco como
símbolo de multiplicación.  De esta forma podemos escribir una
figura con el aspecto de un valor dado, pero con la duración
reducida o ampliada según un factor.

Lo usaremos para rellenar compases de 5/4 con silencios de compás
completo, que se escriben mediante la letra R~mayúscula.  Si además
utilizamos otro factor de multiplicación entero, aparecerá
repetido el compás de silencio.

\begin[verbatim,relative=2]{lilypond}
\time 5/4 R1*5/4 c2. c2 R1*5/4*3
\end{lilypond}


\subsection{Repeticiones de trémolo}

Los trémolos, usados en el golpe repetido del gong, se expresan
mediante el símbolo de dos puntos seguido del valor de repetición.

\begin[verbatim,relative=2]{lilypond}
c2:8 c2:16 c2:32
\end{lilypond}


\subsection{Notas}

\begin{itemize}
\item Consulte en el apartado \ref{rhythmicstaff} de la página
\pageref{rhythmicstaff} la forma de crear un pentagrama de una sola
línea para percusión.

\item Puede probar muchas combinaciones para las variables del
  bloque \verb+\layout+ de la partitura del incipit, pero éstas
  pueden servir de orientación:

\begin{verbatim}
  line-width=2.5\cm
  indent=1\cm
\end{verbatim}

\item Suprimiremos las llaves del sistema de piano de los timbales
  con la asignación

\begin{verbatim}
\new PianoStaff \with { systemStartDelimiter=#'SystemStartBar }
  << ...
\end{verbatim}


\end{itemize}

 \section{Sobreescritura de propiedades: Marte, de Holst (II)}


\subsection{Modelo}

Para completar el ejemplo orquestal, hoy aprenderemos a mover
objetos para ajustar su posición; en el caso que nos ocupa, esto
ahorra espacio y permite un tamaño de los pentagramas algo mayor,
sin que se produzcan colisiones entre los objetos de los distintos
pentagramas, y todo ello de forma que la música quepa en una sola
página.  En el título usaremos el efecto \emph{smallcaps} de
mayúsculas pequeñas para ``Mars''.

En el fragmento aparecen los pentagramas de los fagotes y el
contrafagot, con el matiz \emph{mezzopiano} y la indicación
``III'' del tercer fagot desplazadas a la izquierda y hacia
arriba.  Se ha enmascarado en blanco el pentagrama detrás de estas
indicaciones para evitar la superposición.  Las pautas de
percusión se han acercado entre sí para ahorrar espacio. La
indicación \emph{piano} de los violines está también desplazada
para hacer sitio al texto \emph{col legno} del siguiente
pentagrama.  Para finalizar, hemos reducido el grosor de las
líneas de pauta para suavizar el aspecto demasiado negro de una
partitura orquestal a tamaño reducido.

\bigskip

% Aumentar la separación entre sistemas
\def\betweenLilyPondSystem#1{\vspace{0.4cm}\linebreak}

\begin[line-width=13cm]{lilypond}

\version "2.13.0"

juntaPentagrama = \with { \override VerticalAxisGroup #'staff-staff-spacing =
                   #'((basic-distance . 6) (padding . 0)) 
		   }


		bassoonsI =  \relative  g, {
			\clef bass
			\oneVoice R1*5/4 R1*5/4
			\voiceOne g2. ^"I II a2" ~ ( \p  g2 ~
			g2. ^\< d'2 ) \!  des2. ^\> ~ des2 \! \laissezVibrer % ~ des
		}

		bassoonsIII =  \relative  d, {
			\clef bass
			s1*5/4 s1*5/4 R1*5/4*2
			%  \once \override Voice.DynamicText #'extra-offset = #'(-2.9 . 2.9) 
			\override TextScript #'whiteout = ##t
			\override DynamicText #'whiteout = ##t
			\once \override DynamicText #'X-offset = #-4.5
			\once \override DynamicText #'extra-offset = #'(-0.1 . 2.3)
			\once \override TextScript #'outside-staff-priority = ##f
			\once \override TextScript #'X-offset = #-4
			des2.
			-"III"
			 \mp
			 \>  ~ des2 \! \laissezVibrer % ~ des
		}


%%%%%%%%%%%%%%%%%%%%%%%%%%%%%%%%%%%%%%%%%%%%%%%%%%%%%%%%%%%%%%%%%%%%%%%%%%%%%%%%%%%%%%%%%%%%%%

             doble = \relative  g, { \key c \major
			\clef bass
			R1*5/4 R1*5/4
			g2.\p ~ ( g2 ~ g2. \< d'2 \! ) des2. \> ~ des2 \! \laissezVibrer % ~ des
	     }

%%%%%%%%%%%%%%%%%%%%%%%%%%%%%%%%%%%%%%%%%%%%%%%%%%%%%%%%%%%%%%%%%%%%%%%%%%%%%%%%%%%%%%%%%%%%%%

%%%%%%%%%%%%%%%%%%%%%%%%%%%%%%%%%%%%%%%%%%%%%%%%%%%%%%%%%%%%%%%%%%%%%%%%%%%%%%%%%%%%%%%%%%%%%%%%%%%

	side  = { R1*5/4*5 }

%%%%%%%%%%%%%%%%%%%%%%%%%%%%%%%%%%%%%%%%%%%%%%%%%%%%%%%%%%%%%%%%%%%%%%%%%%%%%%%%%%%%%%%%%%%%%%%%%%%

	cymbals  = { R1*5/4*5 }

%%%%%%%%%%%%%%%%%%%%%%%%%%%%%%%%%%%%%%%%%%%%%%%%%%%%%%%%%%%%%%%%%%%%%%%%%%%%%%%%%%%%%%%%%%%%%%%%%%%

	drum  = { R1*5/4*5 }


       violinI = \relative g {
	\once \override DynamicText #'extra-offset = #'(-0.8 . 1)
	\once \override DynamicText #'X-offset = #-2.5
	\times 2/3 { g8\p ^"col legno" g g }  g4 g g8 g g4
	\times 2/3 { g8 g g } g4 g g8 g g4
	\times 2/3 { g8 g g } g4 g g8 g g4
	\times 2/3 { g8 \< g g } g4 g g8 g g4\!
	\times 2/3 { g8 \> g g } g4 g g8 g g4\! }

%%%%%%%%%%%%%%%%%%%%%%%%%%%%%%%%%%%%%%%%%%%%%%%%%%%%%%%%%%%%%%%%%%%%%%%%%%%%%%%%%%%%%%%%%%%%%%%%%%%%%%

	violinII = \relative g {
		\once \override DynamicText #'extra-offset = #'(-0.8 . 1)
		\once \override DynamicText #'X-offset = #-2.5
		\times 2/3 { g8 \p ^"col legno" g g }  g4 g g8 g g4
		\times 2/3 { g8 g g } g4 g g8 g g4
		\times 2/3 { g8 g g } g4 g g8 g g4
		\times 2/3 { g8 \< g g } g4 g g8 g g4\!
		\times 2/3 { g8 \> g g } g4 g g8 g g4\! }


        #(set-global-staff-size 10.5)  % antes 15.5 para a3
	#(set-default-paper-size "a4") % antes a3

\header {
	title = \markup { \fontsize #6 { \smallCaps {  "I. Mars, " } "the Bringer of War" } }
	tagline=##f
}


\score {

    % main
    \new StaffGroup <<   \tempo "Allegro"
	\time 5/4

    %bassoons
    \new PianoStaff  <<
	\new Staff  \with { instrumentName = "3 Bassoons" } { << \bassoonsI \\ \bassoonsIII >> }
	\new Staff  \with { instrumentName = "Double Bassoon" } { \doble }  >>

    %side drum
    \new RhythmicStaff
         \with { \juntaPentagrama
                 instrumentName = "Side Drum" }
               { \side }

    % cymbals
    \new RhythmicStaff
         \with { \juntaPentagrama
                 instrumentName = "Cymbals" }
	       { \cymbals  }

    %bass drum
    \new RhythmicStaff
         \with { \juntaPentagrama
                 instrumentName = "Bass Drum" }
	{ \drum }


    %violins
    \new PianoStaff  <<
	\new Staff \with { instrumentName = "1st Violins" }
		{ \violinI }
	\new Staff \with { instrumentName = "2nd Violins" }
		{ \violinII } >>

>> %main

   \layout { indent=1.5\cm %era 4 para a3

	      \context { \Score
	      \override StaffSymbol #'thickness = #(magstep -3)

	      }
   }


} %score


\paper { ragged-right=##f
       line-width=16.5\cm
	 page-count=1
	 system-count=1
}

\end{lilypond}


\subsection{Sobreescritura de propiedades}
\label{override}

Es importante aprovechar al máximo las posibilidades de tipografiado
automático de partituras que LilyPond ofrece, sin ninguna intervención
manual.  Sin ambargo, en el apartado \ref{tamano-global}
(pág. \pageref{tamano-global}) utilizamos tímidamente la
sobreescritura de propiedades para modificar el tamaño de un
pentagrama.  Las propiedades de los objetos gráficos tienen un valor
determinado que se usa para especificar la forma en que el objeto se
imprime.  Hay varias instrucciones que hacen posible la modificación
de estos valores, y la más frecuente es \verb+\override+.  Los valores
exactos son algo que se puede determinar mediante ensayo y error,
aunque existen ayudas muy valiosas como la herramienta Regla de
LilyPondTool (que no explicaremos aquí).  La
instrucción \verb+\override+ se utiliza de la siguiente manera:

\begin{verbatim}
\override contexto.objeto #'propiedad = #valor
\end{verbatim}

Que significa: asignar el \emph{valor} a la \emph{propiedad}
del \emph{objeto} dentro del \emph{contexto}.  El contexto
predeterminado es Voice y muchas veces se puede dejar sin
especificar.  Veamos a continuación un ejemplo del uso de la
sobreescritura de propiedades para mover objetos.


\subsection{Mover objetos}

Los matices dinámicos son objetos llamados internamente
DynamicText, que se imprimen en el lugar determinado por una serie
de variables.  Apliquemos la formulación general de la
instrucción \verb+\override+ que acabamos de mostrar, y
consignemos lo siguiente para cada uno de los apartados:

\medskip

\begin{tabular}{c|c|c|c}
Contexto & Objeto & Propiedad & Valor \\ \hline
Voice & DynamicText & 'extra-offset & '(-0.8 . 1) \\
Voice & DynamicText & 'X-offset & -2.5
\end{tabular}

\medskip

Estas medidas están expresadas en espacios de pentagrama, por lo que
(afortunadamente) no dependen del tamaño de éste.  Los dos números
entre paréntesis se refieren a las dimensiones X e Y.  El efecto de la
sobreescritura permanece hasta que se vuelva a sobreescribir o hasta
que se encuentre una instrucción \verb+\revert+ con el nombre del
objeto y la propiedad.  En el ejemplo se ve que las tres indicaciones
están afectadas por una sola sobreescritura:

\begin[fragment,verbatim]{lilypond}
    \override DynamicText #'extra-offset = #'(-0.8 . 1)
    \override DynamicText #'X-offset = #-2.5
    g1\p g\p g\p
\end{lilypond}


\subsection{Aplicación por una sola vez}

Las sobrreescrituras permanecen hasta nueva orden, pero por
comodidad, en caso de que sólo se necesite una vez, podemos
preceder la instrucción de sobreescritura por la palabra
clave \verb+\once+.  Aquí podemos ver que sólo la primera
indicación dinámica está afectada por \verb+\once \override+:

\begin[fragment,verbatim]{lilypond}
    \once \override DynamicText #'extra-offset = #'(-0.8 . 1)
    \once \override DynamicText #'X-offset = #-2.5
    g1\p g\p  g\p
\end{lilypond}


\subsection{Enmascarar en blanco}

Cuando se quieren tapar las líneas que caen detrás de una
indicación dinámica o textual, se le da un valor verdadero a la
propiedad \verb+whiteout+.

\medskip
\begin{tabular}{c|c|c|c}
Contexto & Objeto & Propiedad & Valor \\ \hline
Voice & TextScript, DynamicText  & 'whiteout & verdadero (\#t) o falso (\#f) \\
\end{tabular}
\medskip

Por ejemplo:
\begin{verbatim}
\override TextScript #'whiteout = ##t
\end{verbatim}

\begin{lilypond}
\new PianoStaff <<\new Staff 
	{ 
	\voiceTwo c'1 
	\override TextScript #'whiteout = ##t
	\override DynamicText #'whiteout = ##t
	
	\once \override TextScript #'outside-staff-priority = ##f
	\once \override TextScript #'X-offset = #-3

	\once \override DynamicText #'X-offset = #-4.5
	\once \override DynamicText #'extra-offset = #'(0 . 1.4)
	 c'2  
	\mp

	-"III"
	c'2 
}

\new Staff { c'1 c' }
>>
\end{lilypond}


\subsection{Grosor de las líneas del pentagrama}

Pruebe la siguiente sobreescritura para conseguir líneas más
delgadas en pautas sueltas o en toda la partitura:

\medskip
\begin{tabular}{c|c|c|c}
Contexto & Objeto & Propiedad & Valor \\ \hline
Staff, Score & StaffSymbol  & 'thickness & \#(magstep -3) \\
\end{tabular}
\medskip

Por ejemplo:
\begin{verbatim}
\new Staff \with { \override StaffSymbol #'thickness = #(magstep -3) }
\end{verbatim}

En el ejemplo que aparece a continuación podemos ver dos aplicaciones
de sentido opuesto, y el aspecto predeterminado en segundo lugar.


\begin[staffsize=10]{lilypond}
<<
  \new Staff \with { \override StaffSymbol #'thickness = #(magstep -6) } { s1 -"-6" }
  \new Staff \with { \override StaffSymbol #'thickness = #(magstep 0) } { s1 -"0"}
  \new Staff \with { \override StaffSymbol #'thickness = #(magstep 6) } { s1 -"+6" }
>>
\end{lilypond}


\subsection{Separación de pautas}

El espaciado vertical es un asunto delicado.  El ajuste de la
separación entre cada pauta y la siguiente se hace también mediante
sobreescritura de propiedades.  Aquí tenemos las que valen para la
última versión en el momento de escribir esto, la
\emph{estable}\ 2.14:

\medskip
\begin{tabular}{c|c|c|c}
Contexto & Objeto & Propiedad & Valor \\ \hline
Staff & VerticalAxisGroup  & 'staff-staff-spacing & \#((basic-distance . 6) (padding . 0)) \\
\end{tabular}
\medskip

Esta sobreescritura se puede almacenar en una variable para utilizarla
repetidas veces:

\begin[verbatim]{lilypond}
juntaPauta = \with {
                    \override VerticalAxisGroup #'staff-staff-spacing =
                      #'((basic-distance . 3) (padding . 0))
                  }
<<
  \new RhythmicStaff
    \with {
      instrumentName= "Side Drum"
      \juntaPauta
    }
    { c4 c c8 c c4 }
  \new RhythmicStaff
    \with {
      instrumentName= "Cymbals"
      \juntaPauta
    }
    { c4 c c8 c c4 }
>>
\end{lilypond}


\subsection{Notas}

\begin{itemize}
\item Para los nombres de instrumentos que contienen un bemol, use \verb+\flat+ dentro del elemento de marcado.

\begin[verbatim]{lilypond}
\new Staff
  \with {
    instrumentName= \markup {  "3 Clarinets in B" \flat }
  }
  s1
\end{lilypond}

\item La instrucción de marcado \verb+\smallCaps+ produce un estilo ``versalitas'' en que las minúsculas son mayúsculas pequeñas:

\begin[verbatim]{lilypond}
\markup { \smallCaps "Marte" }
\end{lilypond}


\end{itemize}

 % \version "2.17.0"

\section{Nombres de las notas en español}


\subsection{Modelo}

Es posible escribir la música en el lenguaje de LilyPond con los
nombres de las notas en español.  Sin embargo, no lo hemos visto
antes por las siguientes razones:

\begin{itemize}
\item La comunidad de usuarios de LilyPond a nivel global utiliza
  los nombres predeterminados (holandeses) principalmente.
\item Es bueno acostumbrarse a leer y escribir con soltura la
  música en el idioma en que están escritos la mayoría de los
  documentos que circulan entre usuarios de cualquier
  nacionalidad.
\item No es posible copiar y pegar directamente los ejemplos de un
  idioma dentro de un documento que utiliza otro idioma, y no se
  pueden mezclar fácilmente varios idiomas en el mismo documento.
\end{itemize}

A pesar de ello, es posible que algunos usuarios prefieran escribir
los nombres de las notas en su propio idioma, y por ello lo
mencionamos aquí.  El ejemplo que presentamos es el final del primero
de los \emph{Intermezzi para piano Op.4} de Schumann y contiene gran
cantidad de expresiones, digitaciones y articulaciones, polifonía en
el pentagrama inferior, notas de pentagrama cruzado y reguladores
textuales; proponemos que se tipografíe utilizando nombres de nota en
español.

\bigskip

% Aumentar la separación entre sistemas
\def\betweenLilyPondSystem#1{\vspace{0.4cm}\linebreak}

\begin[staffsize=15]{lilypond}
\version "2.17.0"

% Schumann, Op.4, I, 11 last measures

%#(set-global-staff-size 18)

rone = \relative c { \override Voice.Fingering #'avoid-slur = #'inside
\oneVoice R2.
R2.
\clef bass \voiceOne cis4(
\once \override DynamicText #'extra-offset = #'(-1.5 . -4.5)
d'^\sf <cis-4>8.. b32) \clef treble
<ais-1-2>4( <g'-5 e-3>\sf  <fis-4 d-1>8.. <e-5 cis-3>32
<d b>8.. <cis a!>32 \clef bass <b -3-5>4 b \clef treble
\change Staff = "LH"
<a e cis>8..)^\ff ( <a d,>32 <a e>8)
\change Staff = "RH"
<a b>_. \p <cis a>_. <d a>_.
<e b>8_.  <fis cis>_.\< <gis d>_. <a e>_. <b fis>_. <b e,>_.
\once \override  DynamicText #'whiteout = ##t
<cis a cis,>8.._([ \ff -4 <d a d,>32 <e a, e>8) <b fis>_. \p <cis gis>_. <d a>_. ]
\oneVoice
<e b>8-. <fis cis>-. <gis d>-.  \cresc <a e>-. <b fis>-. <b e,>-.
\once \override  DynamicText #'whiteout = ##t
<cis-4 a cis,>8..  \ff ( [ <d a d,>32 <e a, e>8) <gis, d>_> ( <a cis,>) <gis d>_> ( ]
<a cis,>8-.) r <a, cis,>4._> r8 \fermata \bar"|."
}

rtwo = \relative c { s2. s2. s4 fis g <fis_1>  s4 s
s4 gis8.. <a fis>32 <gis e>8.. <fis d!>32 }

lone = \relative c, {  cis4 \f (
\once \override DynamicText #'extra-offset = #'(-0.5 . 4)
d' \sf cis8.. b32
\once \override NoteColumn #'force-hshift = #1.5
<a>8.. gis32 fis4 eis)
fis8..( e!32 d4 e
fis8.. gis32 ais4 b8.. cis32
d8.. dis32 e8-.) r \oneVoice e,4
\stemDown
\override Staff.SustainPedalLineSpanner #'Y-extent = #'(0 . 0)
\override Staff.SustainPedalLineSpanner #'staff-padding = #'10
<a a,>8.. (\sustainOn <b b,>32 <cis cis,>8 ) \sustainOff <d d,>-. <e e,>-. <fis fis,>-.
\stemNeutral <gis gis,>8-. <a a,>-. <a b,>-. <a cis,>-. <a d,>-. <gis e>-.
\override Staff.SustainPedalLineSpanner #'staff-padding = #'6
<a a,>8.. ( [ \sustainOn <b b,>32 <cis cis,>8 ) \sustainOff <d d,>-. <e e,>-. <fis fis,>-. ] \clef treble
<gis gis,>-. <a a,>-. <a b,>-. <a cis,>-. <a d,>-. <gis e>-.
<a a,>8.. ( [ \sustainOn <b b,>32 <cis cis,>8 ) e,( \sustainOff <a a,>) e( ]
<a a,>8-. ) \sustainOn r \clef bass <e,, a,>4. r8 \sustainOff \fermata

}

ltwo = \relative c, { R2. cis4(^\markup{\italic "R."} d'^\sf cis8.. b32
a4) a g8.. d'32
cis4.. fis16~ fis4~
fis4
}

common = { \time 3/4 \key a \major  }


\new PianoStaff \with { instrumentName="Piano" }<<
\new Staff = "RH" { \common << {\rone} \\ {\rtwo} >>  }
\new Staff = "LH" { \common \clef bass << {\ltwo} \\ {\lone} >> }
>>

\paper {  system-count = 2 
line-width=16.5\cm
indent=1\cm
}

\end{lilypond}


\subsection{Selección del idioma de las notas}

La instrucción \verb+\language+ seguida de una cadena específica que
denota el idioma, permite escoger el idioma en que se escriben los
nombres de las notas.  En nuestro caso, la cadena es ``espanol'' con
``n'' en lugar de ``ñ'', como se ve en el ejemplo siguiente:

\begin[verbatim]{lilypond}
\language "espanol"
\new Staff \relative do' { \cadenzaOn
  do8[ dos reb re res mib mi fa fas solb sol sols lab la las sib si do]
}
\end{lilypond}

Los nombres de las notas con sostenido se forman añadiendo
\verb+'s'+ y los bemoles añadiendo \verb+'b'+.  Debemos recordar
que en todo lugar en que aparezca un nombre de nota, ya sea dentro
de una instrucción \verb+\relative+, \verb+\transpose+ o
\verb+\key+, entre otras, debe escribirse en el idoma establecido.


\subsection{Pedal de piano}

Las instrucciones \verb+\sustainOn+ y \verb+\sustainOff+ producen
las marcas clásicas del pedal derecho del piano:

\begin[verbatim]{lilypond}
\new PianoStaff <<
  \new Staff { R1*2 }
  \new Staff { \clef bass c1 ~ \sustainOn c \sustainOff }
>>
\end{lilypond}


\subsection{Crescendo de texto}

Disponemos de instrucciones para tipografiar reguladores de texto,
como puede verse aquí:

\begin[verbatim,relative=1]{lilypond}
c16  \p \cresc c c c c c c c c c c c c c c c
c \f \dim c c c c c c c c c c c c c c c \p
\end{lilypond}


\end{document}

