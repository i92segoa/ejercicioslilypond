
\setcounter{section}{4} %para 05 beethoven


\section{``La llamada del destino'' (Quinta sinfonía de Beethoven)}


\subsection{Modelo}

En este modelo que reproduce el tema del primer movimiento de la
5ª sinfonía de Beethoven, vemos un caso de barrado manual:

\bigskip

\begin[staffsize=17.5,no-ragged-right]{lilypond}
\version "2.11.63"

\relative c''{
 \key c \minor
 \time 2/4
 r8 g8[ g g]
 ees2 \fermata
 r8 f8[  f f]
 d2 ~
 d \fermata
}
\end{lilypond}


\subsection{Número de la versión}

Los archivos de entrada de LilyPond siguen una sintaxis estricta.
Los desarrolladores del programa LilyPond tratan de mantener lo
más estable posible esta sintaxis, pero de vez en cuando se
producen cambios que hacen incompatibles los archivos de entrada
antiguos con las versiones de LilyPond recientes.  Existe un
programa convertidor que no usaremos aún, pero que requiere que
especifiquemos el número de la versión del programa para la que se
escribió la partitura; de esa forma, será posible convertir
automáticamente los archivos para actualizarlos.  El número de la
versión debe escribirse siempre al principio del texto, en la
forma \verb+\version "2.12.0"+, donde aparece entrecomillado el
número de la versión actual del programa.

Si no especificamos ningún número de versión, el programa
registrará una advertencia en el archivo de salida \verb+.log+.


\begin{verbatim}GNU LilyPond 2.13.5
Procesando «05-barras-beethoven-5thsym.ly»
Analizando...
05-barras-beethoven-5thsym.ly:0:
warning: no se ha encontrado ninguna instrucción \version, escriba

\version "2.13.5"

para disponer de compatibilidad en el futuro
\end{verbatim}

\subsection{Barrado manual}
Las barras de corchea, semicorchea y figuras de menor duración se
imprimen automáticamente; sin embargo, en ciertos casos debemos
especificarlas manualmente, por ejemplo en el siguiente caso:

\begin[verbatim,relative=2,staffsize=13]{lilypond}
\time 2/4
r8 g a b
c r r4
\end{lilypond}


Si queremos que las tres primeras corcheas estén unidas mediante
una barra, marcamos la primera con un corchete recto de apertura
'\verb+[+' y la última con un corchete recto de cierre '\verb+]+',
de la siguiente forma:

\begin[verbatim,relative=2,staffsize=13]{lilypond}
\time 2/4
r8 g[ a b]
c r r4
\end{lilypond}

Es importante observar que los corchetes \textbf{no encierran
  conjuntos de notas}, sino que marcan las notas primera y última
de una barra colocándose cada uno \textbf{detrás} de la nota
correspondiente.

