\section{Márgenes. Frère Jacques}


\subsection{Modelo}

Es fácil establecer los márgenes de la partitura y el número de
sistemas, así como otros retoques de aspecto final. En este ejemplo
hemos establecido unos márgenes izquierdo y derecho de 3 cm., (sin
sangrado en la primera línea), un margen superior de 4 cm. y un margen
inferior de 10 cm.  Hemos aumentado el tamaño global de la partitura a
26 puntos y hemos distribuido los compases en cuatro sistemas que
ocupen toda la página hasta el margen inferior.  Se han suprimido la
línea informativa final y los números de compás.  Hemos marcado las
entradas del canon con números en negrita.  Por último, añadimos
indicaciones de verso para señalar el idioma de cada uno de los
textos.

\bigskip
\begin{center}
\fbox{\includegraphics[width=100mm]{frerejacques}}
\end{center}
\bigskip


% Aumentar la separación entre sistemas
%\def\betweenLilyPondSystem#1{\vspace{0.2cm}\linebreak}

\subsection{Márgenes.  Sangrado de la primera línea}

Es bastante amplio el abanico de variables que se pueden ajustar
dentro del bloque \verb+\layout+ para controlar la forma en que se
imprime la partitura.  Cada bloque \verb+\layout+ dentro de una
partitura controla los ajustes para esa partitura; para ajustar todas
las partituras de un documento al mismo tiempo, se usa de la misma
forma el bloque \verb+\paper+ fuera de los bloques \verb+\score+.

La variable que controla el sangrado de la primera línea es
\verb+indent+, y aquellas que controlan la medida de los márgenes son
\verb+left-margin+, \verb+right-margin+, \verb+top-margin+ y
\verb+bottom-margin+, para los márgenes izquierdo, derecho, superior e
inferior, respectivamente.

Para expresar las medidas en milímetros o centímetros, se utiliza
\verb+\mm+ ó \verb+\cm+ después del valor; no es necesario si el valor
es cero.  Por ejemplo, aquí hemos suprimido el sangrado de la primera
línea y establecido el margen izquierdo a 2 centímetros:

\begin{verbatim}
\paper {
  indent = 0
  left-margin = 2\cm
}
\end{verbatim}


\subsection{Número de sistemas}

Aunque es posible insertar saltos de línea en los lugares deseados
mediante la instrucción \verb+\break+, puede ser más cómodo
especificar simplemente cuántos sistemas queremos.  Lo hacemos dando
un valor a la variable \verb+system-count+ en el bloque \verb+\layout+
o \verb+\paper+.


\subsection{Ocupación de la página}

Si se da un valor falso a la variable \verb+ragged-last-bottom+, el
último sistema llegará hasta el final de la última página y la
partitura ocupará todo el papel hasta el margen inferior.

\begin{verbatim}
\paper {
  ragged-last-bottom=##f
}
\end{verbatim}


\subsection{Quitar los números de compás}

Para suprimir los números de compás, se quita el grabador
\verb+Bar_number_engraver+ del contexto \verb+\Score+, véase el
apartado \ref{consists} de la página \pageref{consists}.


\subsection{Negrita}

Dentro de un elemento de marcado, la instrucción \verb+\bold+ establece un estilo negrita:

\begin[verbatim,fragment]{lilypond}
c'^\markup{ Normal, \bold Negrita. }
\end{lilypond}


\subsection{Verso de la letra}

La propiedad de contexto ``\verb+stanza+'' fija un texto que aparece antes
de la primera sílaba de la letra.  Se puede establecer con \verb+\set+
dentro de la expresión que contiene la letra, o dentro de un bloque
\verb+\with+ al crear el contexto \verb+\Lyrics+.

\begin[verbatim,fragment]{lilypond}
\new Staff { c'1 }
\addlyrics { \set stanza = "1." Dooo }
\end{lilypond}


\subsection{Notas}

\begin{itemize}
\item Consulte cómo cambiar el tamaño global de la partitura en el apartado \ref{tamano-global} de la página \pageref{tamano-global}.

\item Suprima la línea informativa final definiendo \verb+tagline=##f+
  dentro del bloque \verb+\header+.

\item Pruebe cómo suena el canon con este bloque de ejemplo: \footnotesize
\begin{verbatim}
\score { << \new Staff \with { midiInstrument="trumpet" }{ \tempo 4=150 \canon }
            \new Staff \with { midiInstrument="muted trumpet" }{  R1*2 \canon  }
            \new Staff \with { midiInstrument="accordion" }{ R1*4 \canon }
            \new Staff \with { midiInstrument="choir aahs" }{  R1*6 \canon  }
	>> \midi{} \layout{} }
\end{verbatim}
\normalsize
\end{itemize}

