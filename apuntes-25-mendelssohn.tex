% \version "2.17.0"

\section{Instrumentos transpositores. Cambio de pentagrama}


\subsection{Modelo}

La música para instrumentos transpositores requiere particellas
preparadas para ellos con la transposición inversa, y las partes en
tono de concierto (p.ej. para el piano) no llevan esta transposición.
Nuestro ejemplo de hoy procede de la segunda pieza de concierto para
dos clarinetes y piano, de Mendelssohn.  Podemos transcribir la música
de los clarinetes a partir de las particellas o a partir de la
partitura de conjunto; en cualquier caso, el MIDI de la partitura de
conjunto debe sonar a la altura correcta.

Debemos producir, a partir de la misma fuente, un documento con la
partitura de piano en tono de concierto, y en páginas o en documentos
separados las particellas de los clarinetes preparadas para un
instrumento en Si$\flat$.

La parte de piano necesita un cambio de pentagrama en el tercer
compás.

\begin{lilypond}
clarinete = \relative c''' { \tempo "Presto"
	\key e \minor \time 2/2  \partial 4. c8( \ff b) ais-.
b4.-> fis8-. g4.-> dis8-.
e4-. e,-. r8 e'8( g) fis-. e( d) c-. b-. a-. g-. fis-. e-.
c'4-. fis,-.
}

clarineteUno = { \transposition bes \clarinete }

clarineteDos = \transpose c c, { \transposition bes \clarinete }
pianoMD = \relative c'' { %\tempo "Presto"
	\key d \minor \time 2/2 \partial 4. bes8( \ff  a) gis-.
	a4.-> e8 f4.-> cis8
	d4-. <f, a>-. r8 d'( f) e-.
	d8( c) bes-. a-. \change Staff = "abajo" \voiceOne g-. f-. e-. d-.
	\change Staff ="arriba" <bes' d e>4-. q-.
	}
pianoMI = \relative c { \key d \minor \clef bass \time 2/2 \partial 4. bes8( a) gis-.
	a4.-> <e e'>8 <f f'>4.-> <cis cis'>8
	<d d'>4-> q-> r8 d'( f) e-.
	d8-. c-. bes-. a-.  \voiceTwo g-. f-. e-. d-.
	g4-. g-.
}


\score {
<< 	\new Staff \with { fontSize = #-3 
		\override StaffSymbol #'staff-space = #(magstep -3)
		instrumentName="Clarinete I" 
	%	\override StaffSymbol #'thickness = #(magstep -3)
	} \transpose c bes, { \clarineteUno }
	\new Staff \with { fontSize = #-3 
		\override StaffSymbol #'staff-space = #(magstep -3)
		instrumentName="Clarinete II"
	%	\override StaffSymbol #'thickness = #(magstep -3)
	} \transpose c bes, { \clarineteDos }
	\new PianoStaff \with { instrumentName="Piano" } <<
		\new Staff ="arriba" { \pianoMD }
		\new Staff ="abajo" { \pianoMI }
	>>
>>
\layout{}
\midi{}
}


\score {
	\new Staff \with { instrumentName="Clarinete I" } { \clarineteUno }
\layout{}
\midi{}
}

\score {
	\new Staff \with { instrumentName="Clarinete II" } { \clarineteDos }
\layout{}
\midi{}
}

\end{lilypond}

% Aumentar la separación entre sistemas
%\def\betweenLilyPondSystem#1{\vspace{0.2cm}\linebreak}

\subsection{Cambios de pentagrama}

Podemos cambiar la música de un pentagrama a otro si en la creación de
los contextos hemos dado un nombre a cada uno.  Después, la
instrucción \verb+\change+ nos permite especificar el contexto al que
deseamos cambiar, llamándolo por su nombre.  Por ejemplo:

\begin[verbatim]{lilypond}
musica = \relative c''{ g16 e c \change Staff = "abajo" g c,4 }
<< \new Staff = "arriba" { \musica }
   \new Staff = "abajo"   { \clef bass s2 }
>>
\end{lilypond}

El contexto que recibe la música debe existir mientras duran las notas
cambiadas, lo que podemos hacer con silencios de separación como puede
verse en el ejemplo.  Después del cambio, puede ser necesario orientar
las plicas mediante una instrucción del tipo \verb+\voiceOne+.


\subsection{Instrumentos transpositores}

Para fijar el intervalo de transposición de un instrumento, se usa la
instrucción \verb+\transposition+ seguida de una nota en altura
absoluta.  El intervalo que forma la nota con el Do central, \verb+c'+,
corresponde al intervalo transpositor del instrumento.  Por ejemplo,
si un instrumento como el clarinete en Si$\flat$ transporta un tono
hacia abajo, debemos poner lo siguiente:

\begin{verbatim}
\transposition bes
\end{verbatim}

A continuación podemos escribir la música tal y como aparece en su
particella.

\begin[verbatim]{lilypond}
\relative c''' { \transposition bes
  \key e \minor
  \time 2/2
  \partial 4. c8( \ff b) ais-. }

\end{lilypond}



Más tarde podemos imprimir la partitura en tono de concierto aplicando
a esta música el transporte del instrumento, en este caso un tono
hacia abajo:

\begin[verbatim]{lilypond}
clarinete = \relative c''' { \transposition bes
  \key e \minor
  \time 2/2
  \partial 4. c8( \ff b) ais-. }

\transpose c bes, { \clarinete }
\end{lilypond}


Si no nos preocupa la producción de MIDI, para producir una
particella con destino a un instrumento transpositor a partir de
música en tono de concierto, sólo hay que aplicar el transporte
inverso en el momento de imprimir la particella.

\subsection{Notas}
\begin{itemize}
\item El nombre del sistema de piano puede establecerse con una de
  las dos instrucciones siguientes:

\begin{tabular}{l     r}
\verb+\set PianoStaff.instrumentName = "Piano"+ & (dentro de la música) \\
\verb+\new PianoStaff \with{ instrumentName = "Piano" }+ & (al crear el contexto)
\end{tabular}

\medskip

Véase la sección \ref{instrumentname}
(pág. \pageref{instrumentname}).  Vimos también el uso de
\verb+\with+ en el apartado \ref{with} (pág. \pageref{with}).

\item La literatura en inglés se refiere a los cambios de
  pentagrama como \emph{cross-staff beams}, es decir, barras de
  pentagrama cruzado.

\item Algunas ediciones de música para instrumentos transpositores
  imprimen las partes de éstos en la partitura general con el
  transporte aplicado, igual que en las particellas, lo que puede
  ser confuso para el pianista (pues no son los sonidos reales)
  pero tiene la ventaja de que los nombres de las notas son los
  mismos en todos los papeles.
\end{itemize}
