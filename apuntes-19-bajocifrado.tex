% \version "2.17.0"

\section{Bajo cifrado. \emph{Polonaise}, de Bach.}


\subsection{Modelo}

Para realizar el siguiente modelo será necesario aprender a introducir
e imprimir bajos cifrados y a elaborar una estructura de contextos
anidados.  Procede de la \emph{Polonesa} de la \emph{Suite orquestal
  núm.2 en Si menor, BWV 1067}.  Es recomendable declarar la
repetición en cada pentagrama, para poder reutilizar los materiales en
las particellas.


\bigskip

%\hspace{3cm}
\begin[staffsize=15,
line-width=17\cm,
indent=2\cm
]{lilypond}

% canción del emperador. Narváez

\header{ title = "Polonaise" }

flautoTraverso = \relative c''' {
		\time 3/4
		\key b \minor
		\repeat volta 2 {
			b8. d16 cis8 b \appoggiatura b8 ais8.( b32 cis)
			b8. d16 cis8-. b-. cis16( b ais) cis-.
			b8. d16 cis8-. b-. a!-. g-.
			fis \trill e16 fis d2
		}
	}

violinoI = \relative c'' {
		\time 3/4
		\key b \minor
		\repeat volta 2 {
			b8. d16 cis8 b \appoggiatura b8 ais8.( b32 cis)
			b8. d16 cis8-. b-. cis16( b ais) cis-.
			b8. d16 cis8-. b-. a!-. g-.
			fis \trill e16 fis d2
		}
	}

violinoII = \relative c' {
		\time 3/4
		\key b \minor
		\repeat volta 2 {
			fis8-. d-. g-. b,-. cis-. fis-.
			fis8 b ais b g fis
			fis b g fis e e
			d8\trill cis16 d a2
		}
	}

viola = \relative c' {
		\time 3/4
		\key b \minor
		\clef alto
		\repeat volta 2 {
			b8-. fis'-. e-. e-. fis-. cis-.
			d8-. fis-. e-. d-. e-. cis-.
			d8 fis e fis16 g a8 cis,
			a8 g16 a fis2
		}
}

continuo = \relative c {
		\time 3/4
		\key b \minor
		\clef bass
		\repeat volta 2 {
			d8 b e g fis e
			d8^\markup{ \italic piano } b fis' g e fis
			d8^\markup{ \italic forte } b e d cis a
			d4~ d8 a16 fis d4
		}
	}

bajoCifrado = \figuremode {
		<6>4 <6>8 <6> <_+>4
		<6>4 <_+>8 <5> <6 5> <_+>
		<6>4 <6>8 <6> <6> <7>
		}

\new StaffGroup <<
	\new Staff { \tempo "Moderato e staccato"
          \set Staff.instrumentName = "Flauto traverso" \flautoTraverso }
	\new PianoStaff <<
			  \new Staff { \set Staff.instrumentName = "Violino I" \violinoI }
			  \new Staff { \set Staff.instrumentName = "Violino II" \violinoII }
			>>
	\new Staff { \set Staff.instrumentName = "Viola" \viola }
	\new Staff { \set Staff.instrumentName = "Continuo" \continuo }
	\new FiguredBass { \bajoCifrado }
>>

\end{lilypond}


\subsection{Bajo cifrado}

La escritura de bajos cifrados es muy sencilla: basta inaugurar un
modo especial \verb+\figuremode+ para que la expresión se interprete
adecuadamente como cifras.  En este modo, introducimos las cifras
dentro de ángulos simples y las duraciones después del ángulo de
cierre, como si fueran acordes normales.

\verb+cifras = \figuremode{ <6>2 <6 5>4 }+

Después, imprimimos este material dentro de un contexto \verb+FiguredBass+:

\begin[verbatim,staffsize=17.5]{lilypond}
cifras = \figuremode{ <6>2 <6 5>4 }
<<
  \new Staff { \clef bass c4 d e }
  \new FiguredBass { \cifras }
>>
\end{lilypond}

Usamos un signo más \verb'+' para el sostenido, y un signo menos
\verb'-' para el bemol, escritos después de la cifra.  Si la
alteración no lleva ninguna cifra, escribimos un guión bajo y a
continuación la alteración.

\begin[verbatim,staffsize=17.5]{lilypond}
\new FiguredBass \figuremode{ <3->4 <_+> }
\end{lilypond}


\subsection{Contextos anidados}

Los contextos que agrupan pentagramas y que trazan llaves o corchetes,
pueden formar grupos secundarios dentro del grupo general.

Por ejemplo, si queremos agrupar mediante una llave dos pentagramas
dentro de otro grupo de pentagramas con corchete recto, elaborado con
\verb+\new StaffGroup+, lo hacemos abriendo un grupo \verb+PianoStaff+
en el lugar correspondiente, sin olvidar los ángulos dobles:


\begin[verbatim,staffsize=17.5]{lilypond}
\new StaffGroup <<
  \new PianoStaff <<
    \new Staff { s1 }
    \new Staff { s }
  >>
  \new Staff { s }
  \new Staff { s }
>>
\end{lilypond}
