\section{Varias partituras en un documento: libros.  Marcados de alto nivel}


\subsection{Modelo}

Cuando se quiere elaborar un documento de varias páginas que contiene
distintas piezas (p.ej. los movimientos de una sonata) o distintas
obras completas que deben ir en páginas separadas, se hace necesario
utilizar los conceptos de libro, parte y pieza.

Para este ejercicio debemos elegir \emph{ocho} piezas ya realizadas
que se insertarán en el texto del documento.  Produciremos, a partir
de un solo documento de LilyPond, dos archivos de salida PDF o
\emph{libros}, cada uno de los cuales tendrá dos partes en páginas
separadas, y cada parte de cada libro contendrá dos partituras con
especificación de la pieza en la cabecera correspondiente.

\bigskip
\parindent=0mm
\fbox{\includegraphics[width=39mm]{outA1}}
\fbox{\includegraphics[width=39mm]{outA2}}
\fbox{\includegraphics[width=39mm]{outB1}}
\fbox{\includegraphics[width=39mm]{outB2}}
\parindent=6mm
\bigskip

Se deberá tener cuidado en que los nombres de variable,
procedentes de distintas piezas, no se confundan al incluirlos en
el mismo texto.

% Aumentar la separación entre sistemas
%\def\betweenLilyPondSystem#1{\vspace{0.2cm}\linebreak}

\subsection{Bloques de partitura: distintas piezas}

Si en un documento aparece más de un bloque \verb+\score+, una pieza
aparecerá a continuación de la otra, en la misma página. Sólo se
imprimen, para cada una, las cabeceras \verb+piece+ y \verb+opus+
(véase el apartado \ref{pieceopus}, pág. \pageref{pieceopus}). Se
pueden volver a definir los títulos de cabecera para cada partitura.

Para imprimir el título de cada partitura, se requiere un bloque
\verb+\paper+ (fuera de todas las partituras) que contenga el
siguiente ajuste:

\begin{verbatim}
\paper{
  print-all-headers = ##t
}
\end{verbatim}

En nuestro ejemplo esto no es necesario, pero debe tenerse en cuenta
que las cabeceras de cada partitura deben ir \emph{después} de la
música dentro del bloque \verb+\score+, así:

\begin{verbatim}
\score{ \cumple
  \header { piece="Cumpleaños feliz" }
}
\end{verbatim}


\subsection{Elementos de marcado del nivel superior}

Si se escribe un elemento \verb+\markup+ fuera de cualquier partitura,
aparecerá en su lugar correspondiente.  Estos elementos se denominan
``Marcados de alto nivel'' porque se encuentran en lo alto de la
jerarquía de la sintaxis de un documento, es decir, fuera de cualquier
bloque.  Los elementos de marcado del nivel superior se pueden emplear
para escribir las letras de las estrofas de una canción.  También
puede insertarse una separación para distanciar dos partituras entre
sí, o antes de una partitura para separarla del borde superior:

\begin{verbatim}
\score { ... }
\markup{ \vspace #5 }
\score { ... }
\end{verbatim}

La instrucción de marcado \verb+\vspace+ está disponible a partir de
LilyPond versión 2.13.6.

\subsection{Bloques de libro: distintos archivos de salida}

Cuando en un documento aparecen bloques \verb+\book+, se produce un
archivo de salida PDF distinto para cada bloque, con todas las
partituras que contenga.  El primer archivo tiene el mismo nombre del
documento, como es usual, y a partir del segundo libro se añade un
número consecutivo al final del nombre.

Es preciso tener en cuenta la siguiente salvedad: los documentos de
partitura completos, que pueden contener indicaciones de alto nivel
como definiciones de variables, no pueden incluirse mediante la
instrucción \verb+\include+ dentro de un bloque \verb+\book+.  Las
definiciones de variables son instrucciones del nivel sintáctico
superior y por tanto deben ir fuera de cualquier bloque.

\subsection{Bloques de parte: distintas páginas en un libro}

Es sencillo mantener en páginas separadas dos o más conjuntos de
partituras dentro de un libro: tan sólo hay que encerrar cada conjunto
en un bloque \verb+\bookpart+.  Las cabeceras de título de la primera
parte serán las que se impriman al principio del libro.

\subsection{Pie de página}
\label{tagline}

La cabecera \verb+tagline+ contiene las indicaciones que aparecen en
la última página de cada libro.  Si no se especifica una línea de pie,
la línea predeterminada contiene una nota de autopromoción del
programa y la versión del mismo con que se hizo el archivo de salida.

\subsection{Ejemplo}

La siguiente estructura es un ejemplo que resume lo explicado más
arriba.

\begin{verbatim}
musicaUno={...}
musicaDos={...}
...

\book{                          % primer libro
  \bookpart{                      % primera parte
    \header{ title = ... }          % cabeceras de este libro y parte  
    \score{ \músicaUno...             % primera partitura
      \header { piece = ... }         % cabeceras de esta partitura  
    }
    \score{ ... }                   % segunda partitura
  }                               % fin de la parte y salto de página
  \bookpart{ ... }                % segunda parte
}
\book{ ... }                    % segundo libro
\end{verbatim}
