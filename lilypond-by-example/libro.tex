\documentclass[a4paper,10pt,oneside,headinclude,titlepage]{article} % la clase article estándar
\usepackage[spanish]{babel}
\pagestyle{empty}
\usepackage[utf8]{inputenc}
\usepackage[T1]{fontenc}
\usepackage[margin=2cm]{geometry}
\usepackage{graphicx}
\usepackage[utopia]{mathdesign}
\parskip=6pt\clubpenalty=10000\widowpenalty=10000

\newcommand{\preLilyPondExample}{\vspace{-10pt}}
\newcommand{\lpversion}{2.13.4}
\newcommand{\defsep}{\textbf{$\|$}}
\newcommand{\software}{\emph{software}}
\newcommand{\negspace}{\vspace{-10pt}}  %{\vspace{-20pt}}
\newcommand{\seppar}{
\bigskip
%\vspace{6pt}
}

\title{LilyPond por ejemplo: la guía visual}
\author{Francisco Vila}
\date{\today}

%%%%%%%%%%%%%%%%%%%%%%%%%%%%%%%%%%%%%%%%%%%%%%%%%%%%%%%%%%%%%
\begin{document}

\nonfrenchspacing

\begin{titlepage} %%%%%%%%%%%% PORTADA
  \makeatletter
  \begin{center}
    %\textbf{\Large Universidad de Extremadura}\par
    \vfill
    \includegraphics[width=60mm]{lily-logo.png}\par
    \vfill
    \textbf{\huge\@title}\par
%    \textbf{\large\@subtitle}\par
    {\@date}
    \vfill
    \textbf{\large\@author}
    \vfill
  \end{center}
  \makeatother
\end{titlepage}

%\singlespace

%\begin{singlespace} % los listados van a un espacio. Necesita el paquete setspace
 % \tableofcontents
 % \listoffigures
 % \listoftables
%\end{singlespace}

% chuleta de texto sencillo
%Las entradas con asterisco son complementarias

% LilyPond by example -- a visual guide
% text-only cheatsheet

% *-entries are complementary

\section*{Una nota: la}
\begin[verbatim,fragment]{lilypond}
    a
\end{lilypond}

\section*{Un silencio}
\begin[verbatim,fragment]{lilypond}
    r
\end{lilypond}

\section*{*Silencio de compás completo: 4/4}
\begin[verbatim,fragment]{lilypond}
   R1
\end{lilypond}

\section*{*Silencio de compás completo: 3/4}

\begin{verbatim}
   R2.
\end{verbatim}

\begin[fragment]{lilypond}
   \time 3/4 R2.
\end{lilypond}

\section*{Una expresión musical}
\begin[verbatim,fragment]{lilypond}
    { a }
\end{lilypond}

\section*{Una octava más aguda}
\begin[verbatim,fragment]{lilypond}
    a'
\end{lilypond}

\section*{Una octava más grave}
\begin[verbatim,fragment]{lilypond}
    a,
\end{lilypond}

\section*{*Una escala de do a si}
\begin[verbatim,fragment]{lilypond}
    c d e f g a b
\end{lilypond}

\section*{*Alturas en modo absoluto: una escala de cuatro octavas}
\begin[verbatim,fragment]{lilypond}
  {
  \clef "bass"
  c,4 d, e, f,
  g,4 a, b, c 
  d4 e f g
  a4 b c' d'
  \clef "treble"
  e'4 f' g' a'
  b'4 c'' d'' e''
  f''4 g'' a'' b''
  c'''1
}
\end{lilypond}

\section*{Alturas en modo relativo: una escala}
\begin[verbatim,fragment]{lilypond}
    \relative c' { c d e f g a b c }
\end{lilypond}

\section*{Alteración: do sostenido}
\begin[verbatim,fragment,relative=2]{lilypond}
    cis
\end{lilypond}

\section*{Alteración cuando hay armadura: si bemol en fa mayor}
\begin[verbatim,fragment,relative=2]{lilypond}
    \key f \major
    bes
\end{lilypond}

\section*{Alteración: la bemol}
\begin[verbatim,fragment,relative=2]{lilypond}
    aes
\end{lilypond}

\section*{*Alteración forzada: becuadro}
\begin[verbatim,fragment,relative=2,notime]{lilypond}
   a!
\end{lilypond}

\section*{*Alteración de cortesía}
\begin[verbatim,fragment,relative=2,notime]{lilypond}
   aes?
\end{lilypond}

\section*{*Becuadro de cortesía}

\begin[verbatim,fragment,relative=2,notime]{lilypond}
   a?
\end{lilypond}

\section*{Efecto de las alteraciones: la bemol seguido de la natural}

\begin[verbatim,fragment,relative=2,notime]{lilypond}
    aes a
\end{lilypond}

\section*{*Escala cromática ascendente con sostenidos}
\begin[verbatim,fragment,relative=1,notime]{lilypond}
    c cis d dis e f fis g gis a ais b
\end{lilypond}

\section*{*Escala cromática descendente con bemoles}
\begin[verbatim,fragment,relative=2,notime]{lilypond}
    c b bes a aes g ges f e es d des
\end{lilypond}

\section*{Una corchea}
\begin[verbatim,fragment,relative=2,notime]{lilypond}
    a8
\end{lilypond}

\section*{Varias corcheas}
\begin[verbatim,fragment,relative=2,notime]{lilypond}
    a8 a a a 
\end{lilypond}

\section*{*Todos los valores, de redonda a semifusa}
\begin[verbatim,fragment,relative=2,notime]{lilypond}
    a1 a2 a4 a8 a16 a32 a64
\end{lilypond}

\section*{Negra con puntillo}
\begin[verbatim,fragment,relative=2,notime]{lilypond}
    a4.
\end{lilypond}

\section*{*Doble puntillo}
\begin[verbatim,fragment,relative=2,notime]{lilypond}
    a4..
\end{lilypond}

\section*{Dos notas unidas por ligadura}
\begin[verbatim,fragment,relative=2,notime]{lilypond}
    a ~ a
\end{lilypond}

\section*{Un acorde}
\begin[verbatim,fragment,relative=1,notime]{lilypond}
    <c e g>
\end{lilypond}

\section*{Duración de un acorde}
\begin[verbatim,fragment,relative=1,notime]{lilypond}
    <c e g>8
\end{lilypond}

\section*{Ligadura de expresión: dos notas}
\begin[verbatim,fragment,relative=2,notime]{lilypond}
    a( b)
\end{lilypond}

\section*{*Ligadura de expresión: varias notas}
\begin[verbatim,fragment,relative=2,notime]{lilypond}
    c( d e f g a)
\end{lilypond}

\section*{Ligadura de fraseo}
\begin[verbatim,fragment,relative=2,notime]{lilypond}
    c\( d e f g a\)
\end{lilypond}

\section*{Asignación de una expresión musical a una variable}
\begin{verbatim}
    violin = { a b c }
\end{verbatim}

\section*{Uso de una variable}
\begin{verbatim}
    \violin
\end{verbatim}

\begin{lilypond}
    violin = \relative c''{ a b c }
    { \violin }
\end{lilypond}

\section*{Tresillo de corcheas}
\begin[verbatim,fragment,relative=2,notime]{lilypond}
    \times 2/3 { a8 a a }
\end{lilypond}

\section*{Clave de Sol en 2ª}
\begin[verbatim,fragment,notime]{lilypond}
    \clef treble
\end{lilypond}

\section*{Clave de Fa en 4ª}
\begin[verbatim,fragment,notime]{lilypond}
    \clef bass
\end{lilypond}

\section*{Tonalidad: Re mayor}
\begin{verbatim}
    \key d \major
\end{verbatim}
\begin[fragment]{lilypond}
  \hide Staff.TimeSignature
  \key d \major s1
\end{lilypond}

\section*{Compás: 3/4}
\begin{verbatim}
    \time 3/4
\end{verbatim}

\begin[fragment]{lilypond}
    \time 3/4 s2.
\end{lilypond}

\section*{Tempo}
\begin[verbatim,fragment]{lilypond}
    \tempo "Allegro"
\end{lilypond}

\section*{Indicación metronómica}
\begin[verbatim,fragment]{lilypond}
    \tempo 4=60
\end{lilypond}

\section*{Articulación: picado}
\begin[verbatim,fragment]{lilypond}
    a-.
\end{lilypond}

\section*{*Otras articulaciones}
\begin[verbatim,fragment]{lilypond}
    c4-^ c-+ c-- c-| c4-> c-. c2-_
\end{lilypond}

\section*{Digitaciones}
\begin[verbatim,fragment]{lilypond}
    c-1 e-3 g-5
\end{lilypond}

\section*{Forzar articulación por encima}
\begin[verbatim,fragment]{lilypond}
    a^.
\end{lilypond}

\section*{Forzar articulación por debajo}
\begin[verbatim,fragment]{lilypond}
    a_.
\end{lilypond}

\section*{Arco arriba, arco abajo}
\begin[verbatim,fragment]{lilypond}
    a\upbow a\downbow
\end{lilypond}

\section*{Nota con calderón}
\begin[verbatim,fragment]{lilypond}
    a \fermata
\end{lilypond}

\section*{Matiz dinámico}
\begin{verbatim}
    \pp
\end{verbatim}

\begin[fragment]{lilypond}
    s1 \pp
\end{lilypond}

\section*{*Otros matices}
\begin{verbatim}
    \p \mf \f \ff
\end{verbatim}

\begin[fragment]{lilypond}
    s1 \p s1 \mf s1 \f s1 \ff
\end{lilypond}

\section*{Regulador (crescendo), fin de regulador}
\begin[verbatim,fragment]{lilypond}
    a\< b\!
\end{lilypond}

\section*{Regulador (diminuendo), fin de regulador}
\begin[verbatim,fragment]{lilypond}
    a\> b\!
\end{lilypond}

\section*{Final implícito de un regulador}
\begin[verbatim,fragment]{lilypond}
    a\p a\< a\f
\end{lilypond}

\section*{Texto sobre una nota, sin formato}
\begin[verbatim,fragment]{lilypond}
    a-"dolce"
\end{lilypond}

\section*{Texto sobre una nota, con formato (elemento de marcado)}
\begin[verbatim,fragment]{lilypond}
    a-\markup{ \italic dolce }
\end{lilypond}

\section*{Unir corcheas manualmente con una barra}
\begin[verbatim,fragment]{lilypond}
    a8[ a a a a a a a a]
\end{lilypond}

\section*{Compás de anacrusa: valor de corchea}
\begin[verbatim,fragment]{lilypond}
    \partial 8
    a8
\end{lilypond}

\section*{Notas de adorno (genéricas)}
\begin[verbatim,fragment]{lilypond}
    \grace{ g16 b }
    a1
\end{lilypond}

\section*{Notas de adorno: apoyatura}
\begin[verbatim,fragment]{lilypond}
    \appoggiatura c4
    d2
\end{lilypond}

\section*{Notas de adorno: mordente de una nota}
\begin[verbatim,fragment]{lilypond}
    \acciaccatura c16
    d2
\end{lilypond}

\section*{Dos expresiones en secuencia}
\begin[verbatim,fragment]{lilypond}
    {{ a b c } { d e f }}
\end{lilypond}

\section*{Dos expresiones simultáneas}
\begin[verbatim,fragment]{lilypond}
    << { a b c } { d e f } >>
\end{lilypond}

\section*{Dos variables en secuencia}
\begin{verbatim}
    { \parteUno \parteDos }
\end{verbatim}

\begin{lilypond}
parteUno = { c' }
parteDos = { e' }
    { \parteUno \parteDos }
\end{lilypond}
    
\section*{Dos variables simultáneas}
\begin{verbatim}
    << \parteUno \parteDos >>
\end{verbatim}

\begin{lilypond}
parteUno = { c' }
parteDos = { e' }
    << \parteUno \parteDos >>
\end{lilypond}

\section*{Un contexto de partitura explícito}
\begin[verbatim]{lilypond}
    \score {
            { a b c }
     }
\end{lilypond}

\section*{Un contexto de pentagrama explícito}
\begin[verbatim,fragment]{lilypond}
    \new Staff { a b c }
\end{lilypond}

\section*{Dos expresiones simultáneas dentro de un pentagrama explícito}
\begin[verbatim,fragment]{lilypond}
   \new Staff << { a b c } { d e f } >>
\end{lilypond}

\section*{Dos voces independientes dentro de un pentagrama explícito}
\begin[verbatim,fragment]{lilypond}
   \new Staff << { a b c } \\ { d e f } >>
\end{lilypond}

\section*{Un contexto de voz explícito}
\begin[verbatim,fragment]{lilypond}
   \new Voice { a b c }
\end{lilypond}

\section*{Dos voces explícitas dentro de un pentagrama}
\begin[verbatim,fragment]{lilypond}
   \new Staff <<
      \new Voice { a b c }
      \new Voice { d e f }
   >>
\end{lilypond}

\section*{Tres voces explícitas dentro de un pentagrama, con control de las plicas}
\begin[verbatim,fragment]{lilypond}
   \new Staff <<
      \new Voice { \voiceOne   a b c }
      \new Voice { \voiceTwo   d e f }
      \new Voice { \voiceThree g a b }
   >>
\end{lilypond}

\section*{Dos pentagramas simultáneos}
\begin{verbatim}
   << \new Staff { \parteUno }
      \new Staff { \parteDos } >>
\end{verbatim}

\begin{lilypond}
parteUno = { c' }
parteDos = { e' }
   << \new Staff { \parteUno }
      \new Staff { \parteDos } >>
\end{lilypond}

\section*{Sistema de piano: claves de Sol y Fa}
\begin{verbatim}
    \new PianoStaff <<
        \new Staff { ... }
	\new Staff { \clef bass ... }
    >>
\end{verbatim}

\begin{lilypond}
    \new PianoStaff <<
        \new Staff { s1 }
	\new Staff { \clef bass s1 }
    >>
\end{lilypond}

\section*{Sistema de coro: SATB}
\begin{verbatim}
    \new ChoirStaff <<
        \new Staff { \soprano }
	\new Staff { \alto }
	\new Staff { \clef "G_8" \tenor }
	\new Staff { \clef bass \bajo }
    >>
\end{verbatim}

\begin{lilypond}
soprano = { s }
alto = { s }
tenor = { s }
bajo = { s }
    \new ChoirStaff <<
        \new Staff { \soprano }
	\new Staff { \alto }
	\new Staff { \clef "G_8" \tenor }
	\new Staff { \clef bass \bajo }
    >>
\end{lilypond}

\section*{Una breve canción}
\begin[verbatim,fragment]{lilypond}
    { a } \addlyrics{ Aaaah }
\end{lilypond}

\section*{Una canción con dos letras}
\begin[verbatim,fragment]{lilypond}
    { a }
      \addlyrics { Aaaah }
      \addlyrics { Eeeeh }
\end{lilypond}

\section*{Asignar letra a una variable}
\begin{verbatim}
    letraTenor = \lyricmode { Aaaah }
\end{verbatim}

\section*{Uso de una variable de letra}
\begin{verbatim}
    { a } \addlyrics { \letraTenor }
\end{verbatim}

\begin{lilypond}
    letraTenor = \lyricmode { Aaaah }
    { a } \addlyrics { \letraTenor }
\end{lilypond}

\section*{Una canción que usa variables}
\begin[verbatim]{lilypond}
    tenor = { a }
    letraTenor = \lyricmode { Aaaah }
    { \tenor } \addlyrics { \letraTenor }
\end{lilypond}
    
\section*{Letra: separar sílabas}
\begin[verbatim,fragment]{lilypond}
    { a b c }
       \addlyrics{ Ky -- ri -- e }
\end{lilypond}

\section*{Letra: melisma con ligadura de expresión}
\begin[verbatim,fragment]{lilypond}
    { a( b c) d e }
       \addlyrics { Ky -- ri -- e }
\end{lilypond}

\section*{Letra: melisma sin ligadura de expresión}
\begin[verbatim,fragment]{lilypond}
    { a b c d e }
       \addlyrics { Ky -- _ _ ri -- e }
\end{lilypond}

\section*{Letra: línea extensora sobre la última sílaba}
\begin[verbatim,fragment]{lilypond}
    { a( b c) }
       \addlyrics { son __ }
\end{lilypond}

\section*{Letra: más de una sílaba por nota, con ligadura de letra}
\begin[verbatim,fragment]{lilypond}
    { a } \addlyrics { e_e }
\end{lilypond}

\section*{Letra: más de una sílaba por nota, sin ligadura de letra}
\begin[verbatim,fragment]{lilypond}
    { a } \addlyrics { "e e" }
\end{lilypond}

\section*{Un contexto de letra explícito}
\begin{verbatim}
    \new Lyrics { \letraTenor }
\end{verbatim}

\begin{lilypond}
letraTenor = \lyricmode { Ah }
    \new Lyrics { \letraTenor }
\end{lilypond}

\section*{Una canción que usa contextos con nombre}
\begin{verbatim}
    << \new Voice = "vozUno" { \musicaUno }
       \new Lyrics \lyricsto "vozUno" { \letraUno }
    >>
\end{verbatim}

\begin{lilypond}
musicaUno = { c' }
letraUno = \lyricmode { Ah }
    << \new Voice = "vozUno" { \musicaUno }
       \new Lyrics \lyricsto "vozUno" { \letraUno }
    >>
\end{lilypond}

\section*{Título de la obra}
\begin[verbatim]{lilypond}
    \header {
      title = "Sinfonía"
    }
\end{lilypond}

\section*{Título y autor}
\begin[verbatim]{lilypond}
    \header {
      title = "Sinfonía"
      composer = "Beethoven"
    }
\end{lilypond}

\end{document}
