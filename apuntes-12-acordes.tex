\section{Esta noche es Nochebuena (II). Acordes.}


\subsection{Modelo}

En esta ocasión hemos añadido al villancico ``Deck the Halls'' unos acordes en cifrado americano:

\bigskip

\begin[staffsize=17.5,line-width=17\cm]{lilypond}

<<
\new ChordNames \chordmode { f2 c4:7 f c:7 f c:7 f
f2 c4:7 f bes f/c c:7 f }
\relative c'' { \key f \major
c8. bes16 a8 g
f g a8 f
g16 a bes g a8. g16
f8 e f4
c'8. bes16 a8 g
f g a8( f)
d'16 d d d c8. bes16
a8 g f4
}
\addlyrics { Es -- ta no che~es No -- che -- bue -- na,
	fa la la la la, la la la la.
	Y no~es no  -- che de dor -- mir
	fa la la la la, la la la la. }

>>
\end{lilypond}


\subsection{Contextos de acordes}

El contexto de nombres de acorde se llama ChordNames, y su contenido
debe ir precedido de \verb+\chordmode+ para que se interprete como
acordes. Se escriben las fundamentales de los acordes con sus
duraciones, y si no hay acorde se escribe ''r'' como silencio, así:

\begin[verbatim,relative=3,staffsize=17.5]{lilypond}
<<
  \new ChordNames \chordmode { r4 c2. g4 }
  \new Staff \relative c'' { \time 3/4 \partial 4 g8. g16 a4 g c b }
  \new Lyrics \lyricmode { Cum8. -- ple16 -- a4 -- ños fe -- liz }
>>
\end{lilypond}

Las variantes como séptima dominante se escriben después de los dos
puntos, y las inversiones se indican escribiendo una barra inclinada y
luego la nota del bajo.

\begin[verbatim,relative=3,staffsize=17.5]{lilypond}
<<
  \new ChordNames \chordmode { r4 c4 c/e g:7 }
  \relative c' { \partial 4 e8 f g4 c b8 b r4 }
  \addlyrics { ¿Dón -- de~es -- tán las lla -- ves? }
>>
\end{lilypond}

