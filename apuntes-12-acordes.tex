%\documentclass[12pt,a4paper,oneside]{scrbook} % la clase book del Koma-script bundle
\documentclass[a4paper,10pt,oneside,headinclude,titlepage]{article} % la clase book del Koma-script bundle
%\linespread{1.25}
\usepackage{setspace}
%\usepackage{tikz}
%\usetikzlibrary{fit,shapes}
\usepackage[spanish]{babel}
%\usepackage{verbatim} %para el entorno comment
%\usepackage{moreverb} %para los ejemplos de lilypond, aporta verbatimtabinput
%\usepackage{alltt} %para los ejemplos de lilypond, aporta verbatiminput
%\usepackage{sverb} %para los ejemplos de lilypond, aporta verbinput
%\usepackage{fancyvrb} %para los ejemplos de lilypond, aporta VerbatimInput
\pagestyle{empty}
\usepackage[utf8]{inputenc}
\usepackage[T1]{fontenc} %posiblemente sirva para eliminar el problema del enguionado de palabras acentuadas. Lo quitamos provisionalmente para evitar un error
\usepackage{textcomp} % recomendación de Javier Bezos para completar la fuente

\usepackage[margin=2cm]{geometry}
\usepackage{graphicx}
%\usepackage{url}

\usepackage[utopia]{mathdesign}
%\usepackage{mathptmx} %mejor que Times    % alternativa a Charter


%\typearea[0mm]{13}% same as class options above
%\usepackage{newcent}
%\addtokomafont{part}{\mdseries} %encabezamientos sin negrita
%\addtokomafont{partnumber}{\mdseries} %encabezamientos sin negrita
%\addtokomafont{chapter}{\mdseries} %encabezamientos sin negrita
%\setkomafont{disposition}{\normalcolor\bfseries} %no sans serif
%\setkomafont{disposition}{\normalcolor\mdseries} %no negrita

\parskip=6pt\clubpenalty=10000\widowpenalty=10000

\newcommand{\preLilyPondExample}{\vspace{-10pt}}

\newcommand{\lpversion}{2.13.4}
\newcommand{\defsep}{\textbf{$\|$}}
\newcommand{\software}{\emph{software}}
\newcommand{\negspace}{\vspace{-10pt}}  %{\vspace{-20pt}}
\newcommand{\seppar}{
\bigskip
%\vspace{6pt}
}

%%%%%%%%%%%%%%%%%%%%%%%%%%%%%%%%%%%%%%%%%%%%%%%%%%%%%%%%%%%%%%%%%%%%%%%%%%%%%%%%%%%%%%%%%%%
\begin{document}

\setcounter{section}{11} %para 12 acordes


\section{Esta noche es Nochebuena (2). Acordes.}


\subsection{Modelo}

En esta ocasión hemos añadido al villancico ``Deck the Halls'' unos acordes en cifrado americano:

\bigskip

\begin[staffsize=17.5]{lilypond}

<<
\new ChordNames \chordmode { f2 c4:7 f c:7 f c:7 f
f2 c4:7 f bes f/c c:7 f }
\relative c'' { \key f \major
c8. bes16 a8 g
f g a8 f
g16 a bes g a8. g16
f8 e f4
c'8. bes16 a8 g
f g a8( f)
d'16 d d d c8. bes16
a8 g f4
}
\addlyrics { Es -- ta no che~es No -- che -- bue -- na,
	fa la la la la, la la la la.
	Y no~es no  -- che de dor -- mir
	fa la la la la, la la la la. }
	
>>
\end{lilypond}


\subsection{Contextos de acordes}

El contexto de nombres de acorde se llama ChordNames, y su contenido
debe ir precedido de \verb+\chordmode+ para que se interprete como
acordes. Se escriben las fundamentales de los acordes con sus
duraciones, y si no hay acorde se escribe ''r'' como silencio, así:

\begin[verbatim,relative=3,staffsize=17.5]{lilypond}
<<
  \new ChordNames \chordmode { r4 c2. g4 }
  \new Staff \relative c'' { \time 3/4 \partial 4 g8. g16 a4 g c b }
  \new Lyrics \lyricmode { Cum8. -- ple16 -- a4 -- ños fe -- liz }
>>
\end{lilypond}

Las variantes como séptima dominante se escriben después de los dos
puntos, y las inversiones se indican escribiendo una barra inclinada y
luego la nota del bajo.

\begin[verbatim,relative=3,staffsize=17.5]{lilypond}
<<
  \new ChordNames \chordmode { r4 c4 c/e g:7 }
  \relative c' { \partial 4 e8 f g4 c b8 b r4 }
  \addlyrics { ¿Dón -- de~es -- tán las lla -- ves? }
>>
\end{lilypond}

\end{document}

