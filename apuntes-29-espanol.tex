\section{Nombres de las notas en español}


\subsection{Modelo}

Es posible escribir la música en el lenguaje de LilyPond con los
nombres de las notas en español.  Sin embargo, no lo hemos visto
antes por las siguientes razones:

\begin{itemize}
\item La comunidad de usuarios de LilyPond a nivel global utiliza
  los nombres predeterminados (holandeses) principalmente.
\item Es bueno acostumbrarse a leer y escribir con soltura la
  música en el idioma en que están escritos la mayoría de los
  documentos que circulan entre usuarios de cualquier
  nacionalidad.
\item No es posible copiar y pegar directamente los ejemplos de un
  idioma dentro de un documento que utiliza otro idioma, y no se
  pueden mezclar fácilmente varios idiomas en el mismo documento.
\end{itemize}

A pesar de ello, es posible que algunos usuarios prefieran
escribir los nombres de las notas en su propio idioma, y por ello
lo mencionamos aquí.  El ejemplo que presentamos es el final del
primero de los Intermezzi para piano Op.4 de Schumann y contiene
gran cantidad de expresiones, digitaciones y articulaciones,
polifonía en el pentagrama inferior, notas de pentagrama cruzado y
reguladores textuales; proponemos que se tipografíe utilizando
nombres de nota en español.

\bigskip

% Aumentar la separación entre sistemas
\def\betweenLilyPondSystem#1{\vspace{0.4cm}\linebreak}

\begin[staffsize=15]{lilypond}
\version "2.11.54"

% Schumann, Op.4, I, 11 last measures

%#(set-global-staff-size 18)

rone = \relative c { \override Voice.Fingering #'avoid-slur = #'inside
\oneVoice R2.
R2.
\clef bass \voiceOne cis4(
\once \override DynamicText #'extra-offset = #'(-1.5 . -4.5)
d'^\sf <cis-4>8.. b32) \clef treble
<ais-1-2>4( <g'-5 e-3>\sf  <fis-4 d-1>8.. <e-5 cis-3>32
<d b>8.. <cis a!>32 \clef bass <b -3-5>4 b \clef treble
\change Staff = "LH"
<a e cis>8..)^\ff ( <a d,>32 <a e>8)
\change Staff = "RH"
<a b>_. \p <cis a>_. <d a>_.
<e b>8_.  <fis cis>_.\< <gis d>_. <a e>_. <b fis>_. <b e,>_.
\once \override  DynamicText #'whiteout = ##t
<cis a cis,>8.._([ \ff -4 <d a d,>32 <e a, e>8) <b fis>_. \p <cis gis>_. <d a>_. ]
\oneVoice %\crescTextCresc
<e b>8-.  \cresc <fis cis>-. <gis d>-. <a e>-. <b fis>-. <b e,>-.
\once \override  DynamicText #'whiteout = ##t
<cis-4 a cis,>8..  \ff ( [ <d a d,>32 <e a, e>8) <gis, d>_> ( <a cis,>) <gis d>_> ( ]
<a cis,>8-.) r <a, cis,>4._> r8 \fermata \bar"|."
}

rtwo = \relative c { s2. s2. s4 fis g <fis_1>  s4 s
s4 gis8.. <a fis>32 <gis e>8.. <fis d!>32 }

lone = \relative c, {  cis4 \f (
\once \override DynamicText #'extra-offset = #'(-0.5 . 4)
d' \sf cis8.. b32
\once \override NoteColumn #'force-hshift = #1.5
<a>8.. gis32 fis4 eis)
fis8..( e!32 d4 e
fis8.. gis32 ais4 b8.. cis32
d8.. dis32 e8-.) r \oneVoice e,4
\stemDown
\override Staff.SustainPedalLineSpanner #'Y-extent = #'(0 . 0)
\override Staff.SustainPedalLineSpanner #'staff-padding = #'10
<a a,>8.. (\sustainOn <b b,>32 <cis cis,>8 ) \sustainOff <d d,>-. <e e,>-. <fis fis,>-.
\stemNeutral <gis gis,>8-. <a a,>-. <a b,>-. <a cis,>-. <a d,>-. <gis e>-.
\override Staff.SustainPedalLineSpanner #'staff-padding = #'6
<a a,>8.. ( [ \sustainOn <b b,>32 <cis cis,>8 ) \sustainOff <d d,>-. <e e,>-. <fis fis,>-. ] \clef treble
<gis gis,>-. <a a,>-. <a b,>-. <a cis,>-. <a d,>-. <gis e>-.
<a a,>8.. ( [ \sustainOn <b b,>32 <cis cis,>8 ) e,( \sustainOff <a a,>) e( ]
<a a,>8-. ) \sustainOn r \clef bass <e,, a,>4. r8 \sustainOff \fermata

}

ltwo = \relative c, { R2. cis4(^\markup{\italic "R."} d'^\sf cis8.. b32
a4) a g8.. d'32
cis4.. fis16~ fis4~
fis4
}

common = { \time 3/4 \key a \major }


\new PianoStaff \with { instrumentName="Piano" }<<
\new Staff = "RH" { \common << {\rone} \\ {\rtwo} >>  }
\new Staff = "LH" { \common \clef bass << {\ltwo} \\ {\lone} >> }
>>

\paper {  system-count = 2 
line-width=16.5\cm
indent=1\cm
}

\end{lilypond}


\subsection{Inclusión de archivos de inicio predeterminados}

Ya hemos visto (consulte el apartado \ref{include} en la
pág. \pageref{include}) cómo se incluyen archivos externos en el
documento actual.  Normalmente es necesario especificar la ruta
del archivo si éste no se encuentra en el mismo directorio.  Son
una excepción a esta regla los archivos de inicio predeterminados,
que se leen directamente del directorio de instalación del
programa, y para los que no hay necesidad de especificar una ruta
especial.  Estos archivos contienen ajustes diversos y entre ellos
se encuentran los archivos de idioma: después de incluirlos se
produce un cambio en las definiciones de los nombres de nota de
tal forma que podemos elegir el idioma en que se escribe la
música.  Veamos un ejemplo para el idioma español:

\begin[verbatim]{lilypond}
\include "espanol.ly"
\new Staff \relative do' { \cadenzaOn
  do8[ dos reb re res mib mi fa fas solb sol sols lab la las sib si do]
}
\end{lilypond}

Los nombres de las notas con sostenido se forman añadiendo
\verb+'s'+ y los bemoles añadiendo \verb+'b'+.  Debemos recordar
que en todo lugar en que aparezca un nombre de nota, ya sea dentro
de una instrucción \verb+\relative+, \verb+\transpose+ o
\verb+\key+, entre otras, debe escribirse en el idoma establecido.


\subsection{Pedal de piano}

Las instrucciones \verb+\sustainOn+ y \verb+\sustainOff+ producen
las marcas clásicas del pedal derecho del piano:

\begin[verbatim]{lilypond}
\new PianoStaff <<
  \new Staff { R1*2 }
  \new Staff { \clef bass c1 ~ \sustainOn c \sustainOff }
>>
\end{lilypond}


\subsection{Crescendo de texto}

A partir de la versión 2.13 de LilyPond disponemos de
instrucciones para tipografiar reguladores de texto, como puede
verse aquí:

\begin[verbatim,relative=1]{lilypond}
c16  \p \cresc c c c c c c c c c c c c c c c
c \f \dim c c c c c c c c c c c c c c c \p
\end{lilypond}
