% \version "2.17.0"

\section{\emph{Suite para cello número 1}, de Bach}


\subsection{Modelo}

En este fragmento se utilizan ligaduras de expresión:

\bigskip

\begin[staffsize=17.5,ragged-right]{lilypond}
\version "2.17.0"

\relative g, {

	\clef bass
	\time 4/4
	\key g \major
%	\set Staff.midiInstrument = "cello"

	% 1
	g16(d') b' a b( d, b' d,) g,(d') b' a b( d, b' d,) |
	g,(e') c' b c( e, c' e,) g,(e') c' b c( e, c' e,) |
%	g,(e') c' b c( e, c') e, g,(e') c' b c( e, c' e,) |

}
\end{lilypond}


\subsection{Ligaduras de expresión}

De la misma forma que en el caso de las barras manuales (que se
indican mediante corchetes de manera que no encierran conjuntos de
notas, sino que los corchetes de apertura y cierre marcan las
notas primera y última que pertenecen a la barra), las ligaduras
de expresión se indican mediante paréntesis de apertura y cierre
que marcan por la derecha las notas primera y última de una ligadura de expresión.

\begin[verbatim,relative=1,staffsize=13]{lilypond}
c( d e f g a b c)
\end{lilypond}

