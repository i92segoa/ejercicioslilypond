%\documentclass[12pt,a4paper,oneside]{scrbook} % la clase book del Koma-script bundle
\documentclass[a4paper,10pt,oneside,headinclude,titlepage]{article} % la clase book del Koma-script bundle
%\linespread{1.25}
\usepackage{setspace}
%\usepackage{tikz}
%\usetikzlibrary{fit,shapes}
\usepackage[spanish]{babel}
%\usepackage{verbatim} %para el entorno comment
%\usepackage{moreverb} %para los ejemplos de lilypond, aporta verbatimtabinput
%\usepackage{alltt} %para los ejemplos de lilypond, aporta verbatiminput
%\usepackage{sverb} %para los ejemplos de lilypond, aporta verbinput
%\usepackage{fancyvrb} %para los ejemplos de lilypond, aporta VerbatimInput
\pagestyle{empty}
\usepackage[utf8]{inputenc}
\usepackage[T1]{fontenc} %posiblemente sirva para eliminar el problema del enguionado de palabras acentuadas. Lo quitamos provisionalmente para evitar un error
\usepackage{textcomp} % recomendación de Javier Bezos para completar la fuente

\usepackage[margin=2cm]{geometry}
\usepackage{graphicx}
%\usepackage{url}

\usepackage[utopia]{mathdesign}
%\usepackage{mathptmx} %mejor que Times    % alternativa a Charter


%\typearea[0mm]{13}% same as class options above
%\usepackage{newcent}
%\addtokomafont{part}{\mdseries} %encabezamientos sin negrita
%\addtokomafont{partnumber}{\mdseries} %encabezamientos sin negrita
%\addtokomafont{chapter}{\mdseries} %encabezamientos sin negrita
%\setkomafont{disposition}{\normalcolor\bfseries} %no sans serif
%\setkomafont{disposition}{\normalcolor\mdseries} %no negrita

\parskip=6pt\clubpenalty=10000\widowpenalty=10000

\newcommand{\preLilyPondExample}{\vspace{-10pt}}

\newcommand{\lpversion}{2.13.4}
\newcommand{\defsep}{\textbf{$\|$}}
\newcommand{\software}{\emph{software}}
\newcommand{\negspace}{\vspace{-10pt}}  %{\vspace{-20pt}}
\newcommand{\seppar}{
\bigskip
%\vspace{6pt}
}

%%%%%%%%%%%%%%%%%%%%%%%%%%%%%%%%%%%%%%%%%%%%%%%%%%%%%%%%%%%%%%%%%%%%%%%%%%%%%%%%%%%%%%%%%%%
\begin{document}

\setcounter{section}{5} %para 06 ligaduras


\section{``Suite para cello número 1'', de Bach}


\subsection{Modelo}

En este fragmento se utilizan ligaduras de expresión:

\bigskip

\begin[staffsize=17.5,no-ragged-right]{lilypond}
\version "2.11.63"

\relative g, {

	\clef bass
	\time 4/4
	\key g \major
%	\set Staff.midiInstrument = "cello"

	% 1
	g16(d') b' a b( d, b' d,) g,(d') b' a b( d, b' d,) |
	g,(e') c' b c( e, c' e,) g,(e') c' b c( e, c' e,) |
%	g,(e') c' b c( e, c') e, g,(e') c' b c( e, c' e,) |

}
\end{lilypond}


\subsection{Ligaduras de expresión}

De la misma forma que en el caso de las barras manuales (que se
indican mediante corchetes de manera que no encierran conjuntos de
notas, sino que los corchetes de apertura y cierre marcan las
notas primera y última que pertenecen a la barra), las ligaduras
de expresión se indican mediante paréntesis de apertura y cierre
que marcan por la derecha las notas primera y última de una ligadura de expresión.

\begin[verbatim,relative=1,staffsize=13]{lilypond}
c( d e f g a b c)
\end{lilypond}


\end{document}

