\section{Más sobre transcripción de música antigua}

En los apartados \ref{machaut} y \ref{midi} (véanse
págs. \pageref{machaut} y \pageref{midi}) hemos hablado de música
coral antigua. Completamos aquí el tema con algunos aspectos usuales
de las partituras de transcripciones.

La forma más rápida de preparar una partitura polifónica es tomar una
plantilla (un documento completo con la mínima cantidad posible de
música) y definir las variables que contienen la música de cada
pentagrama y los textos.

A la hora de componer tipográficamente otros elementos propios del
estilo de partitura que nos ocupa, lo haremos como puede verse en los
ejemplos de las secciones siguientes.

\subsection{Corchetes de ligadura}

Las ligaduras de la notación mensural blanca se indican de la forma
que se ve a continuación:

\begin{lilypond}[relative=3,fragment,verbatim,staffsize=15]
 \[ a1 f \]
\end{lilypond}

Observamos que, a diferencia de los paréntesis de ligadura de
expresión, que marcan las notas de inicio y fin \emph{por la derecha},
los corchetes de ligadura \emph{encierran conjuntos de notas}.

\subsection{Alteraciones sugeridas o sobreentendidas}
No debemos olvidar que las alteraciones sugeridas que se colocan sobre
las notas, son verdaderas alteraciones y como tales admiten el
transporte sin ningún problema.

Para colocar la alteración sobre la nota, especificamos la nota con su
alteración de la forma habitual e indicamos que se sitúen arriba
mediante una instrucción que altera una propiedad del contexto
predeterminado, que es \verb+Voice+:

\begin[fragment,relative=1,verbatim,staffsize=15]{lilypond}
\set suggestAccidentals = ##t
 fis1
\end{lilypond}

\noindent Si se va a utilizar esta instrucción varias veces para notas
sueltas, es más práctico definir una variable:

\begin[verbatim,staffsize=15]{lilypond}
ss = \once \set suggestAccidentals = ##t
{ \ss fis'1
  \ss f'!1
  \ss es' }
\end{lilypond}

\subsection{Alteraciones de cortesía}
El signo de interrogación, utilizado como sufijo, produce los
paréntesis de la alteración:
\begin[verbatim,relative=2,fragment,staffsize=15]{lilypond}
a?1
\end{lilypond}

\subsection{Breve y longa}
Estas figuras duran, respectivamente, 2 y 4 redondas.
\begin[verbatim,relative=2,fragment,staffsize=15]{lilypond}
a\breve
a\longa
\end{lilypond}

\noindent Para que ocupen un solo compás aplicamos un factor:
\begin[verbatim,relative=2,fragment,staffsize=15]{lilypond}
a\breve*1/2
a\longa*1/4
\end{lilypond}

\noindent En el estilo barroco alternativo disponemos de la forma
cuadrada para la breve y la longa, así como de la máxima:

\begin[verbatim,relative=2,fragment,staffsize=15]{lilypond}
\override NoteHead #'style = #'baroque
a\breve*1/2
a\longa*1/4
a\maxima*1/8
\end{lilypond}

\subsection{Línea extensora}
Se usa en las palabras cuya última sílaba es un melisma o una nota
prolongada. Dos guiones bajos producen una línea extensora hasta el
final de la nota actual.

\begin[verbatim,staffsize=15]{lilypond}
{ \clef bass
f2 f ~
f e }
\addlyrics { -rat in __
prin- }
\end{lilypond}

\subsection{Letra en itálica}
Una vez más definimos nuestras propias variables para usarlas como
instrucciones.

\begin[verbatim,staffsize=15]{lilypond}
italicas = \override LyricText #'font-shape = #'italic
rectas =   \override LyricText #'font-shape = #'upright
{ \clef bass
c2 d4 d
c4 c'8 c' f4 f
c1 }
\addlyrics { in -- sae -- cu -- \italicas "[la," et] \rectas in sae -- cu -- la }
\end{lilypond}


\subsection{Entonaciones gregorianas}
Desde aquí no podemos recomendar la transcripción de las entonaciones
gregorianas con notas cuadradas, a no ser que sea verdadera notación
neumática vaticana.  LilyPond es capaz de realizar este tipo de
notación pero presenta algunos problemas de espaciado.  En cambio, la
mejor y más legible transcripción del canto gregoriano se efectúa con
figuras blancas y negras sin plica.  Incluyendo un archivo especial de
ajustes predefinidos podemos disponer de divisiones gregorianas.

\begin[verbatim,staffsize=17.5]{lilypond}
\include "gregorian.ly"
\score {
    \relative f {
      a'4( bes) a( g) a2( d,) \divisioMinima 
      f4( g a bes a g f e d) c d d2 \finalis
    }
    \addlyrics { Ký -- ri -- e  "*e" -- lé -- i -- son. }

  \layout {
    \context {
      \Staff
      \remove "Time_signature_engraver"
      \remove "Bar_engraver"
      \override Stem #'transparent = ##t
      \override Stem #'length = #0
    }
  }
}
\end{lilypond}


\subsection{Modos antiguos y transposición}
Si estamos transcribiendo una pieza para editarla en otro tono, se
recomienda transcribir el texto original con las mismas alturas, que
es más fácil de revisar. Después, es inmediato subirlo o bajarlo el
intervalo que deseemos.

No solo están contempladas las tonalidades mayores y menores, sino el
conjunto completo de armaduras de los siete modos.

Supongamos que hemos transcrito lo siguiente, que está en Sol dórico:

\begin[verbatim,staffsize=15]{lilypond}
\new Staff{ \key g \dorian
r2 g' ~ 
g' \[d'' ~ 
d'' e'' ~ \] e''4 }
\end{lilypond}

y lo queremos en Fa dórico (un tono por debajo). Para ello
especificamos un transporte de Sol a Fa.
\begin[verbatim,staffsize=15]{lilypond}
\transpose g f
\new Staff{ \key g \dorian
r2 g' ~ 
g' \[d'' ~ 
d'' e'' ~ \] e''4 }
\end{lilypond}

Una única instrucción \verb+\transpose+ es suficiente para toda la
expresión que le sigue, aunque sea la expresión de partitura de la
pieza entera.
